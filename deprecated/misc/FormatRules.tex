The following rules are applied to the formatting of the text

Special Phrases
1) all special phrases appear in the files

phrase_macros.tex
dedup_phrase_macros.tex

2) In the all text
all special phrases are expressed with latex macros
they are  always capitalized (execept for a very specical cases)

the first instance of the phrase is defined as the first time it appears
in the text, not including the abstract or the table of contents

The first instance of the phrase is to place in an emph{} tag
and, if appropriate to be followed by the associated acroynym, in parenthesis

For example, the special phrase ``Empirical Spectral Density''
forst appears in the main text in the latex file:  210_htsr_setup.tex 
in the subsection ``The \HTSR Setup''

is expressed in the latex file with the macro \EmpiricalSpectralDensity
It appears in the abstract as Empirical Spectral Density (ESD)

The first time it appears in the main text, after the abstract,
the latex is  \emph{\EmpiricalSpectralDensity} (ESD)
it appears in the text as italized

3) In The table of contents , if a special phrase appears in a section,  subsection, or subsubection title
it is NOT emphasized BUT it is followed by the acroynym macro in parenthesis

For example, for subsection 1.2, the latex is
\subsection{Heavy-Tailed Self-Regularization (\HTSR)}


4) If a mathematical symbol appears in the table of contents, or in
a section, subsection, or subsubsection title, then it is placed in parenthesiss

For example, file A64_tanaka.tex, the subsubsection title appears as:
\subsubsection{Expressing the \GEN $(\GNORM(\lambda))$ as the Integrated \RTransform $(R(z)$ of the \CorrelationMatrix $(\AMAT)$}
The 3 math symbols as: \GNORM(\lambda), R(z), and \AMAT





