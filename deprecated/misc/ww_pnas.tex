\documentclass[9pt,twocolumn,twoside,lineno]{pnas-new}
% Use the lineno option to display guide line numbers if required.

\templatetype{pnasresearcharticle} % Choose template 
% {pnasresearcharticle} = Template for a two-column research article
% {pnasmathematics} %= Template for a one-column mathematics article
% {pnasinvited} %= Template for a PNAS invited submission

\usepackage[utf8]{inputenc} % allow utf-8 input                                                                                                        
\usepackage[T1]{fontenc}    % use 8-bit T1 fonts                                                                                                       
\usepackage{hyperref}       % hyperlinks                                                                                                               
\usepackage{url}            % simple URL typesetting                                                                                                   
\usepackage{booktabs}       % professional-quality tables                                                                                              
\usepackage{amsfonts}       % blackboard math symbols                                                                                                  
\usepackage{nicefrac}       % compact symbols for 1/2, etc.                                                                                            
\usepackage{microtype}      % microtypography                                                                                                          
\usepackage{xcolor}         % colors                                                                                                                   

\usepackage{amsmath}
\usepackage{graphicx}                                                                                                                              
\usepackage{enumitem}                                                                                                               
\usepackage{subfigure}                                                                                                                          
\usepackage{makecell}
\usepackage{soul}

\usepackage{wrapfig}

\usepackage{hyperref}

\usepackage{amsmath}

\hypersetup{
     colorlinks   = true,
     linkcolor    = blue,
     citecolor    = green
}
%SumInt symbol
\usepackage{MnSymbol}

\newcommand{\fix}[1]{\textcolor{red}{#1}}
\newcommand{\comment}[1]{\textcolor{blue}{#1}}
\newcommand{\awk}[1]{\textcolor{green}{#1}}

\newcommand{\argmin}{\text{argmin}}
\newcommand{\Probab}[1]{\mbox{}{\bf{Pr}}\left[#1\right]}
\newcommand{\Expect}[1]{\mbox{}{\bf{E}}\left[#1\right]}
\newcommand{\ExpectBracket}[1]{\mbox{}\langle#1\rangle}
%\newcommand{\Trace}[1]{\mbox{}{\mbox{Tr}}\left[#1\right]}
\newcommand{\Trace}[1]{\mbox{}\operatorname{Tr}\left[#1\right]}

% can pick bracket or mathbbf, with subscript 
%\newcommand{\Expected}[2][]{\mathbb{E}_{#1}\left[#2\right]}
\newcommand{\Expected}[2][]{\left\langle #2 \right\rangle_{#1}}

\newcommand{\LAMBDAPL}{\lambda^{PL}_{min}}
\newcommand{\LAMBDADETX}{\lambda^{\vert detX\vert=1}_{min}}

% Here are two macros for comments.                                                                                                                    
\newcommand{\red}[1]{{\color{red}\sf{#1}}}
\newcommand{\cyan}[1]{{\color{cyan}\sf{#1}}}
\newcommand{\nred}[1]{{\color{red}\sf{[#1]}}}
\newcommand{\ngreen}[1]{{\color{green}\sf{[#1]}}}
\newcommand{\ncyan}[1]{{\color{cyan}\sf{[#1]}}}
\newcommand{\nmove}[1]{{\color{brown}\sf{[#1]}}}




\definecolor{darkgreen}{rgb}{0.05, 0.65, 0.06}
%\newcommand {\charles}[1]{{\color{blue}\sf{[charles: #1]}}}
\newcommand {\charles}[1]{}

%\newcommand {\michael}[1]{{\color{red}\sf{[michael: #1]}}}
%\newcommand {\michaeladdressed}[1]{{\color{brown}\sf{[michaeladdressed: #1]}}}
\newcommand {\michael}[1]{}
\newcommand {\michaeladdressed}[1]{}

%\newcommand {\cformike}[1]{{\color{teal}\sf{[charles: #1]}}}
\newcommand {\cformike}[1]{}

%\newcommand {\chris}[1]{{\color{darkgreen}\sf{[chris: #1]}}}
\newcommand {\chris}[1]{}
%\newcommand {\serena}[1]{{\color{orange}\sf{[charles: #1]}}}

\usepackage[normalem]{ulem}

% MM: xspace may have some issues - do a search on ``space after newcommand latex'' - there is no perfect solution 

\usepackage{xspace}
\newcommand{\TR}{\top}

\newcommand{\SPECTRALNORM}{\texttt{SpectralNorm}\xspace}
\newcommand{\LOGSPECTRALNORM}{\texttt{LogSpectralNorm}\xspace}

\newcommand{\FROBENIUSNORM}{\texttt{LogFrobeniusNorm}\xspace}
\newcommand{\ALPHA}{\texttt{Alpha}\xspace}
\newcommand{\QUALITYOFALPHAFIT}{\texttt{QualityOfAlphaFit}\xspace}
\newcommand{\ALPHAHAT}{\texttt{AlphaHat}\xspace}
\newcommand{\ALPHASHATTENNORM}{\texttt{LogAlphaShattenNorm}\xspace}
\newcommand{\DISTANCEFROMINIT}{\texttt{DistanceFromInit}\xspace}
\newcommand{\TRAININGACCURACY}{\texttt{TrainingAccuracy}\xspace}
\newcommand{\SHARPNESS}{\texttt{Sharpness}\xspace}
\newcommand{\SVDSHARPNESS}{\texttt{SVDSharpness}\xspace}
\newcommand{\SVDSMOOTHING}{\texttt{SVDSmoothing}\xspace}
\newcommand{\TASKONE}{\texttt{Task1}\xspace}
\newcommand{\TASKTWO}{\texttt{Task2}\xspace}
\newcommand{\BASELINE}{\texttt{Baseline}\xspace}

\newcommand{\SVDA}{\texttt{SVD20}\xspace}
\newcommand{\SVDB}{\texttt{SVD40}\xspace}


% weightwatcher metrics
\newcommand{\AVG}{\texttt{Avg.}}
\newcommand{\AVGALPHA}{\langle\alpha\rangle}
\newcommand{\ALPHAHATEQN}{\hat\alpha}
\newcommand{\ALPHAHATLONG}{\alpha\log_{10}\lambda_{max}}
\newcommand{\MAXEVAL}{\lambda_{max}}
\newcommand{\AVGALPHADISTANCE}{\langle D_{KS}\rangle}
\newcommand{\AVGLOGSPECTRALNORM}{\langle\log_{10}\Vert\mathbf{W}\Vert^{2}_{2}\rangle}
\newcommand{\AVGLOGNORM}{\langle\log_{10}\Vert\mathbf{W}\Vert^{2}_{F}\rangle}
\newcommand{\AVGLOGSHATTENNORM}{\langle\log_{10}\Vert\mathbf{W}\Vert^{2\alpha}_{2\alpha}\rangle}
\newcommand{\LAMBDA}{\texttt{TPL-Lambda}}
\newcommand{\RANDDIST}{\texttt{Rand-Distance}}
\newcommand{\INITDIST}{\texttt{Init-Distance}}
\newcommand{\WW}{\texttt{WeightWatcher}\xspace}

\newcommand{\HCIZtext}{\texttt{HCIZ}\xspace}
\newcommand{\SLT}{\texttt{SLT}\xspace}
\newcommand{\RMT}{\texttt{RMT}\xspace}
\newcommand{\HTRMT}{\texttt{HTRMT}\xspace}
\newcommand{\STATMECH}{\texttt{StatMech}\xspace}
\newcommand{\SETOL}{\texttt{SETOL}\xspace}
\newcommand{\SMOG}{\texttt{SMOG}\xspace}
\newcommand{\HTSR}{\texttt{HTSR}\xspace}
\newcommand{\QUANTFIN}{\texttt{QuantFin}\xspace}
\newcommand{\SEMIEMP}{\texttt{SemiEmpirical}\xspace}

\newcommand{\ETPL}{\texttt{E\_TPL}}
\newcommand{\TPL}{\texttt{TPL}}
\newcommand{\PL}{\texttt{PL}}
\newcommand{\PLKS}{\texttt{PL KS}}
\newcommand{\MPSOFTRANK}{\texttt{MP SoftRank}}

\newcommand{\LAMBDAMIN}{\texttt{LAMBDA-MIN}\xspace}
\newcommand{\TRACELOG}{\texttt{TRACE-LOG}\xspace}
\newcommand{\IFA}{\texttt{IFA}\xspace}
\newcommand{\ECS}{\texttt{ECS}\xspace}

\newcommand{\PRETFIDF}{\texttt{PRE TFIDF}}
\newcommand{\TFIDF}{\texttt{TFIDF}}
\newcommand{\POWERLAW}{\texttt{PowerLaw}}

\newcommand{\SHAPE}{\emph{Shape}\xspace}
\newcommand{\SCALE}{\emph{Scale}\xspace}

% IZ Free Energy 
\newcommand{\IZFE}{\beta\mathbf{F}^{IZ}}
% IZ Parition Function
\newcommand{\IZZ}{\beta\mathbf{Z}^{IZ}}



% Average Layer Quality
\newcommand{\Q}{\bar{\mathcal{Q}}}
% Average Layer Quality (Squared
\newcommand{\QT}{\bar{\mathcal{Q}}^{2}}

% Generating Functions
\newcommand{\STG}{\beta\mathbf{\Gamma}^{ST}_{\Q}}
\newcommand{\IZG}{\beta\mathbf{\Gamma}^{IZ}_{\QT}}
\newcommand{\IZGINF}{\beta\mathbf{\Gamma}^{IZ}_{\QT,N\gg 1}}


% Tanaka's (Norm) Generat\ing Functons / Generalized Norms
\newcommand{\GN}{\mathcal{G(\lambda)}}
\newcommand{\GNI}{\mathcal{G}(\lambda_{i})}
\newcommand{\GNECSI}{\mathcal{G}(\LambdaECS_{i})}
%\newcommand{\GNECS}{\mathcal{G(\tilde{\lambda})}}
%\newcommand{\GEN}{Generalized Norm}
\newcommand{\GEN}{Norm Generating Function}
%\newcommand{\EPSL}{\epsilon_\mathcal{L}}




\newcommand{\OVERLAP}{\mathbf{R}}
\newcommand{\OLAPTOLAP}{{\OVERLAP}^{\TR}\OVERLAP}
\newcommand{\OLAPSQD}{\Trace{\mathbf{R}^{\TR}\mathbf{R}}}
\newcommand{\THRMAVGIZ}[1]{\left\langle #1 \right\rangle_{\mathbf S}^{\beta}}
\newcommand{\THRMAVG}[1]{\left\langle #1 \right\rangle_{\mathbf s}^{\beta}}
\newcommand{\THRMAVGw}[1]{\left\langle #1 \right\rangle_{\mathbf w}^{\beta}}
\newcommand{\HCIZAVG}[1]{\left\langle #1 \right\rangle_{\mathbf S}^{IZ}}
\newcommand{\ZQ}{Z_{\mathcal{Q}}^{IZ}}

% ECS superscript--deprecated
\newcommand{\EFF}{ecs}
% HCIZ Tanaka style avarage:  uses \AECS, below
%\newcommand{\HCIZ}{\mathbb{E}_{\tilde{\mathbf{A}}}[\cdots]}
%\newcommand{\HCIZ}{\mathbb{Z}^{IZ}(\mathbf{A})}
\newcommand{\HCIZ}{\mathbb{Z}^{IZ}}
%
\newcommand{\NDX}{\mathbf{x}^{n}}

% model training and generalization error
% averagess
\newcommand{\AVGE}{\bar{\mathcal{E}}}
\newcommand{\AVGTE}{\bar{\mathcal{E}}_{train}}
\newcommand{\AVGGE}{\bar{\mathcal{E}}_{gen}}

%  duplicates from below--need to clean up
\newcommand{\MTE}{\bar{\mathcal{E}}^{M}_{train}}
\newcommand{\MGE}{\bar{\mathcal{E}}^{M}_{gen}}

% totals
\newcommand{\TTE}{{\mathcal{E}}_{train}}
\newcommand{\TGE}{{\mathcal{E}}_{gen}}

% student-teacher training and generalization error
\newcommand{\STTE}{{\mathcal{E}}^{ST}_{train}}
\newcommand{\STGE}{{\mathcal{E}}^{ST}_{gen}}
\newcommand{\AVGSTTE}{\bar{\mathcal{E}}^{ST}_{train}}
\newcommand{\AVGSTGE}{\bar{\mathcal{E}}^{ST}_{gen}}

% neural network training and generalization error      
\newcommand{\NNTE}{{\mathcal{E}}^{NN}_{train}}
\newcommand{\NNGE}{{\mathcal{E}}^{NN}_{gen}}
\newcommand{\AVGNNTE}{\bar{\mathcal{E}}^{NN}_{train}}
\newcommand{\AVGNNGE}{\bar{\mathcal{E}}^{NN}_{gen}}

% empirical training and generalization error      
\newcommand{\EMPTE}{{\mathcal{E}}^{emp}_{train}}
\newcommand{\EMPGE}{{\mathcal{E}}^{emp}_{gen}}
\newcommand{\AVGEMPTE}{\bar{\mathcal{E}}^{emp}_{train}}
\newcommand{\AVGEMPGE}{\bar{\mathcal{E}}^{emp}_{gen}}

%actual data

%  data variables
\newcommand{\Ntrain}{N^{train}}
\newcommand{\Ntest}{N^{test}}
\newcommand{\DX}{\mathbf{x}}
\newcommand{\DXtrain}{\mathbf{x}^{train}}
\newcommand{\DXtest}{\mathbf{x}^{test}}
\newcommand{\DATA}{\mathbf{x}_{\mu}}
\newcommand{\DATAtrain}{\mathbf{x}_{\mu}^{train}}
\newcommand{\DATAtest}{\mathbf{x}_{\mu}^{test}}

% teacher and student labels
\newcommand{\MY}{\mathrm{y}}

\newcommand{\Ys}{\MY_{\mu}^{S}}
\newcommand{\Yt}{\MY_{\mu}^{T}}
\newcommand{\Ytrue}{\MY_{\mu}^{true}}
\newcommand{\Ytrain}{\MY_{\mu}^{train}}
\newcommand{\Ytest}{\MY_{\mu}^{test}}
\newcommand{\YsVEC}{\mathbf{y}^{S}}
\newcommand{\YtVEC}{\mathbf{y}^{T}}
\newcommand{\Ymu}{\MY_{\mu}}

%model Gaussian data
\newcommand{\XI}{\boldsymbol{\xi}}
\newcommand{\XImu}{\boldsymbol{\xi}_{\mu}}
\newcommand{\NDXI}{\boldsymbol{\xi}^{N}}
\newcommand{\NDXIn}{\boldsymbol{\xi}^{n}}
\newcommand{\NDXIinfty}{\boldsymbol{\xi}^{\infty}}

\newcommand{\XItest}{\boldsymbol{\xi}^{test}}
\newcommand{\XItrain}{\boldsymbol{\xi}^{train}}
\newcommand{\XImutest}{\boldsymbol{\xi}_{\mu}^{test}}
\newcommand{\XImutrain}{\boldsymbol{\xi}_{\mu}^{train}}
\newcommand{\NDXItest}{[\boldsymbol{\xi}^{test}]^{N}}
\newcommand{\NDXItrain}{[\boldsymbol{\xi}^{train}]^{N}}

% Perceptron vectors
\newcommand{\SVEC}{{\mathbf{s}}}
\newcommand{\TVEC}{{\mathbf{t}}}
\newcommand{\WVEC}{{\mathbf{w}}}
\newcommand{\XVEC}{{\mathbf{x}}}
\newcommand{\YVEC}{{\mathbf{y}}}

% Data Averaged ST Error: uses \SVEC, \TVEC, etc
\newcommand{\EPSL}{\epsilon}
\newcommand{\EPSLR}{\epsilon(R)}
% Total and  Data-Averaged ST Error: explicit:  uses other defintsions
\newcommand{\EPSLSTw}{\epsilon(\WVEC)}
\newcommand{\EPSLw}{\epsilon(\WVEC)}
\newcommand{\ETOTSTx}{\mathcal{E}(\SVEC, \TVEC )}
\newcommand{\EPSLSTx}{\epsilon(\SVEC, \TVEC )}
\newcommand{\ETOTST}{\mathcal{E}(\SVEC, \TVEC )}
\newcommand{\EPSLST}{\epsilon(\SVEC, \TVEC )}

%  Data Averaged NN Error: explicit \SMAT, \TMAT, \NDXI
\newcommand{\ETOTNN}{\mathcal{E}(\mathbf{S}, \mathbf{T}| \NDXI)}
\newcommand{\EPSLNN}{\epsilon(\mathbf{S}, \mathbf{T})}

% Data Averaged ST Error (old version)
% deprecated
\newcommand{\DEEPS}{\epsilon(\mathbf{w})}
%\newcommand{\DEEPSX}{\epsilon(\mathbf{w}|\mathbf{x})}

\newcommand{\DET}{\Delta E_{\mathcal{L}}(\mathbf{w},\XI)}
\newcommand{\DETeff}{\Delta E^{eff}_{\mathcal{L}}(\mathbf{w},\XI)}
\newcommand{\DETST}{\Delta E_{\mathcal{L}}(\SVEC,\TVEC,\XI)}
\newcommand{\DETSTL}{\Delta E_{\mathcal{L}}(\SVEC,\TVEC,\XI)}
\newcommand{\DETOPSTL}{\mathbf{\Delta {E}}_{\mathcal{L}}(\SVEC,\TVEC,\XI)}
\newcommand{\DETOPSTx}{\mathbf{\Delta {E}}_{\mathcal{L}}(S,T,\DX)}
\newcommand{\DETSTLL}{\Delta E_{\ell_2}(\SVEC,\TVEC,\XI)}
\newcommand{\DETOP}{\mathbf{\Delta {E}}_{\mathcal{L}}(\mathbf{w})}
\newcommand{\DETOPX}{\mathbf{\Delta {E}}_{\mathcal{L}}(\mathbf{w}, \NDX)}
\newcommand{\DETOPXY}{\mathbf{\Delta {E}}_{\mathcal{L}}(\mathbf{w}, \NDX, \MY^{n})}
\newcommand{\DETOPXI}{\mathbf{\Delta {E}}_{\mathcal{L}}(\mathbf{w}, \XI^{n})}
\newcommand{\DETOPXILL}{\mathbf{\Delta {E}}_{\ell_2}(\mathbf{w}, \XI)^{n}}
\newcommand{\DETOPSTLL}{\mathbf{\Delta {E}}_{\ell_2}(\SVEC,\TVEC, \XI^{n})}
\newcommand{\DETOPST}{\mathbf{\Delta {E}}_{\ell_2}(\SMAT,\TMAT)}
\newcommand{\DETOPNN}{\mathbf{\Delta {E}}_{\ell_2}(\SMAT,\TMAT, \XI)}
\newcommand{\DETOPNNXI}{\mathbf{\Delta {E}}_{\ell_2}(\SMAT,\TMAT|\XI^{n})}
\newcommand{\DETOT}{\mathcal{E}(\mathbf{w})}
\newcommand{\DETOTXI}{\mathcal{E}(\mathbf{w}|\NDX)}
%deprecated
\newcommand{\DETOTX}{\mathcal{E}(\mathbf{w}|\mathbf{x}^{n})}
\newcommand{\DETOTXY}{\mathcal{E}(\mathbf{w}|\mathbf{x}^{n}, y^{n})}

% Data set size
\newcommand{\ND}{N}}

% Matrices
\newcommand{\SMAT}{{\mathbf{S}}}
\newcommand{\TMAT}{{\mathbf{T}}}
\newcommand{\WMAT}{{\mathbf{W}}}
\newcommand{\AECS}{\tilde{\mathbf{A}}}
\newcommand{\WECS}{\tilde{\mathbf{W}}}
\newcommand{\XECS}{\tilde{\mathbf{X}}}
\newcommand{\TECS}{\tilde{\mathbf{T}}}
\newcommand{\LambdaECS}{\tilde{\lambda}}
\newcommand{\LambdaECSmin}{\lambda^{ECS}_{min}}
\newcommand{\LambdaPLmin}{\lambda^{PL}_{min}}

\newcommand{\MECS}{\tilde{M}}
\newcommand{\kECS}{\tilde{k}}

\newcommand{\AMAT}{\mathbf{A}}
\newcommand{\AHAT}{\hat{\mathbf{A}}}

\newcommand{\BMAT}{\mathbf{B}}
\newcommand{\XMAT}{\mathbf{X}}
\newcommand{\DMAT}{\mathbf{D}}

%Annealed Hamiltonian
\newcommand{\GAN}{H^{an}(\WVEC)}
\newcommand{\HAN}{H^{an}}
\newcommand{\GANR}{H^{an}(R)}
\newcommand{\GANMAT}{H^{an}(\mathbf{R})}
\newcommand{\GANHT}{H^{an}_{hT}(\WVEC)}
\newcommand{\GANHTR}{H^{an}_{hT}(R)}
\newcommand{\GANMATHT}{H^{an}_{hT}(\mathbf{R})}
\newcommand{\HANHT}{H^{an}_{hT}}
\newcommand{\HEFF}{\mathbf{H}^{ECS}_{\QT}}
\newcommand{\HBARE}{\mathbf{H}_{\QT}}
\newcommand{\HANPP}{h^{an}(\OVERLAP)}
\newcommand{\HANPPHT}{H^{an}_{hT}(\OVERLAP)}

%Data Distributions
\newcommand{\ADD}{\mathbf{D}}
\newcommand{\MDD}{\mathcal{D}}

\newcommand{\XINORM}{\mathcal{N}}

% High T Annealed Approximation
\newcommand{\ZAN}{Z^{an}_{n}}
\newcommand{\ZHT}{Z^{hT}_{n}}
\newcommand{\ZANHT}{Z^{an,hT}_{n}}

% Hamiltonian
\newcommand{\HH}{\mathbf{H}}

% Replica averages 
\newcommand{\REPLICA}{[\mathbf{\XI}^{n}]_{r}}
\newcommand{\DEREPLICA}{\mathcal{E}(\mathbf{w}|[\mathbf{\XI}^{n}]_{r})}

% TraceLog Integrals and Normalizaton
\newcommand{\INTA}{\int_{\mathbf{A}}}
\newcommand{\INTS}{\int_{\SMAT}}
\newcommand{\INTsvec}{\int_{\SVEC}}
\newcommand{\INTAHAT}{\int_{\AHAT}}
\newcommand{\INTiAHAT}{\int_{i\AHAT}}

\newcommand{\XHAT}{\hat{\mathbf{X}}}

\newcommand{\NORM}{\mathcal{N}}

\newcommand{\GNORM}{\mathbb{G}_{\AMAT}}
\newcommand{\GMAX}{\mathcal{G}^{max}}
\newcommand{\GFANCY}{\mathcal{G}(\XMAT)}
\newcommand{\ZIZ}{\mathbb{Z}^{IZ}}
\newcommand{\ZD}{\mathbb{Z}^{IZ}}
%\newcommand{\ZD}{\mathbb{Z}(\mathbf{D})}
\newcommand{\EZDA}{\mathbb{E}_{\AMAT}[\ZIZ]}
\newcommand{\EZDX}{\mathbb{E}_{\XMAT}[\ZIZ]}
\newcommand{\EZDW}{\mathbb{E}_{\WMAT}[\ZIZ]}
\newcommand{\EZDAONE}{\mathbb{E}_{\mathbf{A}_1}[\ZIZ]}
\newcommand{\EZDATWO}{\mathbb{E}_{\mathbf{A}_2}[\ZIZ]}

\newcommand{\EQN}{Eqn.}
\newcommand{\Det}[1]{\mbox{}{\text{det}}\left(#1\right)}
%\newcommand{\DeltaMu}{\delta_{\mu}}
\newcommand{\DeltaMu}{\vartheta_{\mu}}
\newcommand{\IM}{\mathbf{I}_{M}} % identity operator
\newcommand{\IH}{\mathbb{I}_{H}} % integral
\newcommand{\DETX}{\texttt{DetX}\xspace}
\newcommand{\ETA}{\eta\xspace}
\newcommand{\ETAw}{\eta(\WVEC)\xspace}
%\newcommand{\ETAMLP}{\eta(\XI)_{MLP3}\xspace}
\newcommand{\ETAMLPXI}{\eta([\SMAT_l,\TMAT_l],\XI)\xspace}
\newcommand{\ETAMLPAVG}{\langle\eta([\SMAT_l,\TMAT_l],\XI)\rangle_{\XI}\xspace}
\newcommand{\ETAMLP}{\eta([\SMAT_l,\TMAT_l])\xspace}
\newcommand{\ETAMAT}{\eta(\SMAT_l,\TMAT_l,\XI)\xspace}
\newcommand{\ETAMATAVG}{\langle\eta(\SMAT_l,\TMAT_l,\XI)\rangle_{\XI}\xspace}



\title{SETOL: A Semi-Empirical Theory of Learning}

% Use letters for affiliations, numbers to show equal authorship (if applicable) and to indicate the corresponding author
\author[a,1]{Charles H. Martin}
\author[b,2]{Michael W. Mahoney} 

\affil[a]{Calculation Consulting}
\affil[b]{ICSI, LBNL, and Department of Statistics at UC Berkeley}


% Please give the surname of the lead author for the running footer
\leadauthor{Martin} 

% Please add a significance statement to explain the relevance of your work
%\significancestatement{Authors must submit a 120-word maximum statement about the significance of their research paper written at a level understandable to an undergraduate educated scientist outside their field of speciality. The primary goal of the significance statement is to explain the relevance of the work in broad context to a broad readership. The significance statement appears in the paper itself and is required for all research papers.}
\significancestatement{
Deep Neural Networks (DNNS) have emerged as one of the most powerful modeling methods in science, engineering, and a wide range of other applications, including computer vision (CV) and natural language processing (NLP).
Despite this, there is no working theory which can explain \emph{why deep learning works}.
Here, we provide a theory that shows how to compute the generalization accuracy, or quality, of a (pre-)trained DNN simply by computing the empirical properties of the layer weight matrices, fitting the tail of the empirical spectral density (ESD) to a Power Law (PL), and plugging the PL exponent $(\alpha)$ into our resulting theoretical expression for the test error.
Using this, we derive the \WW~ \ALPHAHAT metric, which has been shown to correlate well with the test accuracies of the (pre-)trained DNNs across hundreds of models.
Our Semi-Empirical theory provides practical guidance for engineers designing complex, modern DNNs.  





}

% Please include corresponding author, author contribution and author declaration information
\authorcontributions{Please provide details of author contributions here.}
\authordeclaration{Please declare any competing interests here.}
\equalauthors{\textsuperscript{1}A.O.(Author One) contributed equally to this work with A.T. (Author Two) (remove if not applicable).}
\correspondingauthor{\textsuperscript{2}To whom correspondence should be addressed. E-mail: author.two\@email.com}

% At least three keywords are required at submission. Please provide three to five keywords, separated by the pipe symbol.
\keywords{Deep  $|$ Learning $|$ Theory $|$ ...} 

%Please provide an abstract of no more than 250 words in a single paragraph. Abstracts should explain to the general reader the major contributions of the article. References in the abstract must be cited in full within the abstract itself and cited in the text.
\begin{abstract}
We present a \SemiEmpirical Theory of Learning (\SETOL)
that explains the remarkable performance of \StateOfTheArt (SOTA) Neural Networks (NNs).
We provide a formal explanation of the origin of the
fundamental quantities in the phenomenological theory of  \HeavyTailedSelfRegularization (\HTSR), the 
\HeavyTailed \PowerLaw \LayerQuality metrics,
\ALPHAHAT $(\alpha)$ and \ALPHAHAT $(\hat{\alpha})$.
In prior work, these metrics have been shown to predict trends in the test accuracies of pretrained SOTA NN models,
and, importantly,  without needing access to the testing or even training data.
Our \SETOL
uses techniques from \StatisticalMechanics (\STATMECH) as well as advanced methods from \RandomMatrixTheory (\RMT). Our derivation suggests new mathematical preconditions for \emph{\Ideal} learning, including the new \TRACELOG metric (which is equivalent to applying the Wilson Exact Renormalization Group).
We test the assumptions and predictions of our \SETOL on a simple 3-layer
\MultiLayer \Perceptron (MLP), demonstrating excellent agreement with the key theoretical assumptions. 
For SOTA NN models, we show how to estimate the \ModelQuality of a trained NN by simply computing the \EmpiricalSpectralDensity (ESD) of the layer weight matrices and
then plugging this ESD into our \SETOL formulae.
Notably, we examine the performance of the HTSR $\alpha$ and the \SETOL \TRACELOG \LayerQuality metrics, and find that they align
remarkably well, both on our MLP and SOTA NNs.



\end{abstract}

\dates{This manuscript was compiled on \today}
\doi{\url{www.pnas.org/cgi/doi/10.1073/pnas.XXXXXXXXXX}}

\begin{document}

\maketitle
\thispagestyle{firststyle}
\ifthenelse{\boolean{shortarticle}}{\ifthenelse{\boolean{singlecolumn}}{\abscontentformatted}{\abscontent}}{}

% If your first paragraph (i.e. with the \dropcap) contains a list environment (quote, quotation, theorem, definition, enumerate, itemize...), the line after the list may have some extra indentation. If this is the case, add \parshape=0 to the end of the list environment.


\input{introduction}
\input{lsa}
\input{smog}
\input{semiempirical}
\input{matgen}
\input{empirical}
\input{convexity}
\input{discussion}
%\input{methods} % not sure 

%\subsection*{References}

%%MM%%}

\showacknow{} % Display the acknowledgments section

% Bibliography
% \clearpage
\bibliographystyle{pnas-sample}
\bibliography{dnns}


%\newpage
%\appendix
%\input{_appendix}

\appendix
\break
%FORMATTING% \onecolumngrid
%COLS% 
%\onecolumngrid
%

\section{Appendix}
\label{sxn:appendix}

%%\charles{STIL TRYING TO GET THE LATEX FORMATING RIGHT HERE}

\subsection{Data Vectors, Weight Matrices, and Other Symbols}
\label{sxn:appendix_A}

See Table~\ref{tab:dimensions} for a summary of various vectors and matrices, including their dimensions;
see Table~\ref{tab:symbols} for a summary of various various symbols used throughout the text; and 
see Table~\ref{tab:energies} for a summary of types of ``Energies'' used throughout the text.

%\paragraph{Dimensions of various matrices and vectors.}
%See Table~\ref{tab:dimensions} for a summary of various matrices and vectors used, including their dimensions.

%\renewcommand{\arraystretch}{1.5} % Increase row spacing
\renewcommand{\arraystretch}{1.2} % Increase row spacing

\begin{center}
\begin{table}[ht]
  \begin{tabular}{| l | c | r |}
    \hline
    Number of NN Layers & index $L$ & $N_{L}$ \\ \hline
    Number of Data Examples & index $\mu$ & $\ND$ \\ \hline
    Number of (input) Features & index $i,j$ & $m$ \\ \hline
    Actual Data (Matrix) & $D$ & $n \times m$ \\ \hline
    Model Data (Matrix) & $\mathcal{D}$ & $n \times m$ \\ \hline    
    \Teacher \Perceptron Weight Vector & $\TVEC$ & $m$ \\ \hline    
    \Student \Perceptron Weight Vector & $\SVEC$ & $m$ \\ \hline        
    Actual Input Data Vector & $\DATA$ & $N_{f}\times 1$ \\ \hline
    Gaussian model of Input Data Vector & $\boldsymbol{{\xi}}_{\mu}$ & $N_{f}\times 1$ \\ \hline
    Actual Input Data Label & $\MY_{\mu}$ & $+1|-1$ \\ \hline
    Model Input Data Label & $y_{\mu}$ & $+1|-1$ \\ \hline      
    General Weight Matrix & $\mathbf{W}$ & $N\times M$ \\ \hline
    General \CorrelationMatrix & $\mathbf{X}=\frac{1}{N}\mathbf{W}^{\top}\mathbf{W}$ & $M\times M$ \\ \hline
    Input Layer Weight Matrix & $\mathbf{W}_{1}$ & $N \times M$ \\ \hline
    Hidden Layer Weight Matrix & $\mathbf{W}_{2}$ & $N\times M$ \\ \hline
    Output Layer Weight Matrix & $\mathbf{W}_{3}$ & $M\times 2$ \\ \hline
    \Teacher Weight Matrix & $\mathbf{T}$ & $N\times M$ \\ \hline
    \Student Weight Matrix & $\mathbf{S}$ & $N\times M$ \\ \hline
    \StudentTeacher Overlap Matrix & $\OVERLAP=\tfrac{1}{N}\mathbf{S}^T\mathbf{T}$ & $M\times M$ \\ \hline              
    Inner \Student \CorrelationMatrix & $\mathbf{A}_M=\tfrac{1}{N}\mathbf{S}^{\top}\mathbf{S}$ & $M\times M$  \\ \hline
    Outer \Student \CorrelationMatrix & $\mathbf{A}_N=\tfrac{1}{N}\mathbf{S}\mathbf{S}^{\top}$ & $N\times N$  \\ \hline
   Inner ~\ECS \Student \CorrelationMatrix & $\AECSM$ & $M\times M$  \\ \hline
   Outer ~\ECS \Student \CorrelationMatrix &$\AECSN$ & $N \times N$  \\ \hline
   ~\ECS \Teacher \CorrelationMatrix & $\XECS$ & $M\times M$  \\ \hline
    \hline
  \end{tabular}
  \caption{Summary of of various vectors and matrices, including their dimensions.}
\label{tab:dimensions}
\end{table}
\end{center}


%\pagebreak
%\paragraph{Various other symbols.}
%See Table~\ref{tab:symbols} for a summary of various other symbols used throughout the text.

\renewcommand{\arraystretch}{1.35} % Increase row spacing

\begin{center}
\begin{table}[ht]
  \begin{tabular}{| l | c |}
    \hline
    \Perceptron \StudentTeacher (ST) Overlap & $R=\SVEC^{\top}\TVEC=\sum_{i}s_{i}t_{i}$ \\ \hline
    \StudentTeacher (ST) Overlap Operator& $\OVERLAP=\tfrac{1}{N}\SMAT^{\top}\TMAT$ \\ \hline       
    Matrix Generalized ST Overlap & $\tfrac{1}{N^2}\OLAPSQD$  \\ \hline
    \StudentTeacher \SelfOverlap & $\eta(\NDXI)=\mathbf{y}^{\top}_{T}\mathbf{y}_{S}$  \\ \hline
    $\ell_2$-Energy or $\ell_2$-Error & $\DETSTLL$ \\ \hline
    $\ell_2$-Energy Operator Form & $\DETOPSTLL:=\sum_{\XI}\DETSTLL$ \\ \hline
    $\ell_2$-Energy Matrix Operator Form & $\DETOPST:=N(\IM-\tfrac{1}{N}\SMAT^{\top}\TMAT)$\\ \hline
    \EffectivePotential  & $\epsilon(R)=\epsilon(S,T)=\langle\DETOPSTLL\rangle_{\AVGNDXI}$ \\ \hline
    \LinearPerceptron  $\epsilon(R)$ at high-T, large-$N$ & $\epsilon(R)=1-R$ \\ \hline
    \AnnealedApproximation (AA)& $\langle\ln Z\rangle_{\AVGNDXI}\approx\ln\langle Z\rangle_{\AVGNDXI}$ \\ \hline
    \AnnealedHamiltonian &  $\GAN$ \\ \hline
    \AnnealedHamiltonian at high-T &  $\GANHT=\EPSL(\WVEC)$ \\ \hline  
   %\Annealed Partition Function & $Z^{an}{_\ND}=\THRMAVGw{n\GAN}$ \\ \hline
   % \Annealed Partition Function at high-T & $\ZANHT=\THRMAVGw{n\EPSLSTw}$ \\ \hline    
    Average \StudentTeacher \GeneralizationError & $\AVGGE^{ST}=\THRMAVG{\epsilon(R)}$ \\ \hline
    Average \StudentTeacher \GeneralizationAccuracy & $1-\AVGGE^{ST}=\THRMAVG{\eta(R)}$ \\ \hline
    Matrix Layer \QualitySquared & $\QT=\THRMAVGIZ{\OLAPTOLAP}$ \\ \hline
    Equivalent Notation for Averages & $\langle\cdots\rangle_{A}=\int\cdots d\mu(\mathbf{A})=\mathbb{E}_{\mathbf{A}}[\cdots]$\\ \hline
    Projection Operator onto ~\ECS & $\mathbf{P}^{ecs} := \sum |\LambdaECS_{i}\rangle\langle\LambdaECS_{i}|,\;i=1\cdots\MECS$ \\ \hline
    Average over~\ECS \Student Correlation Matrices & $ \langle \cdots \rangle^{\beta}_{\AECS}=\int \cdots d\mu(\AECS)=  \mathbb{E}_{\AECS}[\cdots]$ \\    \hline


    \TRACELOG Trace-Log-Determinant Relation & $\Trace{\ln\mathbf{A}}=\ln\det\mathbf{A}$ \\ \hline
    Effective Correlation Measure Transform & $d\mu(\mathbf{S})\rightarrow d\mu(\AECS)$ \\ \hline
    HCIZ Integral (Tanaka's Notation)&${\mathbb{E}_{\AECS}}[\exp(\ND\beta\Trace{\TMAT^{\top}\AMAT\TMAT)}$\\ \hline
      HCIZ Integral (\LargeN in $N$ explicit) & 
$\Expected[\AECS]{\exp\!\bigl(\ND\beta N\,\mathrm{Tr}\bigl[\tfrac{1}{N}\,\TMAT^{\top}\,\AECSN\,\TMAT\bigr]\bigr)}$
\\ \hline
    \LayerQualitySquared \GeneratingFunction & $\IZGINF := \ND\beta \sum_{\mu=1}^{\MECS}\int^{\LambdaECS_{\mu}}_{\LambdaECS_{\min}} dz R(z)$
 \\ \hline
    \GEN & $G_{A}(\gamma)=\int_{\LambdaECSmin}^{\LambdaECS}R_{A}(z)dz$ \\ \hline
%    \RTransform (for Levy Matrix)& $R(z)=\nred{b}z^{\alpha-1}$ where \red{$\alpha=3$?or$2$} \\ \hline
    Eigenvalue for $\mathbf{X}=\tfrac{1}{N}\mathbf{W}^{\top}\mathbf{W}$ & $\lambda$ or $\lambda_{i}$ for $i=1\cdots M$ \\ \hline
    \PowerLaw ESD Tail for $\mathbf{X}$ & $\rho_{tail}(\lambda)\sim\lambda^{-\alpha}$ \\ \hline
    \EffectiveCorrelationSpace ESD Tail for $\mathbf{X}$ & $\rho^{ECS}_{tail}(\LambdaECS),\;\Trace{\ln\prod_{j=1}^{\MECS}\LambdaECS_{j}}=0$ \\ \hline
    Schatten Norm & $\Vert\mathbf{X}\Vert^{\alpha}_{\alpha}=\sum_{j}\lambda_{j}^{\alpha}$ \\ \hline
    ECS Tail Norm & $\tfrac{1}{\MECS}\sum_{i}^{\MECS}\LambdaECS_{i},\;\;\LambdaECS_{i}\in\rho^{ECS}_{tail}(\lambda)$\\ \hline
    Spectral Norm & $\Vert\mathbf{X}\Vert_{\infty}=\lambda_{max}$ \\ \hline
    \WW Start of PL Tail & $\lambda^{PL}_{min}=\lambda_{min}$ \\ \hline
    Start of~\ECS Tail & $\lambda^{ECS}_{min}=\lambda^{|detX|=1}_{min}$ \\ \hline
   ~\ECS-PL Gap between start of tails  & $\Delta\lambda_{min}:=\lambda^{ECS}_{min}-\lambda^{|detX|=1}_{min}$ \\ \hline            
    \WW~\ALPHA (layer) quality metric & $\alpha$ \\ \hline
    \WW~\ALPHAHAT (layer) quality metric & $\hat{\alpha}=\alpha\log_{10}(\lambda_{max})$ \\ \hline
  \end{tabular}
  \caption{Summary of various various symbols used throughout the text. 
          }
\label{tab:symbols}
\end{table}
\end{center}


%%\pagebreak
%%\paragraph{Summary of types of 'Energies'}
%%with simplified examples of the notation, and references to definitions.
%%This is meant to be a guide to understanding how the varies Energies are defined and used.
%%Please refer to the text for exact definitions, dependent variables, etc.

\newcommand{\hthinline}{\rule{0pt}{2.5ex}} % Adjust the 2.5ex to control the amount of space

%\nred{Check $\mathcal{E}(R):=\langle \Delta E\rangle_{\XI}$.  Is this only at AA and highT?}

\begin{table}[ht]
  \hspace*{-2cm}
\begin{tabular}{|p{10cm}|p{6.2cm}|p{2.25cm}|}
\hline
\textbf{Explanation} & \textbf{Examples} & \textbf{Refs} \\
\hline
\textbf{\EnergyLandscape or NN Output function} & $\NNOUT$ & Sec.~\ref{sxn:htsr_setup}\\
\hthinline
The output of the NN given a single input data point. & &\ref{eqn:dnn_energy},\ref{eqn:T_ENN},\ref{eqn:S_ENN},\ref{eqn:nflow} \\
\hline
\textbf{Energy or \StudentTeacher (ST) Error} & $\DEL$, $\DELBF$ & Sec.~\ref{sxn:SMOG_main},~\ref{sxn:summary_sst92} \\
\hthinline
The squared between the output of a \Student NN and its prescribed Teacher label $\Yt$ for a single data point &$\DEL(\XImu):=(\Yt-\SOUT(\XImu))^2 $ & \ref{eqn:DEy} \\
And as the total error for a sample of $\ND$ data points  &  $\DELBF=\sum_{\mu=1}^{n} \DEL(\XImu)$ & \ref{eqn:detopxy}  \\
Or between the outputs of the \Student and the \Teacher NNs. & $
\DELBF:=\sum_{\mu=1}^{n} (\SOUT(\XImu) - \TOUT(\XImu))^2$ & \ref{eqn:DE_L},\ref{eqn:DE},\ref{eqn:DETOPNN} \\
\hline

\textbf{\AnnealedHamiltonian (and Potentials)} & $\HAN$ $(\mathcal{E}(R), \EPSL(R))$ & Sec.~\ref{sxn:mathP_annealed},\ref{sxn:summary_sst92} \\
\hthinline
The \EffectivePotential (for the Error) is defined as:  &
$\EPSL(R)=\tfrac{1}{\ND}\mathcal{E}(R)=\langle \DELBF\rangle_{\AVGNDXI}$ & ~\ref{eqn:epslR}\\
The \AnnealedHamiltonian (for the Error):  &
$\beta\HAN:=\tfrac{1}{\ND}\ln\langle e^{-\beta\DELBF}\rangle_{\NDXI}$ & ~\ref{eqn:Gan_def},~\ref{eqn:Gan0} \\
At high-T, the relation between $\HANHT$ and $\EPSL(R)$ is &
$\HANHT(R):=\EPSLR$ & ~\ref{eqn:Gan_highT} \\
\hthinline
The full ST model Hamiltonian: &
$\beta\HAN(\beta,R):= \tfrac{1}{2}\ln[{1+2\beta(1-R)}]$ & ~\ref{eqn:Gan2} \\
At high-T, the ST model Hamiltonian is: &
$\beta\HANHT(R):=\beta(1-R)$ & ~\ref{eqn:epslR},~\ref{eqn:Gan3} \\
\hthinline
The per-parameter ST model Hamiltonian: &
$\beta\HANPP:=\tfrac{1}{2}\ln[1 + \tfrac{2\beta N}{M}\Trace{\IM-\mathbf{R}}]$  & ~\ref{eqn:Gan_lnI_final_mwm} \\
At high-T, the matrix-generalized ST model Hamiltonian is: &
$\GANMATHT:=N(\IM-\OVERLAP)$&
~\ref{eqn:GANHTmatR} \\
The \LayerQualitySquared Hamiltonian (for the Accuracy): &
$\HBARE:=\OLAPTOLAP$ & ~\ref{eqn:RG},~\ref{eqn:HBARE} \\
\hline
\textbf{Different Average Model Errors} & $\AVGE$ & Sec.~\ref{sxn:mathP_errors}\\
Empirical Training, \Teacher, and Test Errors  & $\AVGEMPTE$, $\AVGE^{T}\approx\AVGEMPGE$ & \ref{eqn:Eg_train},\ref{eqn:Eg_test},\ref{eqn:emp_gen_error}\\
\hthinline
Empirical \GeneralizationGap & $\AVGE^{emp}_{gap}:=\AVGEMPTE-\AVGEMPGE$ & \ref{eqn:gen_gap} \\
\hthinline
\StudentTeacher Training and Generalization Errors  & $\AVGSTTE$, $\AVGSTGE$ & \ref{eqn:EtM2},\ref{eqn:EgCanonical}\\
\hthinline
Neural Network (MLP) Training and Generalization Errors, the (abstract) matrix generalization of ST error
& $\AVGNNTE$, $\AVGNNGE$ & \\
\hline
\textbf{Average Training and/or \GeneralizationError} & $\AVGTE, \AVGGE$ & Sec.~\ref{sxn:mathP_errors}\\ 
In the AA and at High-T, these are the same, 
and are just the \ThermalAverage of $\EPSL(R)$ & $\newline \AVGTE^{an,hT}=\AVGGE^{an,hT}=\THRMAVG{\EPSL(R)}$ &
\ref{eqn:avgte_anhT},\ref{eqn:avgge_anhT}\\
For the ST model, we always assume AA and High-T & $\AVGSTGE=\AVGGE^{an,hT}$ & \\
Likewise, when generalizing $\AVGSTGE$ to matrices,  & $\AVGSTGE\rightarrow\AVGNNGE=\AVGGE^{an,hT}$ & \\
\hline
\textbf{Layer Qualities} & $\Q$, $\QT$ & Sec.~\ref{sxn:matgen_quality_hciz_A},~\ref{sxn:quality} \\
\hthinline
For the ST \Perceptron, $\Q^{ST}$ is the generalization accuracy & $\Q^{ST}:= 1-\AVGSTGE = 1-\THRMAVG{\EPSL(R)}$ & \\
in terms of the \SelfOverlap $\ETA(R)$ & $\Q^{ST}:= \THRMAVG{\ETA(R)}$ & \\
In the AA, and at high-T, $\THRMAVG{\EPSL(R)}=1-R$ & $\Q^{ST}:= 1-\AVGGE^{an,ht} = \THRMAVG{R}$ & \ref{eqn:QST_final} \\
For an MLP / NN, we approximate the total accuracy as a product of layer qualities $\Q$ (in the AA, at high-T) &  $\Q^{NN}:=\prod \Q^{NN}_{L}$ &  \ref{eqn:ProductNorm}\\
For a matrix, the \LayerQualitySquared $\QT$ & $\QT:=\THRMAVGIZ{\OLAPTOLAP}$ &\ref{eqn:QT_1}\\
We approximate $\Q$ using the quality squared & $\Q:=\sqrt{\QT}\approx Q^{NN}_{L}$ & \ref{eqn:QT},\ref{eqn:QT_2}\\
\hline
\end{tabular}
  \caption{Summary of types of ``Energies,'' with simplified examples of the notation, and references to definitions.
          }
\label{tab:energies}
\end{table}

\clearpage
\renewcommand{\arraystretch}{1.0} % reset row spacing

\subsection{Summary of the \StatisticalMechanicsOfGeneralization (\SMOG)}
\label{sxn:summary_sst92}


In this section, we derive the Annealed Hamiltonian for two variants of the ST model, in the high-T limit:
in Appendix~\ref{app:st-gen-err-annealed-ham}, we derive an expression for $\GANR$ for the ST \Perceptron model, when the students and teachers are modeled as $N$-vectors $\WVEC$ (as in~\cite{SST92}); and
in Appendix~\ref{sxn:appendix_Gan}, we derive an expression for $\GANMAT$ for Matrix-Generalized case, i.e., when the students and teachers are modeled as $N \times M$ matrices $\WMAT$ (as our \SETOL requires).
From these, we will obtain expressions for the Average ST \ModelGeneralizationError $\AVGSTGE$ and the Average NN \ModelGeneralizationError $\AVGNNGE$, as well as for the corresponding data-averaged errors.
Although the functional form for these quantities will be the same for the vector case and the matrix case, there are several important differences in the derivation of $\GANMAT$, most notably having to do with a normalization for the weight matrix. 


\subsubsection{Annealed Hamiltonian \texorpdfstring{$\GANR$}{H(R)} when Student and Teachers are Vectors}
\label{app:st-gen-err-annealed-ham}

In this section, we derive an expression for the Annealed Hamiltonian $\GANR$, 
in the AA and the high-T approximation,
when student and teachers are are modeled as $N$-vectors.
%
From this, we obtain an expression for the data-averaged ST error $\EPSL(R)$, 
which is the same as the expression given in \EQN~\ref{eqn:e0}.

The procedure starts by computing the associated quenched average of
the \FreeEnergy, defined for the model error as
\begin{align}
\nonumber
\langle-\beta F \rangle_{\AVGNDXI} 
   :=& \langle \ln Z \rangle_{\AVGNDXI} \\ 
\nonumber
   =&\left\langle \ln \int d\mu(\SVEC)e^{-\beta\DETOPSTLL} \right\rangle_{\AVGNDXI}  \\ 
\label{eqn:qaF}
   =& \frac{1}{N}\int d\mu(\NDXI)\ln\int d\mu(\SVEC)e^{-\beta\DETOPSTLL}   ,
\end{align}
where $d\mu(\NDXI):=\prod_{i=1}^{N}d\XI_{i} P(\NDXI)$ and
where the data-dependent ST error, $\DETOPSTLL$, is defined in \EQN~\ref{eqn:DE_L},
with an $\mathcal{L}=\ell_2$ loss.  \footnote{Also, recall that the Teacher $T$ is fixed and is not learned, so we do not integrate over $d\mu(\TVEC)$. In fact, for the (vector) Perceptron model, the Teacher $T$ weights are simply subsumed into the overlap $R$, and even in the more general cases, such as non-Linear/Boolean Perceptron, in the full Replica calculations, etc.  See the original literature for more details. \cite{Opper01,SST92,EngelAndVanDenBroeck} as well as \cite{MM17_TR}.}
\charles{Do we need a $1/N$ to ensure the free energy is extensive ? I don't think we do because at the end we want to evaluate the
  generalization accuracy, so we need to take the partial w/r.t. $N$. But we need to check this and section 4.2 carefully.
Of course, with the HCIZ integral, we have to be also be careful since we move the $1/N$ from the L.H.S. to the R.H.S of Tanaka just for this reason}
%
If we apply the AA (see \EQN~\ref{eqn:AA} and \EQN~\ref{eqn:Jensens}) to \EQN~\ref{eqn:qaF}, then we obtain
\begin{align}
\label{eqn:qaFan}
\langle-\beta F \rangle_{\AVGNDXI} \simeq 
%%   \ln \langle Z \rangle_{\AVGNDXI} = 
   \ln \frac{1}{\ND}\int d\mu(\NDXI) \int d\mu(\SVEC) e^{-\beta\DETOPSTLL}   .
\end{align}
%
Notice that we have interchanged the logarithm $(\ln)$ and
and the data average (the “disorder average”) 
$\langle\cdots\rangle_{\AVGNDXI}$ over the data; 
this is the essence of the AA, as it lets the disorder fluctuate rather than forcing the system to be quenched to the data.
%
We will now switch the order of integration in \EQN~\ref{eqn:qaFan}, giving
\begin{align}
\label{eqn:qaFan3}
\langle-\beta F \rangle_{\AVGNDXI} \simeq 
%%   \ln \langle Z \rangle_{\AVGNDXI} = 
   \ln \int d\mu(\SVEC)\frac{1}{\ND}\int d\mu(\NDXI)e^{-\beta\DETOPSTLL)}   .
\end{align}


We now recall the definition of the \AnnealedHamiltonian, $\GANR$ (see \EQN~\ref{eqn:Gan_def} in Section~\ref{sxn:mathP}, which is analogous to \EQN~(2.31) of \cite{SST92}):
\begin{align}
\label{eqn:Gan0}
\beta\GANR = \beta\HAN(\beta,\SVEC,\TVEC):=-\frac{1}{\ND}\ln \int d\mu(\NDXI)e^{-\beta\DETOPSTLL}  .
\end{align}
where we have denoted the Hamiltonian as $\HAN(\beta,\SVEC,\TVEC)$ to indicate the explicit dependence on $\beta$,
and we have added $\beta$ to the R.H.S. to because the L.H.S. is unitless.
%
Using this definition, we can express the \Annealed \PartitionFunction, $Z^{an}_{\ND}$, in the AA in terms of the \Annealed Hamiltonian $\GANR$
\michael{@charles: There the macro is \Annealed and elsewhere it is \AnnealedHamiltonian; does it matter here or elsewhere.}
(as in \EQN~\ref{eqn:Zan_def} in Section~\ref{sxn:mathP}, and as in \EQN~(2.31) of \cite{SST92}:
\begin{align}
 \label{eqn:Zan}
Z^{an}_{\ND}:=\ND\langle Z\rangle_{\AVGNDXI}=\int d\mu(\SVEC)e^{-\ND\beta\HAN(\beta,\SVEC.\TVEC)}  .
\end{align}

Following Section~\ref{sxn:mathP_errors}, we can write the \emph{Average Model \TrainingError} $\AVGSTTE$, in the AA,
in terms of \Annealed \PartitionFunction, $Z^{an}_{\ND}$:
\begin{align}
 \label{eqn:AVGSTTE_AA}
\AVGSTTE:= \red{-}\frac{1}{\ND}\dfrac{\partial }{\partial \beta} \ln Z^{an}_{\ND}
\end{align}


This now lets us write the Average Model \TrainingError $\AVGSTTE$ in the AA in terms of the \AnnealedHamiltonian $\GANR$:
  \begin{align}
  \nonumber
  \AVGSTTE
   =& \dfrac{1}{Z^{an}_{\ND}}\int d\mu(\SVEC)\dfrac{\partial \beta\HAN}{\partial \beta} e^{-\ND\beta\HAN(\beta,\SVEC,\TVEC)} \\ 
  \label{eqn:EtM2}
   =& \dfrac{1}{Z^{an}_{\ND}}\int d\mu(\SVEC)\langle  \DELBFell(\SVEC,\TVEC,\XI) \rangle_{\AVGNDXI} e^{-\ND\beta\HAN(\beta,\SVEC,\TVEC)}  ,
  \end{align}
  where $\DELBFell(\SVEC,\TVEC,\XI) \rangle^{\beta}_{\AVGNDXI}$
  is a \ThermalAverage but defined over the specific ST error in the AA for the chosen set of training data $\XItrain=\NDXI$, and is denoted by 
 \begin{align}
   \langle  \DELBFell(\SVEC,\TVEC,\XI) \rangle^{\beta}_{\AVGNDXI}:=\dfrac{\partial \beta\HAN}{\partial \beta}.
 \end{align}
 This is analogous to defining the average error $\DETOPXI$ as a \ThermalAverage, but as one over the data $\NDXIn$ instead of the weights.  This can be seen by expanding \EQN~\ref{eqn:Gan_def}, setting $\WVEC=\SVEC$, fixing $\TVEC$ (implicitly), and taking the partial derivative:
 \begin{align}
  \label{eqn:partial_Gan}
  \dfrac{\partial \beta\HAN}{\partial \beta}
  &=  \dfrac{\partial }{\partial \beta} \left(-\frac{1}{\ND}\ln \int d\mu(\NDXIn)e^{-\beta\DETOPXILL}\right) \\ \nonumber
  &=  -\frac{1}{\ND}\dfrac{\partial }{\partial \beta} \ln \int d\mu(\NDXIn)e^{-\beta\DETOPXILL} \\ \nonumber
  &=  -\frac{1}{\ND} \left( \int d\mu(\NDXIn)e^{-\beta\DETOPXILL}\right)^{-1}\dfrac{\partial }{\partial \beta} \int d\mu(\NDXIn)e^{-\beta\DETOPXILL} \\ \nonumber
    &=  -\frac{1}{\ND} \left(\int d\mu(\NDXIn)e^{-\beta\DETOPXILL}\right)^{-1} \int d\mu(\NDXIn)(-\DETOPXI)e^{-\beta\DETOPXILL}
% &=  \frac{1}{\ND} \langle\DETOPXI\rangle_{\AVGNDXIn}
 \end{align}

We can also write the \ModelGeneralizationError $\AVGSTGE$  as \BoltzmannWeightedAverage
of $\epsilon(R)$, weighted by $\HAN(\beta,S;T)$ (as in (2.32) in \cite{SST92}), as:
\begin{align}
\label{eqn:EgCanonical}
\AVGSTGE=\dfrac{1}{Z^{an}_{\ND}}\int d\mu(\SVEC)\epsilon(R)e^{-\ND\beta\HAN(\beta,\SVEC,\TVEC)} ,
\end{align}
where $\epsilon(R)=\epsilon(\SVEC)$ is the average ST error, for a fixed \Teacher T,
averaged over \emph{all} possible data inputs, i.e., not just over the specific training data.
(Note that we have dropped the subscript $train$ on $\XI$ since it is clear from the context.)


%Notice that $\GANR$ is like a \emph{data-averaged ST error}, $\epsilon(\SVEC,\TVEC)$, (\EQN (\ref{eqn:STerror}),)
%but now expressed as the logarithm of the (un-normalized) Boltzmann-weighted the data $\XImu$.
%\michael{Charles, I rewrote that sentence, since it was confusing, would you make sure I didnt mess it up.}
%\michael{Is \cite{SST92} the ref for that?}
%%\michael{Is this integral over $d\mu(\NDXI)$ the same as $\sum_{\mu=1}^{N_aD}$ above, or the latter is the empirical version of the former, or are they different?


%  Notice that for the model \GeneralizationError $\MGE$, we evaluate the data-averaged ST error $\epsilon(R)$ directly,
%  hereas for the   model \TrainingError $\MTE$,
%  we have to evaluate $\GANR$ first, which is like an (un-normalized) thermal average over the ST error.
%  (check this; maybe confusing ?)


In the high-T (small $\beta$) limit, the two model errors become formally equivalent
(i.e., $\AVGSTTE=\AVGSTGE$ as $T\rightarrow\infty$).
To show this, consider the \AnnealedHamiltonian $\GANR$, for the \LinearPerceptron with the $\ell_2$ loss. 
As shown in \EQN~(C6) of~\cite{SST92}, this takes a simple analytic form--in the \LargeN limit in $\ND$--in terms of the ST overlap $R$:
\begin{align}
\label{eqn:Gan2}
\GANR = \dfrac{1}{2}\ln\left[{1+2\beta(1-R)}\right]  .
\end{align}
\EQN~\ref{eqn:Gan2} holds in the AA, but not in the High-T limit.

If we evaluate $\dfrac{\partial \HAN}{\partial \beta}$ in the High-T (small $\beta$) limit, 
then we can use the approximation $(\ln[1+x]\simeq x+\cdots)$ to obtain the High-T approximation:
\begin{align}
\label{eqn:Gan3}
\beta\GANR \simeq 
\beta\GANHTR:=\beta(1-R),\;\;\;\;\beta\;\text{small}  .
\end{align}
where we now see that $\GANHTR$ no longer explicitly depends on $\beta$.
%
By \EQN~\ref{eqn:Gan_highT_final}, this gives 
\begin{align}
\label{eqn:Gan4}
\EPSL(R) =
\langle  \mathbf{E}_{\ell_2}(\SVEC,\TVEC,\XI \rangle_{\AVGNDXI} \simeq 1-R\;\;\text{as}\;\ND\rightarrow\infty  ,
\end{align}
which we recognize as the same as the data-averaged ST error $\epsilon(R)$ in \EQN~\ref{eqn:e0}.
\michael{@charles: also, there is a $1/N$ in \EQN~\ref{eqn:e0}; does that belong here.}


\subsubsection{Annealed Hamiltonian \texorpdfstring{$\GANMAT$}{H(R)} for the Matrix-Generalized ST Error}
\label{sxn:appendix_Gan}

In this section, we derive an expression for our matrix generalization of the \AnnealedHamiltonian of the \LinearPerceptron,
in the AA and the high-T approximation, when student and teachers are are modeled as $N \times M$ matrices $\WMAT$,
i.e., $\GANR\rightarrow\GANMAT$, which has the same form as \EQN~\ref{eqn:Gan0} for the vector case.

From this, we obtain an expression for the data-averaged ST error $\EPSL(R)$, again when the student and teachers are are modeled as matrices.
There is a subtle normalization issue here, about which we need to be careful.
However, when we normalize appropriately, we will obtain an expression
for ``data-averaged ST error'' (i.e., \EffectivePotential) $\EPSL(R)$ that is of the same form as we obtained in the vector case (as given in \EQN~\ref{eqn:Gan4} and \EQN~\ref{eqn:e0}).
%
The difference will be that in the vector case we take $R=\tfrac{1}{N}\SVEC^{\top}\TVEC$, while in the matrix case we take $R=\tfrac{1}{N}\SMAT^{\top}\TMAT$.

We will need to evaluate an average over the $\ND$ random $M$-dimensional training data vectors $\NDXI$,
which are i.i.d Gaussian with $0$ mean and $\sigma^{2}$ variance: 
\begin{equation}
  \label{eqn:XInorm}
  \Vert \NDXI \Vert^2 :=\sum_{\mu=1}^{\ND} \XI_\mu \XI_\mu^{\top} = \sigma^{2}\IM ,
\end{equation}
where each $\XI_{\mu}$ is a vector of length $M$, and $\IM$ is an $M \times M$ identity matrix.
%
The expected value of the squared norm is:
\begin{equation}
\mathbb{E}[\Vert \XI_{\mu} \Vert^2] = M \sigma^2 .
\end{equation}
If we let $\sigma^{2}\sim\tfrac{1}{M}$, 
then $\mathbb{E}[\Vert \XI_{\mu} \Vert^2]=1$, i.e., the data vectors can be normalized to $1$.
%
Let the probability distribution over the $N$ data vectors~be
\begin{align}
\nonumber
  P(\NDXI) &= \prod_{\mu=1}^{\ND} \left( \frac{1}{\sqrt{(2 \pi \sigma^2)^M}} \right) e^{-\frac{\|\XI_{\mu}\|^2}{2 \sigma^2}} \\ 
\nonumber
  &= \left( \frac{1}{\sqrt{(2 \pi \sigma^2)^M}} \right)^{\ND} \exp\left[-\frac{M}{2}\sum_{\mu=1}^{\ND}\|\XI_{\mu}\|^2\right] \\ 
  \label{eqn:pndx_vec}
  &= \XINORM \exp\left[-\frac{M}{2}\sum_{\mu=1}^{\ND}\|\XI_{\mu}\|^2\right] ,
\end{align}
where $M=N_f$ is the number of features in the data, where the normalization $\XINORM$ is
\begin{align}
\label{eqn:xinorm}
\XINORM 
:=\left( \frac{1}{\sqrt{(2 \pi \sigma^2)^M}} \right)^{\ND}
 =\left( \frac{M}{2\pi} \right)^{\ND M/2} .
\end{align}

\paragraph{The Total Data Sample Error $(\DETOPST)$ and the Matrix Normalization}
First, let us express the matrix-generalized \TotalDataSampleError, $\DETOPST$, for a single layer,
in operator form (for each  of the $\ND$ training examples)
%\begin{align}
% \label{eqn:DETOPST}
% \frac{1}{\ND}\DETOPST  := \Trace{\IM - \frac{1}{N}\SMAT^{\top}\TMAT} = M - \Trace{\OVERLAP} 
%\end{align} 
\begin{align}
 \frac{1}{\ND}\DETOPST := N\Trace{\IM - \frac{1}{N}\SMAT^{\top}\TMAT} = NM - N\Trace{\OVERLAP} 
\end{align}

where $\IM$ is a diagonal matrix of dimension $M$.
Note that the matrices are by default data-averaged empirical quantities, so we can drop the $1/\ND$ on the RHS.

Also, notice that $\DETOPST$ scales as $N\times M$, the total number of parameters in the system.
Also,  importantly, when all the overlaps are perfect, then the error is zero, i.e. if $\Trace{\OVERLAP}=M$ then $\DETOPST=0$.
                                                                        
We can define the data-dependent form (i.e., in the basis of the data $\XI$) as
\begin{align}
\nonumber
\DETOPNN
   :=& N\sum_{\mu=1}^{\ND} (\XI^{\mu})^{\top} \left( \IM - \frac{1}{N} \SMAT^{\top}\TMAT \right) \XI^{\mu} \\
\label{eqn:DETOPNN}
    =&  N\sum_{\mu=1}^{\ND} \sum_{i,j=1}^{M} \XI_i^{\mu} \left( \delta_{ij} - \frac{1}{N} [\SMAT^{\top}\TMAT]_{ij} \right) \XI_j^{\mu}  .
\end{align}

\paragraph{The \AnnealedHamiltonian (per-parameter, $\HANPP)$}
The definition of the \AnnealedHamiltonian, $\GANMAT$, for the idealized case must be extended to account for the $N\times M$ parameters per training example.  We then have that the total energy is then the sum of the entries of $M$ (feature) vectors, as expected by 
\SizeExtensivity in $N$ and
\SizeConsistency in $M$:
\begin{align}
 \label{eqn:hanpp}
 \Trace{\GANMAT}=M\left(N\Trace{\HANPP}\right)
\end{align}
where the \AnnealedHamiltonian per-parameter, $\HANPP$, is obtained from \EQN~\ref{eqn:Gan0} as
\begin{align}
\label{eqn:Gan_lnI}
\beta\HANPP
   &:=-\frac{1}{\ND}\ln   \int\mathcal{D}\NDXI \, e^{-\beta \DETOPST} P(\NDXI) \\
\nonumber
   &=-\frac{1}{\ND}\ln \IH ,
\end{align}
where
\begin{align}
\label{eqn:InsideGan}
\IH := \int\mathcal{D}\NDXI \, e^{-\beta \DETOPST} P(\NDXI) .
\end{align}
That is, $\HANPP$ represents the Energy or Error that each of the $N\times M$ parameters contributes
(averaged over the $N$ training examples $\NDXI$).

The goal will be to derive the high-Temperature \AnnealedHamiltonian, $\GANMATHT$, which is now  defined such that:
\begin{align}
 \label{eqn:hanpp2}
  \Trace{\GANMATHT}:=MN\left(\Trace{\HANPPHT  }\right)
\end{align}

If examining the the trace of $\HANPPHT$, then we can infer the functional form necessary to define
the matrix-generalized \EffectivePotential for each parameter:
\begin{align}
  \label{eqn:EPSL_mat}
  \EPSL(\OVERLAP):=\Trace{\HANPPHT},
\end{align}
which would be like a mean-field potential, but we need something different for the matrix case.

%%We now evaluate the integral over $d\mu(\NDXI)$ in \EQN~\ref{eqn:Gan0}. 

To evaluate the integral, notice that $\IH$ is really just an average over i.i.d. data, and so it is just a product over $\ND$ independent terms ($1$ for each training example).
\begin{align}
\IH := \int\mathcal{D}\NDXI  e^{-\beta \DETOPST} P(\NDXI)  \rightarrow\left[\int\mathcal{D}\XI \;[\cdots]\; \right]^{\ND} ,
\end{align}
as in \EQN~\ref{eqn:I_4} below.
Moreover, when taking $\ln \IH$, the $N$ term pulls down and become a prefactor
\begin{align}
-\ln \IH = -\ln\left[\int\mathcal{D}\XI \;[\cdots]\; \right]^{\ND}= -\ND\ln\left[\int\mathcal{D}\XI \;[\cdots]\; \right] .
\end{align}
Thus, as with the vector case, $\GANMAT$ is like a mean-field average over the data $\XI$, indepedent of the sample size $N$.
Also, since the final result must scale as $N\times M$, the integral should scale as $M$, i.e.,
$\left[\int\mathcal{D}\XI \;[\cdots]\; \right]\sim M$.


If we substitute $\DETOPNN$, \EQN~\ref{eqn:DETOPNN}, into the integral $\IH$, \EQN~\ref{eqn:InsideGan}, then we obtain
\begin{align}
\label{eqn:I_1} 
\IH 
  & =  \int \mathcal{D}\NDXI \, \exp \left( -\beta\sum_{\mu=1}^{\ND} N(\XI^{\mu})^{\top} \left(\IM-\frac{1}{N} \SMAT^{\top}\TMAT \right) \XI^{\mu} \right) P(\NDXI)  \\
\nonumber
%%  \IH
  & =  \int \mathcal{D}\NDXI \, \exp \left( -\beta\sum_{\mu=1}^{\ND} N(\XI^{\mu})^{\top} \left(\IM-\frac{1}{N} \SMAT^{\top}\TMAT \right) \XI^{\mu} \right) \NORM \exp\left( - \sum_{\mu=1}^{N} \frac{\|\XI^{\mu}\|^2}{2 \sigma^2} \right)  \\
    \nonumber
    %%  \IH
  &= \NORM \int \mathcal{D}\NDXI \, \exp \left(
    -\beta\sum_{\mu=1}^{\ND} N(\XI^{\mu})^{\top} (\IM-\tfrac{1}{N}\SMAT^{\top}\TMAT) (\XI^{\mu}) 
    - \sum_{\mu=1}^{\ND} \frac{\|\XI^{\mu}\|^2}{2 \sigma^2} \right) \\ 
\nonumber
  &= \NORM \int \mathcal{D}\NDXI \, \exp \left(
    -\frac{1}{2\sigma^2}\sum_{\mu=1}^{\ND}2\beta\sigma^{2} N(\XI^{\mu})^{\top} (\IM-\tfrac{1}{N}\SMAT^{\top}\TMAT) (\XI^{\mu}) 
    +  \Vert\XI^{\mu}\Vert^{2} \right) \\ 
\label{eqn:I_3} 
  &= \NORM \int \mathcal{D}\NDXI \, \exp \left(
    -\frac{1}{2\sigma^2}      
      \sum_{\mu=1}^{\ND}
          (\XI^{\mu})^{\top}[
      2\beta\sigma^{2}N (\IM-\tfrac{1}{N}\SMAT^{\top}\TMAT)+\IM] (\XI^{\mu})\right)  .
\end{align}
%
By combining the exponents, we obtain
\begin{align}
\nonumber
\IH
  &=  \NORM\int \mathcal{D}\NDXI 
  \exp\left[
    -\frac{1}{2\sigma^2}\sum_{\mu=1}^{N}(\XI^{\mu})^{\top}
    \left(\mathbf{M}
    \right)
    \XI^{\mu}
    \right ]\\ 
\label{eqn:I_4} 
  &=  \NORM\int \mathcal{D}\XI  
 \exp\left[
    -\frac{1}{2\sigma^2}(\XI)^{\top}
    \left(\mathbf{M}
    \right)
    \XI
    \right]^{\ND}   ,
\end{align}
where $\mathbf{M}=2\beta\sigma^{2}N(\IM - \tfrac{1}{N}\SMAT^{\top}\TMAT)+\IM$ is an $M \times M$ matrix.
We now use the familiar property of multi-variant Gaussian integrals,
\begin{align}
\label{eqn:det_M}
\int d\mathbf{x}  e^{-\frac{1}{2\sigma^{2}}(\mathbf{x})^{\top}\mathbf{M}(\mathbf{x}) } = (2\pi\sigma^{2})^{M/2}\frac{1}{\sqrt{\Det{ \mathbf{M}}}}
\end{align}
where $\mathbf{x}$ is an $m$-dim vector (with zero mean),
and $\mathbf{M}$ is a square positive-definite matrix, and $\Det{ \mathbf{M}}$ is the determinant of $\mathbf{M}$.
%
Using \EQN~\ref{eqn:det_M}, we can rewrite $\IH$ in \EQN~\ref{eqn:I_4} as
\begin{align}
\label{eqn:I_5}
\IH &=   \NORM\left[\frac{(2\pi\sigma^{2})^{M/2}}{\sqrt{\Det{ \mathbf{M}}}}\right]^{\ND} \\ \nonumber
    &=   \NORM(2\pi\sigma^2)^{NM/2}\left[
         \sqrt{\Det{ 2\beta\sigma^2N(\IM-\tfrac{1}{}\SMAT^{\top}\TMAT)+\IM}}\right]^{-\ND}  \\
%%\end{align}
%%%
%%\begin{align}
\label{eqn:I_6}
%%\IH 
   &=  \left( \frac{1}{2\pi\sigma^{2}} \right)^{N M/2}
       (2\pi \sigma^{2})^{M/2}
         \left[\sqrt{\Det{ \IM + 2\beta\sigma^{2}N(\IM-\tfrac{1}{N}\SMAT^{\top}\TMAT)}}\right]^{-\ND}  ,
\end{align}
where \EQN~\ref{eqn:I_6} follows by inserting $\XINORM$ from \EQN~\ref{eqn:xinorm}.
We can now identify $\sigma^{2}=\tfrac{1}{M}$ to obtain
\begin{align}
\nonumber
\IH &= \left[\sqrt{\Det{ \IM + 2\beta\sigma^{2}N(\IM-\tfrac{1}{N}\SMAT^{\top}\TMAT)}}\right]^{-\ND}   \\ 
\nonumber
    &= \left[\sqrt{\Det{ \IM  + \tfrac{2\beta}{M}N(\IM-\tfrac{1}{N}\SMAT^{\top}\TMAT})}\right]^{-\ND} \\ 
\label{eqn:I_7}
    &= \left[\Det{ \IM  + \tfrac{2\beta}{M}N(\IM-\tfrac{1}{N}\SMAT^{\top}\TMAT})\right]^{-\ND/2}  .
\end{align}

\paragraph{The High-Temperature Limit.}
In the high-T approximation, $\beta$ becomes small, giving the expression
%%\begin{align}
$
\det(\IM+\epsilon\mathbf{\Omega})\approx1+\epsilon\Trace{\mathbf{\Omega}}  ,
$
%%\end{align}
which holds for an arbitrary matrix $\Omega$ for small $\epsilon$.
Using this, we can evaluate the determinant in \EQN~\ref{eqn:I_7} in the large-$N$ approximation, 
which gives
\begin{align}
  \label{eqn:I_8}
  \IH &\approx  \left[1  + \tfrac{2\beta}{M}N(\Trace{\IM-\tfrac{1}{N}\SMAT^{\top}\TMAT})\right]^{-\ND/2}   .
\end{align}
Inserting this into \EQN~\ref{eqn:Gan_lnI}, we obtain
\begin{align}
\beta\HANPP
   &=-\frac{1}{\ND}\ln \left[1  + \tfrac{2\beta}{M}N(\Trace{\IM-\tfrac{1}{N}\SMAT^{\top}\TMAT})\right]^{-\ND/2}  \\ \nonumber
      &=\frac{1}{2}\ln \left[1  + \tfrac{2\beta}{M}N(\Trace{\IM-\tfrac{1}{N}\SMAT^{\top}\TMAT})\right]
      \label{eqn:Gan_lnI_final_mwm}
\end{align}

This form of the Hamiltonian, $\GANMAT$, however, is not symmetric, and we will
eventually want a symmetric operator or matrix.
Fortunately, the high-T form, $\GANMATHT$, can be made symmetric, as
shown below.  But first, let show that this result is consistent with the our previous Percpetron result.
  
\paragraph{Matrix-Generalized ST Error $\GANMATHT$ for $N=1$.}
To start, observe that when $N=1$, \EQN~\ref{eqn:Gan_lnI_final_mwm} becomes
\begin{align}
\nonumber
\beta\GANMAT\vert_{N=1}
&= \beta\HANPP\vert_{N=1}  \\ 
  &=  -\frac{1}{\ND}\ln  \left[1  + 2\beta\tfrac{1}{M}\Trace{(M-\SVEC^{\top}\TVEC}\right]^{-\ND/2} \\ 
\nonumber
  &= \frac{1}{2}\ln  \left[1  + 2\beta\tfrac{1}{M}\Trace{M-\SVEC^{\top}\TVEC}\right] \\ 
\label{eqn:GANmat_m_equals_1}
  &=  \frac{1}{2}\ln \left[1 + 2\beta(1-R)\right]  ,
\end{align}
where we recall that $\SVEC$ and $\TVEC$ are implicitly normalized to $1/m$, where here $m=M$.  This result shows that \EQN~\ref{eqn:Gan_lnI_final_mwm} reduces to \EQN~\ref{eqn:Gan2}, as desired.

\charles{We want $\GANMATHT$ to represent the total energy per training example, so it should NOT include the $\tfrac{1}{M}$
FIX THIS}
This ensures the Hamiltonian scales as $M$ so the Free Energy scales as $N \times M$, the
number of free paramaters in the system.
Notice that for the final \LayerQualitySquared Hamiltonian $\HBARE$, this will change.
\michael{MM TO DO: adjust phrasing to say what the dimensions of those matrices are, once I figure out how to describe $\mathbf{A}_{1}$ versus $\mathbf{A}_{2}$ above, maybe explicit there that there are two different $\mathbf{A}$ matrices.}
\charles{@michael: working on this, but IDK its  about A1 and A2; REWORKING THIS:}
\begin{align}
  \label{eqn:I_9}
    \IH\approx  \left[1  + \frac{2\beta}{M}N\Trace{\IM-\tfrac{1}{N}\SMAT^{\top}\TMAT}\right]^{-\ND/2} ,
\end{align}
for any $M>1$.
Given this, it follows from \EQN~\ref{eqn:Gan_lnI_final_mwm} and \EQN~\ref{eqn:Gan_lnI} that we can define 
\begin{align}
\label{eqn:GANmat}
\beta\GANMAT \vert_{N=1}
  &:=  \frac{M}{2}\ln\det \left[1 + \frac{2\beta}{M}(\IM-\mathbf{R})\right]  ,
\end{align}
and reduces to the same functional form as \EQN~\ref{eqn:Gan2}, as desired (recalling that $\SVEC$ and $\TVEC$ are implicitly normalized by $M=m$).

To obtain the high-Temperature form, we use
\begin{align}
\ln\det\!\Bigl[\IM + \epsilon\mathbb{M}\Bigr]
\approx \mathrm{Tr}\!\bigl(\epsilon\mathbb{M}\bigr)
\quad(\epsilon\ll 1),
\end{align}
with $\epsilon=\tfrac{2\beta}{M}$ and $\mathbb{M}=\IM-\OVERLAP$.   
We now obtain the following result for the matrix-generalized high-T of $\GANMATHT$ using
\begin{align}
\label{eqn:GANHTmatRN1}
\Trace{\GANMATHT}\vert_{N=1} = \Trace{\IM-\OVERLAP} = M-\Trace{\OVERLAP}
\end{align}

The final expression for $\GANMATHT$ is
\begin{align}
\label{eqn:GANHTmatR}
\GANMATHT = N(\IM-\OVERLAP)
\end{align}

% We will use this normalization for both the \TRACELOG  condition and deriving the
%  \Quality (Squared) \GeneratingFunction $\IZG$ in Section~\ref{sxn:matgen}.
%  In contrast, for modeling the \RTransform in SubSection~\ref{sxn:r_transforms},
%  we will use the original normalization, and divide the correlation matrix $\XMAT$
%  (or, really, $\XECS$) by $M$ when computing the matrix moments.
%  We make these choices because this is the normalization currently used
%  in the opensource~\WW package for these two features.




\subsection{Expressing the Layer Quality}
\label{sxn:quality}

In this section, we obtain an approximation expression for the \LayerQualitySquared from the IZ \FreeEnergy for the \GeneralizationError, 
given in \EQN~\ref{eqn:betaIZG_S} in Section~\ref{sxn:matgen_quality_hciz_A}.

For the required \FreeEnergy $\IZFE$, we will use the matrix-generalized Hamiltonian
from \EQN~\ref{eqn:GANHTmatR} for the
\LayerQuality, $\GANMATHT=\mathbb{I}_{M}-\OVERLAP$,
giving a  Boltzmann distribution and the corresponding \ThermalAverage.  
\michaeladdressed{@charles: I couldntt find where we actually had the equation $\EPSL(\mathbf{R}) = 1 - \sqrt{\OLAPSQD}$ in the main text; where is it?}
\charles{@michael: \EQN~\ref{eqn:EPSL_mat}}
Expanding this out, we have
\begin{align}
  \label{eqn:IZFE0}
  -\IZFE =& -  \ln \int d\mu(\mathbf{S}) \exp\left[-N\beta \operatorname{Tr}[\HANHT(\OVERLAP)]  \right] \\
\end{align}
We could also express $\IZFE$ In terms of the matrix-generalized \EffectivePotential $\EPSL(\OVERLAP)$
(\EQN~\ref{eqn:EPSL_mat}), giving
\begin{align}
  -\IZFE =& -  \ln \int d\mu(\mathbf{S}) \exp\left[-N\beta \EPSL(\OVERLAP)  \right] 
\end{align}
In analogy with \EQN~\ref{eqn:Gan_highT}, 
as $\HANHT(\mathbf{R})=M-\OVERLAP$,  write
\begin{align}
-\IZFE  =& -  \ln \int d\mu(\mathbf{S}) \exp\left[-N\beta \operatorname{Tr}[M-\OVERLAP]  \right] 
\end{align}
Using the approximation $\operatorname{Tr}[\OVERLAP]\approx\sqrt{\OLAPSQD}$, we have
\begin{align}
  -\IZFE 
\approx& - \ln \int d\mu(\mathbf{S}) \exp\left[-N\beta(M-\sqrt{\OLAPSQD} ) \right] \\ 
\label{eqn:IZFE1}
=& -  \ln \int d\mu(\mathbf{S}) \exp[-NM\beta]\exp\left[N\beta\sqrt{\OLAPSQD}\right], \\
\label{eqn:IZFE2}
=& -  \ln e^{-NM\beta} \int d\mu(\mathbf{S}) \exp\left[N\beta\sqrt{\OLAPSQD}\right], \\
\label{eqn:IZFE3}
=& -  \ln e^{-NM\beta} - \ln \int d\mu(\mathbf{S}) \exp\left[N\beta\sqrt{\OLAPSQD}\right], \\
\label{eqn:IZFE4}
=& NM\beta - \ln \int d\mu(\mathbf{S}) \exp\left[\beta N\sqrt{\OLAPSQD}\right], 
\end{align}

Notice that, as expected, the Free Energy scales $\IZFE$ as $N \times M$.
\noindent
Since \EQN~\ref{eqn:IZFE0} equals \EQN~\ref{eqn:IZFE1}, we can write the \FreeEnergy in terms of $\OLAPSQD$. From \EQN~\ref{eqn:IZFE4}, 
we can identify a generating function ($\Gamma_{\Q}$) for the layer accuracy, or \Quality.
For example, to compute the average \Quality $\Q$, we would use
\begin{align}
  \label{eqn:IZG_Q}
  \beta\Gamma^{IZ}_{\Q} :=  \ln \int d\mu(\mathbf{S}) \exp\left[N\beta\sqrt{\OLAPSQD}\right],
\end{align}
and to compute the average \Quality (squared) $\QT$, we would use
\begin{align}
    \label{eqn:IZG_QT2}
  \beta\Gamma^{IZ}_{\QT} :=  \ln \int d\mu(\mathbf{S}) \exp\left[N\beta\OLAPSQD\right] .
\end{align}
%Moving forward, we drop the $\QT$ subscript, so $\Gamma^{IZ}=\Gamma^{IZ}_{\QT}$, and we
We have recovered \EQN~\ref{eqn:betaIZG_S}.
We can now also define the \LayerQualitySquared \Hamiltonian as
\begin{equation}
      \label{eqn:HBARE}
  \HH_{\QT}:=\mathbf{R}^{\top}\mathbf{R}
\end{equation}
which is a symmetric operator, as desired.
Consequently, we may also write
\begin{align}
    \label{eqn:IZG_QT3}
  \beta\Gamma^{IZ}_{\QT} :=  \ln \int d\mu(\mathbf{S}) \exp\left[N\beta \operatorname{Tr}[\HH_{\QT}]\right]  .
\end{align}
\michaeladdressed{MM TO DO: Note that I just put $\frac{1}{N}$ into those two equations, I don't fully see how we do the averaging to get it.}
\charles{we have to adjust for the new $\tfrac{1}{M}$ term, which I need to think about.
  The number of parameters does not change just because we take the square, but then  why do we need
$\tfrac{1}{M}$ in computing the matrix moments?  This is unreselved}

\subsection{Derivation of the \TRACELOG Condition}
\label{sxn:TraceLogDerivation}


\subsubsection{Setting up the Saddle Point Approximation (SPA)}
\label{sxn:TraceLogDerivation_A}

\michael{Is the point of this subsubsection to derive Eq.~\ref{eqn:IAA_2} to plug into \EQN~\ref{eqn:betaIZG_S} to get the expression in terms of an HCIZ integral.  If so, what is a good name for this subsubsection.}
\charles{To set up the SPA}
As in \EQN~\ref{eqn:IZG_dmuS}, 
we can write \EQN~\ref{eqn:IZG_QT} in terms of the $\mathbf{A}_{2}$ form of the \Student Correlation matrix, 
giving
\begin{align}
\IZG = \ln\INTS  d\mu(\mathbf{S}) \exp\left(N\beta \Trace{ \tfrac{1}{N}\TMAT^{\top} \mathbf{A}_{2} \TMAT } \right)
\end{align}
where $d\mu(\mathbf{S})$ is the measure over all $N \times M$ real-valued random matrices,
although we really want to limit this to all $N \times M$ real matrices that resemble the \Teacher $\TMAT$,
which we clarify below.
\charles{KEEP THIS ? (Also we have add subscripts $(\SMAT, \AMAT, \SVEC, etc.)$ to the integrals as a visual guide for this derivation).}\michael{Let me go through first.}

To transform $\IZG$ into a form we can evaluate using Tanakas result, 
we need to change the measure from an integral over all random $N \times M$ student weight matrices
$d\mu(\mathbf{S})$ to an integral over all $N \times N$
student correlation matrices $d\mu(\mathbf{A})$, i.e., $d\mu(\mathbf{S})\rightarrow d\mu(\mathbf{A})$.
To accomplish this, we can insert an integral over the Dirac Delta function
\begin{align}
  \label{eqn:I}
  \mathbf{I}:=
  \int d\mu(\AMAT_{1})\delta(N\AMAT_{1}-\mathbf{S}^{\top}\mathbf{S}) =
    \int d\mu(\AMAT)\delta(N\AMAT_{1}-\mathbf{S}^{\top}\mathbf{S}).
\end{align}

\noindent
(This is simply a resolution of the Identity.)
This gives
\begin{align}
\label{eqn:IZG_3}
\IZG= \ln \INTS d\mu(\mathbf{S})\INTA d\mu(\mathbf{A})
           \delta\left( N\mathbf{A}_{1}-\mathbf{S}^{\top}\mathbf{S} \right) 
           e^{ N\beta Tr[\tfrac{1}{N} \TMAT^{\top} \mathbf{A}_{2}\TMAT ] } ,
\end{align}
where $d\mu(\mathbf{A})=\Probab{\mathbf{A}}d\mathbf{A}$ and $\Probab{\mathbf{A}}$ is the
(still unspecified) probability density over the new random matrix $\mathbf{A}$. 
%
Let us express \EQN~\ref{eqn:IZG_3} at large-$N$ as
\begin{align}
  \label{eqn:IZG_4}
  \lim_{N\gg 1}\IZG =
  \lim_{N\rightarrow\infty}\ln
  \int d\mu(\mathbf{A})
  \int d\mu(\mathbf{S})
  \delta(N\AMAT_{1}-\mathbf{S}^{\top}\mathbf{S})
  e^{ N\beta Tr[ \tfrac{1}{N}\TMAT^{\top}\mathbf{A}_{2}\TMAT]) }  .
\end{align}
 
Now we assume we can first evaluate the term 
\begin{align}
  \lim_{N\gg 1} \int d\mu(\mathbf{S})    \delta(N\AMAT_{1}-\mathbf{S}^{\top}\mathbf{S})
\end{align}
at large-$N$ using a \SaddlePointApproximation (SPA).


Using the relation, 
%\begin{align}
%\label{eqn:DeltaA}
%\delta(N\mathbf{A}-\mathbf{S^{\top}S})
%   = \dfrac{1}{(2\pi)^{N\red{M}/2}}\INTAHAT  d\mu(\AHAT) e^{ iN Tr[\AHAT\mathbf{A}] } e^{ -i Tr[\AHAT\mathbf{S^{\top}S}] } 
%\end{align}
\begin{align}
\delta(N\AMAT_{1}-\SMAT^{\top}\SMAT)
   =\mathcal{N_1}\INTAHAT  d\mu(\AHAT) e^{ iN Tr[\AHAT\AMAT_{1}] } e^{ -i Tr[\AHAT\SMAT^{\top}\SMAT] }  ,
\end{align}
where $\AHAT$ is a  $M \times M$ auxiliary matrix, and the domain of integration $d\mu(\AHAT)$ is all $M \times M$ real-valued matrices, and where the normalization $\NORM_{1}$ is
\begin{align}
  \label{eqn:norm_1}
\NORM_1:=\frac{1}{(2\pi)^{NM/2}},
\end{align}
which will cancel out below
(See \EQN~\ref{eqn:int-out-J2}?)
\nred{Is this normlization correct? Or is it
\begin{align}
  \label{eqn:norm_1}
\NORM_1:=\frac{1}{(2\pi)^{M(M+1)/4}},
\end{align}
which enforces the normalization on $M(M+1)/2$ constraints for the $M\times M$ symmetric matrix $\AMAT$
}

This is simply the matrix generalization of 
$\delta(x)=\dfrac{1}{2\pi}\int_{-\infty}^{\infty} e^{i\hat{x}x}d\hat{x}$,
so we can express the delta function as an exponential, giving
\begin{align}
\label{eqn:Q2}
\IZG = \NORM_1 \ln\INTS  d\mu(\mathbf{S}) \INTA d\mu(\AMAT_{1}) 
                           \INTAHAT d\mu(\AHAT) e^{ iN Tr[ \AHAT\AMAT_{1} ] }
                           e^{ -i Tr[ \AHAT\mathbf{S}^{\top} \mathbf{S} ] }
                           e^{  N\beta Tr[\tfrac{1}{N} \TMAT^{\top}\mathbf{A}_{2}\TMAT ] } .
\end{align}

Rearranging terms, we obtain 
\begin{align}
\label{eqn:IZG_Gamma1}
\IZG =  \ln\INTA  d\mu(\mathbf{A}) 
            e^{  N\beta Tr[\tfrac{1}{N} \TMAT^{\top}\mathbf{A}_{2}\TMAT ] } \times
           \Gamma_1  ,
\end{align}
where we define $\Gamma_1$ as 
\begin{equation*}
\Gamma_1 := \Gamma_1(\AMAT_{1}) 
         = \NORM_1 \INTS d\mu(\mathbf{S}) 
                           \INTAHAT d\mu(\AHAT) e^{ iN Tr[ \AHAT\AMAT_{1} ] }
                                                           e^{ -i Tr[ \AHAT\mathbf{S}^{\top} \mathbf{S} ] } .
\end{equation*}
We can simplify the complex integral in $\Gamma_1$ with the Wick Rotation $i\AHAT\rightarrow\AHAT$.
\charles{The Wick rotation ensures that the Gaussian integral converges; need to check the signs here}
We may expect $d\mu(\AHAT)$ to be invariant to rotations in the complex plane
so the Wick rotation does not introduce any complex prefactors.  This gives
\begin{eqnarray}
\label{eqn:QWick}
\Gamma_1 = \NORM_1 \INTS d\mu(\mathbf{S})\INTAHAT d\mu(\AHAT) 
           e^{ N  Tr[ \AHAT\AMAT_{1} ] }
           e^{ - Tr[ \AHAT\mathbf{S}^{\top} \mathbf{S}] } \\
\label{eqn:QWick2}
         = \NORM_1 \INTS d\mu(\mathbf{S})\INTAHAT d\mu(\AHAT) 
           e^{ N  Tr[ \AHAT\AMAT_{1} ] }
           e^{ -Tr[ \mathbf{S}\AHAT\mathbf{S}^{\top} ] } ,
\end{eqnarray}
where the second line follows since the trace is invariant under cyclic permutations (i.e., $\Trace{ABC}=\Trace{BCA}=\Trace{CAB}$).
Swapping the order of the integrals yields
\begin{eqnarray}
\label{eqn:QWick3}
\Gamma_1 =\Gamma_1(\AHAT)  = \NORM_1
           \INTAHAT d\mu(\AHAT) 
           e^{ N Tr[\AHAT\AMAT_{1} ]}\times
           \Gamma_2  ,
\end{eqnarray}
where we define $\Gamma_2$ as
\begin{equation*}
\Gamma_2 := \Gamma_2(\AHAT)
         = \INTS d\mu(\mathbf{S})
           e^{ -Tr[ \mathbf{S}\AHAT\mathbf{S}^{\top} ] } .
\end{equation*}

To evaluate $\Gamma_2$, we will make several mathematically convenient approximations.
(These will yield an approximate expression which can be verified empirically.)
\michaeladdressed{MM TO DO: point explicitly to the main text where we made these explicit, if/when that is the case.}
%
We first assume for the purpose of changing measure that the (data) columns of $\mathbf{S}$ are
statistically independent, so that the measure $d\mu(\mathbf{S})$ factors into $N$ gaussian distributions
\begin{align}
\label{eqn:dMuS}
d\mu(\mathbf{S)} = \prod_{\mu=1}^{N}d\mu(\mathbf{s}_{\mu})=\prod_{\mu=1}^{N}d\mathbf{s}_{\mu} ,
\end{align}
where $\mathbf{s}_{\mu}$ is an M-dimensional vector.
The singular values of $\mathbf{S}$ are invariant to randomly permuting the columns or rows,
so the resulting ESD does not change.  
This is very different from permuting $\mathbf{S}$ element-wise, which will make the resulting ESD Marchenko Pastur (MP).

Using \EQN~\ref{eqn:dMuS}, 
$\Gamma_2$ reduces to a simple Gaussian integral, which can be evaluated as a product of $N$ Gaussian integrals (over the $M\times M$ matrix $\AHAT$)
\begin{align}
\label{eqn:int-out-J2}
\Gamma_2 
   =& \left[\INTsvec d\mathbf{s}e^{-\tfrac{1}{\sigma^{2}} \mathbf{s}\AHAT\mathbf{s}^{\top} }\right]^{N} \\
   =& \left[\NORM_{2}\;\Det{\AHAT}^{-1/2}\right]^{N}  ,
\end{align}
where the normalization $\NORM_2$
\begin{align}
\label{eqn:norm_2}
%\NORM_2 := \left((2\pi)^{N/2}\sigma^{2}\right)^{M}  ,
\NORM_2 := \left(\pi\sigma^{2}\right)^{M/2}  ,
\end{align}
\red{Moving forward, we need to fix the prefactors}
where $\sigma^{2}=\mathbf{s}^{\top}\mathbf{s}=\red{2?}1/N$ ??
\charles{Need to be very careful here.  Is this $1/N$ or $1/M$ ?  See also A2. We pick $\sigma^{2}=1/M$ to ensure the normalization on $\XI$ is correct.
This needs to be double checked.}

\nred{check typos again here}\\
For any square, non-singular matrix $\AHAT$,  $ \Trace{\ln\AHAT}=\ln \Det{\AHAT}$ , so
it follows from \EQN~\ref{eqn:int-out-J2} that
\begin{align}
\nonumber
\ln\Gamma_2 
   &=N\ln\NORM_2\left[( det \AHAT)^{-1/2} \right]  \\  \nonumber %  &=\tfrac{NM}{2}\ln (2\pi)^{N/2}\sigma^{2}-\tfrac{N}{2}\ln det\;\AHAT     \\ \nonumber
&= N\ln\NORM_2 -\tfrac{N}{2}\Trace{ \ln\AHAT }  ,
\end{align}
so that
\begin{align}
\label{eqn:log-Gamma}
\Gamma_2 = \NORM_{2}^{N} e^{ -\tfrac{N}{2} Tr[ \ln\AHAT ] } 
\end{align}

Substituting
\EQN~\ref{eqn:log-Gamma}
into \EQN~\ref{eqn:QWick3},
we can write $\Gamma_1$ as
\begin{eqnarray}
  \label{eqn:gamma1}
\Gamma_1(\AHAT)  =& C_{\Gamma_1}\INTAHAT d\mu(\AHAT)   e^{ N Tr[ \AHAT\AMAT_{1} ] }  e^{ -\tfrac{N}{2} Tr[ \ln\AHAT ] }  ,
\end{eqnarray}
where
\begin{equation}
    C_{\Gamma_1}:=\NORM_1 e^{\NORM_2}  .
\end{equation}

We now can evaluate the integral in \EQN~\ref{eqn:QWick3} over the Lagrange Multiplier $\AHAT$ (i.e., $\INTAHAT $). 
If we call this $\Gamma_1(\AHAT)$,
then (following Tanaka~\cite{Tanaka2008}) we can define the \emph{\RateFunction} $I(\AHAT,\AMAT_{1})$ such that
\begin{align}
\label{eqn:LambdaA}
\Gamma_1(\AHAT)=\INTAHAT  d\mu(\AHAT) e^{-NI(\AHAT,\AMAT_{1})}  ,
\end{align}
where
\begin{align}
\label{eqn:IAA}
I(\AHAT,\AMAT_{1}) = -\Trace{ \AHAT\AMAT_{1}} + \frac{1}{2}\Trace{ \ln\AHAT }  .
\end{align}


We can formally evaluate the integral in \EQN~\ref{eqn:LambdaA} in the large-$N$ limit using a \SaddlePointApproximation (SPA)
(see Section~\ref{sxn:mathP}, \EQN~\ref{eqn:SPA}), as
\begin{align}
\label{eqn:LAMBDA}
\Gamma_1(\AHAT)\rightarrow \sqrt{\dfrac{(2\pi)^{N/2}}{N\Vert I\Vert}}e^{-N I^{*}(\AHAT, \AMAT_{1})}  ,
\end{align}
where $I^{*}(\AHAT,\mathbf{A})$ is the maximum value, obtained using
\begin{align}
  \label{eqn:IAA-sup}
  I^{*}(\AHAT,\AMAT_{1}) :=
\underset{N\gg 1}{\lim} I(\AHAT,\AMAT_{1}) =
 \underset{\AHAT}{\sup}\left[-\Trace{\AHAT\AMAT_{1}}+\frac{1}{2}\Trace{\ln\AHAT}\right]  ,
\end{align}
where $I$ at the SPA is defined as
\begin{align}
  \label{eqn:SP0}
  I:=   \dfrac{\partial}{\partial\AHAT}I(\AHAT,\AMAT_{1}) =& -\AMAT_{1}+\dfrac{1}{2\AHAT}=0  
\end{align}
and  $I$ is defined as
\begin{align}
  \label{eqn:Ixx}
I=\frac{\partial^2 }{\partial \hat{A}^2} I(\AHAT,\AMAT_{1})= -\frac{1}{2} \left( \frac{1}{2} \hat{A}^{-1} \right) \otimes \left( \frac{1}{2} \hat{A}^{-1} \right) = -\frac{1}{8} \AMAT_{1} \otimes \AMAT_{1}
\end{align}
where $\otimes$ is the Kronecker product, and $\AMAT_{1} \otimes \AMAT_{1}$ is the Hessian of $\AMAT_{1}$.

Solving the SPA equation, we find that the auxiliary matrix is $\AHAT=\tfrac{1}{2}\mathbf{A}_{1}^{-1}$
and the prefactor (Hessian) is given as
$ det \left( -\frac{1}{8} \AMAT_{1} \otimes \AMAT_{1} \right) = \left( -\frac{1}{8} \right)^{M^2} \left(  det (\AMAT_{1}) \right)^M$.

\nred{ double check  prefactors.}


Substituting for $\AHAT$ into \RateFunction (\EQN~\ref{eqn:IAA}), $I$ becomes
\begin{align}
\label{eqn:IAA_2}
I^{*}(\AHAT,\AMAT_{1}) = -\Trace{ \mathbb{I}_{M} } + \frac{1}{2}\Trace{ \ln\AMAT_{1} }   \\ \nonumber
 = -M + \frac{1}{2}\Trace{ \ln\AMAT_{1} }  .
\end{align}

\noindent
In order for this result to be physically meaningful, 
we need that if $I^{*}(\AHAT,\AMAT_{1})$ grows,
then it must grow slower than $N$, and,
more importantly, that $\Det{\mathbf{A}}$ be non-zero.
Importantly, When $\Det{\mathbf{A}}=1$ exactly, however, then $\Gamma_1$ becomes a constant,
and this simplifies things considerably!

\subsubsection{Casting the \GeneratingFunction $(\IZG)$ as an HCIZ Integral}
\label{sxn:TraceLogDerivation_B}

In this section, we express the \GeneratingFunction $\IZG$, 
given in \EQN~\ref{eqn:IZG_dmuS} (equivalently, in \EQN~\ref{eqn:betaIZG_S}), 
as an HCIZ Integral, 
as given in \EQN~\ref{eqn:IFA2_braket}.
\michael{@charles: make sure I got those numbers correct.  BTW, I think it makes sense to explicitly call out the HCIZ equation by itself in a self-contained statement, so then we can point to it rather than \EQN~\ref{eqn:IFA2_braket}, which uses HCIZ in our context; if so, would you do that.}
\charles{You mean like what you did in Appendix A6, \EQN~\ref{eqn:hciz2} ?  Thats good, and that can go early on, in section 3 or 4}

Inserting $I^{*}(\AHAT,\mathbf{A})$ from \EQN~\ref{eqn:IAA_2} into $\IZG$, we obtain
\begin{align}
  \label{eqn:IZG_IAA}
  \IZG 
  & =  \ln \left[ C_{\Gamma_1} e^{-NM}\int d\mu(\AMAT)
  e^{N\beta Tr[ \tfrac{1}{N}\TMAT^{\top}\AMAT_{2}\TMAT] }
  e^{\tfrac{N}{2}\ln(\Det{\AMAT_{1}})}\right]  \\ \nonumber
  & =
    \ln  C_{\Gamma_1}
  - \red{N}M
  +  \ln \left[ \int d\mu(\AMAT)
    e^{N\beta Tr[ \tfrac{1}{N}\TMAT^{\top}\AMAT_{2}\TMAT] }
    e^{\tfrac{N}{2}\ln(\Det{\AMAT_{1}})}\right]  .
\end{align}

\noindent
%Since $C_{\Gamma_1}$ is complex, we must choose a branch to evaluate $\ln\;C_{\Gamma_1} $.
%However, this term is constant and is bounded by $1$, 
%so we simply need to choose the branch such that $\ln\;C_{\Gamma_1}$ vanishes in the large-N limit $\lim_{N\rightarrow\infty}\tfrac{1}{N}\IZG$.
So long as  the second term $\Trace{\mathbb{I}_{M}}$ does not depend on $N$, 
it will vanish when we take the partial derivative of $\IZG$ to obtain the $\AVGNNGE$, in which case it is not important.  
We can then simply write the \GeneratingFunction $\IZG$  as in \EQN~\ref{eqn:IFA2_integral} as:
\begin{align}
  \label{eqn:IZG_integral}
  \IZG 
   =  \ln \left[ \int d\mu(\AMAT_{1})
    e^{N\beta Tr[ \tfrac{1}{N}\TMAT^{\top}\AMAT_{2}\TMAT] }
    e^{\tfrac{N}{2}\ln(\Det{\AMAT_{1}})}\right]  ,
\end{align}
or, in \BraKet notation, as
\begin{align}
  \label{eqn:IZG_braket}
  \IZG = 
   \ln\left\langle
  e^{N\beta Tr[ \tfrac{1}{N}\TMAT^{\top}\AMAT_{2}\TMAT] }
  e^{\tfrac{N}{2}\ln(\Det{\AMAT_{1}})}
  \right\rangle_{\AMAT}   .
\end{align}


%In order to evaluate this integral, we must deal with the term $ln(\Det{\AMAT_{1}})$.
%To do so, we will restrict both $\AMAT_{1}$ and $\AMAT_{2}$ to a low-rank subspace where
%$\Det{\AMAT}=\Det{\AMAT_{1}}=\Det{\AMAT_{2}}$
%is well defined, finite, and empirically measurable.

%
%
%\paragraph{Restricting $\mathbf{A}$ to the Effective Correlation Space (\ECS).}
%
%As discussed Section~\ref{sxn:matgen}, 
%we expect the \Student \CorrelationMatrix $\mathbf{A}$ to resemble the \Teacher \CorrelationMatrix $\mathbf{X}$;
%and, specifically, we expect both $\mathbf{A}$ and $\mathbf{X}$ to have the same limiting ESD,
%$\rho^{\infty}_{\mathbf{A}}(\lambda)=\rho^{\infty}_{\mathbf{X}}(\lambda)$.
%Likewise, we might expect the $\Det{\mathbf{A}}$ to equal the empirically measured one, i.e.,
%$\mathbb{E}[\Det{\mathbf{A}}]=\Det{\mathbf{X}}$.
%%%and we are done. But 
%In practice, 
%however, 
%%%$\mathbf{\top}$ is \HeavyTailed Power-Law for the best models, and empirically ,
%$\Det{\mathbf{X}}\approx 0$,
%even for very good models that have ESDs that are well-pit to PL or TPL distributions, 
%because there are frequently a large number of very small eigenvalues ($\ln\lambda<1.0$) in the bulk part of the ESD. 
%
%For this result to be physically meaningful, we must restrict the Correlation Matriecs
%($\mathbf{A}$ and $\mathbf{X}$) to an \EffectiveCorrelationSpace (\ECS), 
%i.e., $\AECS$ (and $\XECS$), such that we $\mathbb{E}[\Det{\AECS}]=\Det{\XECS}>0$.
%We can now express $\IZG$ as an HCIZ integral in \EQN~\ref{eqn:IZG_integral} as
%\begin{align}
%  \label{eqn:IZG_integral2}
%  \IZG = \red{\cancel{\frac{1}{N}}}
%  \ln \int d\mu(\AECS)
%  e^{N\beta Tr[ \tfrac{1}{N}\TMAT^{\top}\AECS_{2}\TMAT] }
%  e^{\tfrac{N}{2}\ln(\Det{\AECS})}  ,
%\end{align}
%where have now written $\AECS_{2}$ explicitly for clarity.
%\michael{We are talking about cutting off small eigenvalues, but then we seem to swap $\AECS_{1}$ and $\AECS_{2}$; need to clarify something.}
%\michael{I think it is okay; MM to confirm.}
%We may also write this using \BraKet notation as
%\begin{align}
%  \label{eqn:IZG_braket2}
%  \IZG = 
%  \ln \left\langle
%  e^{N\beta Tr[ \tfrac{1}{N}\TMAT^{\top}\AECS_{2}\TMAT] }
%  e^{\tfrac{N}{2}\ln(\Det{\AECS})}
%  \right\rangle_{\AECS}  .
%\end{align}
%
%
%\paragraph{Independent Fluctuation Approximation (IFA).}
%
%We now introduce a key approximation, the IFA, 
%in which we factor the terms in \EQN~\ref{eqn:IZG_braket2} into two distinct, 
%statistically independent averages over $\AECS$, giving
%\begin{align}
%  \label{eqn:IFA}
%  \IZG \approx
%  \ln\left[
%  \left\langle
%  e^{N\beta Tr[ \tfrac{1}{N}\TMAT^{\top}\AECS_{2}\TMAT] }
%  \right\rangle_{\AECS}
%  \left\langle
%  e^{\tfrac{N}{2}\ln(\Det{\AECS})}
%  \right\rangle_{\AECS} \right]
%\end{align}
%
%\noindent
%Recall that we expect the \Student Correlation matrix to resemble that of the \Teacher.
%By resemble, we mean that both matrices have the same limiting ESD, $\rho^{infty}_{\AECS}=\rho^{infty}_{\XECS}$.
%We also now assume we can estimate the expected value of $\Det{\AECS}$ with the empirical estimate
%from the \Teacher, i.e.,
%\begin{align}
%  \langle \Det{\AECS}\rangle_{\AECS}\approx  \langle \Det{\XECS} \rangle_{\XECS} .
%\end{align}
%
%Fortunately, and rather remarkably, it turns out that when the PL exponent $\alpha=2$, we can
%select $\XECS$ such that $\Det{\XECS}=1$ by simply defining the ECS as the eigen-components spanned by the PL tail.
%Therefore, in this analysis, we will replace $\mathbf{A}$ with $\AECS$, i.e., $\mathbf{A}\rightarrow\AECS$ 
%and set $\Det{\AECS}=1$.
%\michael{MM: Iths par should maybe be moved to the main text.}
%
%\paragraph{The Final Generating Function $\IZG$ for the Quality (Squared) as an HCIZ Integral.}
%With this, we can write $\IZG$ as an HCIZ integral as in \EQN~\ref{eqn:hciz}
%\michael{@charles: what equation should that HCIZ reference be.}
%in \BraKet notation as
%\begin{align}
%  \label{eqn:IZG_final_hciz}
%  \IZG := & \red{\cancel{\frac{1}{N}}}\left\langle     \exp ( N\beta \Trace{\tfrac{1}{N} \TMAT^{\top}\AECS_{2}\TMAT })\right\rangle_{\AECS}  ,
%\end{align}
%or explicitly as an integral as
%\begin{align}
%  \label{eqn:IZG_final_integral}
% \IZG  := & \red{\cancel{\frac{1}{N}}}\int d\mu(\AECS) \exp \left( N\beta \Trace{\tfrac{1}{N} \TMAT^{\top}\AECS_{2}\TMAT }\right)  ,
%  \end{align}
%or in Tanakas notation as
%\begin{align}
%  \label{eqn:IZG_final_tanaka}
% \IZG  := & \red{\cancel{\frac{1}{N}}}\mathbb{E}_{\AECS}\left[ \exp \left( N\beta \Trace{\tfrac{1}{N} \TMAT^{\top}\AECS_{2}\TMAT }\right)\right]  ,
%\end{align}
%where in each case $\AHAT$ is restricted to the \EffectiveCorrelationSpace (\ECS) 
%such that $ln\Det{\AHAT}=0$ and $\AHAT$ is normalized to $M$.
%\nred{CLEAN THIS UP: As discussed in SubSection~\ref{sxn:appendix_Gan}}.
%
%We note that
%$$ \langle\cdots\rangle_{A} = \int \;[\cdots]\; \; d\mu(A) = \mathbb{E}_{A}[\cdots] $$
%are equivalent notations denoting the expected value (or average) over all \Student Correlation Matrices $\AECS$ spanning the ECS,
%$\AECS_{2}$ is a (random) $N\times N$ square (correlation) matrix, and
%$\TMAT$ is a (non-random) \Teacher $N\times M$ rectangular (weight) matrix.
%That is, $\TMAT$ the actual layer weight matrix $\TMAT=\tilde{\mathbf{W}}$ of the model we wish to study
%(but also only in the span of the ECS).
%Notice also that, here,  $\beta$ is the inverse-Temperature and not the simple constant $1$ or $2$ as in \cite{Tanaka}.
%\michael{It seems like this par should be somewhere else. Maybe the math section.}
%




\subsection{MLP3 Model Details}
\label{sxn:appendix_MLP3details}
The empirical MLP3 Model implements the assumptions described in Section~\ref{sxn:matgen} used the following 
procedures:

A three-layer \MultiLayerPerceptron was trained for classification on the MNIST dataset\cite{MNIST1998}. The first Fully 
Connected (FC) hidden layer has 300 units, the second FC hidden layer has 100 units, and the third FC layer has ten 
units for classification, matching the ten digit classes of MNIST. Input images are grayscale, and were rescaled to the 
$[0, 1]$ range. Following the keras\cite{keras2015} defaults, the weights were initialized using the Glorot 
Normal\cite{GloBen10} method, and the biases were initialized to $0$. Each model was trained using Categorical Cross 
Entropy as the loss function. The loss function was {\em summed} over each mini-batch, which is the default behavior for 
Keras, rather than being {\em averaged}, which is the default for pytorch\cite{pytorch2019}. 

Optimization was carried out by either Stochastic Gradient Descent (SGD) 
without momentum, or the Adam algorithm \cite{kingma2014_TR}. The \LearningRate (LR) was set to 0.01 for SGD, and 0.001 
for Adam. The LR was held constant, i.e., there was no decay schedule. Each algorithm proceeded epoch by epoch until the 
value of the loss function did not decrease by more than 0.0001 for three consecutive epochs. At each epoch, 
the \WW~ tool was used to compute metrics for each layer. Loss values reported are the average loss per labeled example, 
and not the summed loss over each minibatch. Training loss is averaged over all batches in the epoch, whereas test loss 
is evaluated once at the end of the epoch.

In some experiments, only one layer was trained, while the others were left frozen. In other experiments all layers were 
trained. Models were trained using a series of mini-batch sizes ranging from $1$ to $32$. For each separate training 
run, the models were re-initialized to the same starting random weights, all random seeds were reset, and deterministic 
computations were used to train the models.

Separate notebooks are provided for keras and pytorch implementations of the experiments.



\subsection{Tanaka's Result}
\label{sxn:tanaka}

In this section, we will rederive the result by Tanaka~\cite{Tanaka2007,Tanaka2008} that we use in our main derivation,
and, importantly, explain how to address the missing Temperature term.
For completeness, we restate it here using the notation of the main text:
%Tanaka states the as have the following HCIZ (Harish-Chandra/Itzykson-Zuber) integral~\cite{BP2001}
\begin{equation}
  \label{eqn:hciz}
  \lim_{N \gg 1} \frac{1}{N} \ln 
\underbrace{
  \Expected[\mathbf{\AMAT}]{
    \exp\left(\frac{\beta}{2}
    \Trace{\mathbf{W}^{\top} \AMAT_{2} \mathbf{W}}
    \right)
  }
 }_{\text{HCIZ Integral}}
  = \frac{\beta}{2} \sum_{i=1}^{M} \GNORM(\lambda_{i})
\end{equation}
where 
$\mathbf{W}$ is the $N\times M$ \Teacher weight matrix, 
$\AMAT=\AMAT_{2}$ is the $N\times N$ \Student (correlation) matrix, 
but $\beta$ is now the inverse-Temperature(because we are working with real matrices),
and we have added a $\tfrac{1}{2}$ (which will be clear later).
$\GNORM(\lambda)$ is a complex analytic function of the eigenvalues $\lambda$ of (the \Teacher Correlation matrix) $\XMAT$, 
whose functional form will depends on the structure of limiting form of (the \Student) ESD $\rho_{\AMAT}^{\infty}(\lambda)$.
We call it the \emph{\red{Norm?} \GeneratingFunction};
\michael{I think calling $\GNORM(\lambda)$ a Norm \GeneratingFunction will be confusing, unless the thing it generates is precisely a norm, which I think it is not.}
\michael{Also, I think the mathbb is confusing on the $G$ here, but especially on the $Z$ below, since it makes it look like integers, real numbers, etc., and not a function.
}
\charles{Its hard to find different but similar notation for new things.  Suggestions ?}
and we may also write it as $\GNORM(\XMAT)$ below.
\michael{
We refer to Tanaka in at least three sets of different letters: 
$\mathbf{W}$ in \ref{eqn:hciz}; 
$\AMAT$ and $\BMAT$ around \ref{eqn:izgin_def,eqn:hciz_tanaka}; and
$\TMAT$ in \ref{eqn:tanaka_result}.
But this equation here is also referred to as HCIZ, and we refer to HCIZ with different sets of letters.
I am in the process of cleaning this up.
}

To apply this result, we note that
while the term $\beta$ is just a constant in~\cite{Tanaka2008}
($1$ or $2$, depending on whether the random matrix is real or complex),
it is not actually inverse Temperature $\beta=\tfrac{1}{T}$ in the original derivation.
Still, we seek a final result that is linear in $\beta=\tfrac{1}{T}$,
so that we can easily evaluate $\QT$ in the high-T limit, i.e.
$\QT
=\tfrac{\partial}{\partial N}\tfrac{1}{\beta}\IZGINF
=\tfrac{\partial}{\partial \beta}\tfrac{1}{N}\IZGINF$
(see \ref{eqn:IZG_QT}).
We can introduce $\beta=\tfrac{1}{T}$ by
simply changing the scale of $\AMAT_{2}$ since the final result is a sum of \RTransforms, which by definition
are linear, i.e. $\GNORM(\beta\lambda)=\beta \GNORM(\lambda)$, however, it is instructive
rederive the final result, with $\beta$ explicitly included.
\michaeladdressed{@charles: what is the point of this comment?  Is it just that if temperature is constant, then the derivative is simple, since we dont need the chain rule.  Also, this comment is about applying this result; do we need it for the derivation?}

%%Also, notice there is an extra $\tfrac{1}{2}$ term in the exponential; this term will factor out later, but it is added here to make it easier to see how to apply the \emph{Large Deviation Principle} (LDP); this not essential but is nice to have
%%\nred{I may remove this}
%%\michael{I would say include the 1/2, for expositional clarity; it helped me.}


\paragraph{Notation.}

We start by re-writing the Tanaka result, \EQN~(\ref{eqn:hciz}),
in our notation for the expected value $\Expected[\AMAT]{\cdots}$ operator, as follows:
\begin{equation}
\label{eqn:hciz2}
  \tfrac{1}{2}\IZGINF = \lim_{N\gg 1} \ln \underbrace{ \int d\mu(\mathbf{\AMAT})\left[\exp\left(\frac{\beta}{2}\Trace{\mathbf{W}^{\top}\AMAT_{2}\mathbf{W}}\right)\right] }_{\mbox{HCIZ Integral}} 
  = N\beta\tfrac{1}{2}\sum_{i=1}^{M}\GNORM(\lambda_{i})   .
\end{equation}
where we have added a $\tfrac{1}{2}$ for technical convenience (to make the connection with the LDP, below).
\michaeladdressed{What are we trying to rewrite? Since this equation is not just \EQN~(\ref{eqn:hciz}) with a different notation for the expectation: it has a new LHS and it doesnt have the RHS of \EQN~(\ref{eqn:hciz}).}
If we denote the internal HCIZ integral as 
%%MM%% $\HCIZ$,  which
%%MM%% denoting the \PartitionFunction for our matrix generalization of the ST model.
%%MM%% This gives
\begin{equation}
\label{eqn:hciz_def}
  \HCIZ := \int d\mu(\mathbf{\AMAT})\left[\exp\left(\frac{\beta}{2}\Trace{\mathbf{W}^{\top}\AMAT_{2}\mathbf{W}}\right)\right]  ,
 \end{equation}
then it holds that %% Thus, %%MM%% such that
\begin{equation}
  \label{eqn:hciz_def2}
  \IZG :=  \ln\HCIZ  ,
\end{equation}
from which it follows that %%MM%% or, equivalently,
\begin{equation}
\label{eqn:hciz_def3}
  \IZGINF := \lim_{N \gg 1} \ln\HCIZ  .
\end{equation}
%%%CHM%%%\charles{Should this be $\beta\mathbf{\bar{\Gamma}}^{IZ}_{N\gg 1}$,
%%%CHM%%%  specifically with the bar to indicate that this is an average.
%%%CHM%%%  That is, did we get the $1/N$ term right here and in A4?
%%%CHM%%%  And if so, when we defined the generalization error above in Eqn~\ref{eqn:avgge_def},
%%%CHM%%%  did we use $F$ or $\bar{F}$. Or does this depend on how we defined $\HANHT$ ?
%%%CHM%%%  Getting this $1/N$ right is critical because we need to take something like
%%%CHM%%%  \begin{align}
%%%CHM%%%    \QT =\dfrac{1}{\beta} \dfrac{\partial}{\partial N}\IZGINF =
%%%CHM%%%    \dfrac{1}{N} \dfrac{\partial}{\partial \beta}\IZGINF  
%%%CHM%%%  \end{align}
%%%CHM%%%  But now I'm not entirely sure if this is the Average \LayerQuality
%%%CHM%%%  or the Total \LayerQuality
%%%CHM%%%  Moreover, what what we expect is (from Paris Bouchaud, etc)
%%%CHM%%%  \begin{align}
%%%CHM%%%    \HCIZ \rightarrow_{N \gg 1} \exp\left[N\beta\sum \GNORM\right]
%%%CHM%%%  \end{align}
%%%CHM%%%  which gives
%%%CHM%%%  \begin{align}
%%%CHM%%%    \IZGINF = \ln\HCIZ_{N \gg 1} = N\beta\sum \GNORM
%%%CHM%%%  \end{align}
%%%CHM%%%}
\michaeladdressed{Having the mathbb on this $Z$ seems pretty confusing, since it looks like a set of integers. What are the drawbacks of just using $Z$, or some other related letter?}
%%
%%\michael{Those two expressions are consistent with what is going on in the main text, but from the persepctive of this self-contained appendix, they are basically definitions to simplify the derivation of \EQN~(\ref{eqn:hciz}), correct?  }
%%
The SPA approximates the \PartitionFunction $\HCIZ$, which is now an HCIZ integral,  by its peak value.
For this, $\GNORM(\lambda)$ itself must either not explicitly depend on $N$ and/or at least not grow faster than $N$.
%\nred{For that reason, I think we need to normalize the eigenvalue as $\lambda/N$, or rather $\lambda/M$, or maybe $\lambda/\MECS$, as with the normalization constrain on $\XECS$.}

The trick here is we can choose an \RTransform of $\mathbf{\AMAT}$
that is a simple analytic expression based on the observed
the empirical spectral density (ESD) of the $\mathbf{X}$.
And this can readily be done for the ESDs for a wide range of layer weight matrices
observed in modern DNNs because the their ESDs are \HeavyTailed \PowerLaw\cite{MM19_HTSR_ICML}.
We can then readily express the \Quality $\Q$ of the \Teacher
layer in a simple functional form, (i.e  an approximate Shatten Norm),

Importantly, the matrices $\mathbf{X}$  and $\mathbf{\AMAT}$ must be well approximated
by low rank matrices since the derivation in Tanaka requires this.  Fortunately,
this appears to be generally true for the layers in very well trained DNNs,
which is what allows us to apply this withing the~\ECS.

Finally, we note that $\GNORM(\mathbf{X})$ is kind of \emph{Generalized Norm} because 
it can be evaluated as a sum over a function of the $M$ eigenvalues $\lambda_{\mu}$ of the \Teacher
correlation matrix $\mathbf{X}=\frac{1}{N}\mathbf{W}^{\top}\mathbf{W}$.
$\GNORM(\mathbf{X})$  will turn out to be a simple expression similar to the Frobenius Norm or the
Shatten Norm of $\mathbf{X}$, depending on the functional form we choose to model the
lmiting form of the \Student ESD, $\rho_{\AMAT}^{\infty}(\lambda)$.

\michael{Those three paragraphs are more like comments, so shouldnt be here in the appendix. Put at the right spot in the main text.}
\charles{@michael: where ? }
\nred{BELOW: The normalization factors below may be off since these are symmetric matrices.
  For example, for the $M\times M$ $\AMAT$, there are $M(M+1)/2$, not $M^2$, degrees of freedom.
  These minor errors are not material to the derivations below but they needs to be fixed before publication.
  }
\subsubsection{Setup and Outline}
\label{sxn:tanaka_setup}

\noindent
To evaluate \ref{eqn:hciz2},
we %%MM%% In setting up the problem for this paper, we 
want to integrate over all \Student Correlation matrices $\mathbf{\AMAT}$
that ``resemble the \Teacher Correlation matrix $\mathbf{X}$.  
To %%MM%% So, first, we need to 
formalize this idea,
we need to %%MM%% which requires that we 
define the measure over ``all desired $\mathbf{\AMAT}$, $d\mu(\mathbf{\AMAT})$, 
in 
terms of the actual $M$ eigenvalues, $\left\{ \lambda_{i} \right\}_{i=1}^{M}$, of the \Teacher.


\paragraph{Randomness assumption.}
For real weights $\mathbf{W}$,  we assume an \emph{orthogonally invariant} ensemble,
$d\mu(\mathbf W)=d\mu(\mathbf U\mathbf W\mathbf U^{\top})$ for all $\mathbf U\in O(M)$,
mirroring the isotropic Gaussian initialisation widely used in neural networks.
Crucially, Tanaka’s large-$N$ analysis shows the resulting HCIZ exponent depends only on the eigenvalue spectrum, so the final integrated–$R$ expression should remain applicable even when full rotational invariance is later broken in training.

%%\paragraph{Representing $d\mu(\AMAT)$ with $d\mu(\WMAT)$ using the source matrix $\mathbf{D}$.}
\paragraph{Using a source matrix $\mathbf{D}$ to represent $d\mu(\AMAT)$ with $d\mu(\WMAT)$.}

We consider all matrices $\mathbf{\AMAT}$ with the same limiting spectral density, $\rho_{\AMAT}^{\infty}(\lambda)$,
as the limiting (\emph{empirical}) ESD of the \Teacher.
That is, we want $\rho_{\AMAT}^{\infty}(\lambda)=\rho^{\infty}_{\WMAT}(\lambda)$, where $\TMAT=\WMAT$.
%\nred{Comment on the nature of the randomness required to be invariant under unitary transformations.}
Of course, there are infinitely many weight matrices $\mathbf{W}$ with the same $M$ eigenvalues, $\left\{ \lambda_{i} \right\}_{i=1}^{M}$, as the \Teacher.
Let us specify these matrices with the measure $d\mu(\mathbf{W})$.
Doing this lets us then write the measure $d\mu(\mathbf{\AMAT})$ in terms of $d\mu(\mathbf{W})$ as:
\begin{equation}
\label{eqn:dmuA}
d\mu(\mathbf{\AMAT}) 
   := e^{- \frac{\beta}{2} Tr[\mathbf{W}\mathbf{D}\mathbf{W}^{\top}]} d\mu(\mathbf{W})  ,
\end{equation}
where $\mathbf{D}$ is some $M \times M$ matrix, called the \SourceMatrix, to be specified below,
and the $\tfrac{1}{2}$ here as well.
Indeed, the key idea here will be to define $\mathbf{D}$ in such a way as to obtain the desired final result.
Notice also that we have added a $\beta$ term; this will be factored out later.
%Notice that we  define $\mathbf{\AMAT}$ through this change of measure, and 
%$\mathbf{\AMAT}$ should be an $M \times M$ matrix (i.e. $\mathbf{\AMAT}:=\tfrac{1}{N}\mathbf{S}^{T}\mathbf{S}$),
%but we could also let $\mathbf{\AMAT}$ be an $N \times N$ matrix, (i.e $\mathbf{\AMAT}:=\tfrac{1}{N}\mathbf{S}\mathbf{S}^{T}$),
%simply with $N-M$ zero eigenvalues.
%\paragraph{Unitary‐invariant randomness.}
%To ensure the ensemble contains \emph{no information beyond the fixed spectrum} $\{\lambda_i\}$, impose invariance under every orthogonal rotation:
%\begin{equation}
%  d\mu(\mathbf{W}) \;=\; d\mu\!\bigl(\mathbf{U}\mathbf{W}\mathbf{U}^{\top}\bigr),
%  \qquad
 % \forall\,\mathbf{U}\in O(M).
%\end{equation}
%Haar‐averaging over $\mathbf{U}$ renders all eigenvector choices equally likely, so the HCIZ integral collapses to a pure eigenvalue problem; this isotropy drives the saddle‐point/large‐deviation steps in Eqns.~\ref{eqn:ZD_step1}–\ref{eqn:ZD_step2}.


We can now represent the partition function
$\ZD$, by inserting \EQN~\ref{eqn:dmuA} into \EQN~\ref{eqn:hciz_def}.
$\ZD$ is now defined as an integral over all
possible (\Teacher) weight matrices $\mathbf{W}$
\begin{equation}
  \label{eqn:ZD0}
    \ZD=\int d\mathbf{W}\exp[\frac{\beta}{2}\left
    (\Trace{\mathbf{W}^{\top}\AMAT_{2}\mathbf{W}}-\Trace{\mathbf{W}\mathbf{D}\mathbf{W}^{\top}}  
    \right)]  ,
\end{equation}
%where the matrix $\mathbf{D}$ is the called the \SourceMatrix.
\michaeladdressed{@charles: What is this a ``modification of? I get it if it is just a regularized objective, with LAgrange parameters to be determined; but are we going to use it as a generating function or partition function of some effective physical system? Maybe a sentence or two saying why doing this modification makes sense. This is probably just going to be a vector of prices or lagrange parameters that control the constraint satisfaction?}
\michaeladdressed{Also, $Z(D)$ versus $\ZD$?}
Observe that this integral only converges when all the eigenvalues of $\DMAT$, 
$\left\{ \DeltaMu \right\}_{\mu=1}^{M}$, 
are larger than the maximum eigenvalue of $\mathbf{\AMAT}$, i.e., when $\DeltaMu >\lambda_{max}$, for $\mu\in[1,M]$
(although below this will become $\beta\DeltaMu >\lambda_{max}$).
Later, we will place $\mathbf{D}$ in diagonal form, and we will obtain an explicit expression for its $M$ eigenvalues in terms of the $M$ non-zero eigenvalues of $\mathbf{X}$.
The eigenvalues of $\mathbf{D}$ will turn out to Lagrange Multipliers, needed later.


\paragraph{The Saddle Point Approximation (SPA) and the Large Deviation Principle (LDP).}

To evaluate the large-$N$ case of $\IZG$ (see \ref{eqn:hciz_def2},~\ref{eqn:hciz_def3}), 
we assume that the distribution of possible \Teacher correlation matrices,
$\mu(\mathbf{X})$, satisfies a \emph{Large Deviation Principle (LDP)}.
A LDP applies to probability distributions that take an exponential form,
such that $\mu(\mathbf{X})=e^{-N I(\mathbf{X})}d\mu(\mathbf{X})$,
where  $I(\mathbf{X})$ is Entropy or Rate function $I(\mathbf{X})$.
\michael{REF.}
\michael{Maybe write out the general expression, citing that ref, so that it is clear how \ref{eqn:EZD} follows from it, given our setup.}
%\charles{(((
%Is this a resonable assumption ?
%While we model the ESD of $\mathbf{X}$ as a \PowerLaw (PL), it is really
%finite-size distrubtion, best modeled by a Truncated \PowerLaw (TPL),
%either cut off at the largest eigenvalue $\lambda^{max}$
%and/or with an exponential decay~\cite{YTHx22_TR}.
%)))}
In applying a LDP, we effectively restrict measure of student correlation matrices $\mathbf{\AMAT}$
to those most similar to the empirically observed \Teacher correlation matrix $\mathbf{X}$.
%
We expect the measure over all \Teacher correlation matrices
follows an LDP because the ESD is far from Gaussian,
the dominant generalizing components reside in the tail of the ESD,
and at finite-size the tail decays at worst as an exponentially
Truncated \PowerLaw (TPL).

%\nmove{ MOVE THIS BELOW:
%Using the LDP (and following similar approaches in spin glass theory \cite{PP95}),
%below we will show that we can write the expected value of $\ZD$ 
%in terms of $d\mu(\mathbf{X})$ now (as opposed to $d\mu(\mathbf{A})$)
%and in the large-$N$ approximation, as
%\begin{equation}
%  \label{eqn:EZD}
%  \EZDA=
%  \int\exp\left(\beta N Tr[\GNORM(\mathbf{X})]-NI(\mathbf{X})+o(N)\right)d\mu(\mathbf{X})
%\end{equation}
%where $I(\mathbf{X})$ is the \RateFunction, defined below.
%}
%\michael{@charles: Which previous equation from this section does this follow from?}
%\charles{None.  We defined it beliow}


\paragraph{Two steps to evaluate $\Expected[\AMAT]{\ZD}$ in the large-$N$ approximation.}
The goal is to start with \EQN~\ref{eqn:ZD0} and obtain two separate, equivalent
relations, Eqns.~\ref{eqn:ZD_step1} and ~\ref{eqn:ZD_step2}:
%this proceeds in the following two steps:
\begin{enumerate}
   \item
   \textbf{Obtaining an integral transform of $\rho^{\infty}_{\AMAT}(\lambda)$.}
   First, we expand and reduce \EQN~\ref{eqn:ZD0} and evaluate the expected value of
   $\EZDA=\EZDATWO$ in the large-$N$ limit by expressing the $\rho_{\AMAT}(\lambda)$
   for the $N\times N$ matrix $\AMAT=\mathbf{A}_{2}=\tfrac{1}{N}\SMAT\SMAT^{\top}$
   in the continuum representation, i.e., as ]
   $\rho^{emp}_{\AMAT}(\lambda)\rightarrow \rho^{\infty}_{\AMAT}(\lambda)$, to obtain:
   \begin{equation}
      \label{eqn:ZD_step1}
      \lim_{N\gg 1}\dfrac{1}{N}
      \ln\EZDATWO =M\ln(\dfrac{2\pi}{\beta})-\sum_{\mu=1}^{M}\int \ln(\delta_{\mu}-\lambda)\rho^{\infty}_{\AMAT}(\lambda)d\lambda  .
   \end{equation}
   This gives us an $\EZDATWO$ in terms of an integral transform $\rho^{\infty}_{\AMAT}(\lambda)$, which we can model.\footnote{This integral of $\rho^{\infty}_{\AMAT}(\lambda)$  is related to the \emph{Shannon Transform}, an integral transform from information theory that is useful when analyzing the mutual information or the capacity of a communication channel~\cite{Tanaka2007}. }
   \michaeladdressed{I dont think thats obvious, is it? I thought the point was that is is convex/concave when $\delta_{\mu} >\lambda_{max}$, in which case it is a Laplace(?) transformation (which we can then invert)? Or is this something else?}
   \item
   \textbf{Forming the \SaddlePointApproximation (SPA).}
   We evaluate \EQN~\ref{eqn:ZD0} as the expected value of $\EZDA=\EZDAONE$
   for the $M \times M$ matrix $\AMAT=\mathbf{A}_{1}=\tfrac{1}{N}\SMAT^{\top}\SMAT$
   (but explicitly in terms of $d\mu(\mathbf{X})$).
   Then, taking in the large-$N$ approximation using the SPA,
    (and which can be done implicitly using the LDP), we obtain
   \begin{equation}  
  \label{eqn:ZD1} 
  \lim_{N \gg 1} \EZDAONE\simeq\int  \exp\left(\beta N\Trace{\GFANCY}\right)d\mu(\mathbf{X}) \approx \exp(\beta N\GMAX)
\end{equation}
  where  $\GFANCY$ depends on $\GNORM(\mathbf{X})$, and $\GMAX=\sup_{\XMAT}\GFANCY$.
  We can then write
   \begin{equation}
      \label{eqn:ZD_step2}
      \lim_{N \gg 1}\dfrac{1}{N}\ln\EZDAONE \approx \beta\GMAX  ,
   \end{equation}

 \item
 \textbf{Finding the Inverse Legendre Transform.}
  To do this, we now equate
  \begin{equation}
  \lim_{N \gg 1}\frac{1}{N}\ln\EZDAONE=  \lim_{N \gg 1}\frac{1}{N}\ln\EZDATWO
  \end{equation}
 Then, we can form the
 %with a suitable choice for the source matrix $\mathbf{D}$, we can form the
 inverse Legendre transform  which we will let us relate $\GNORM(\lambda)$ in \EQN~\ref{eqn:hciz} to the integrated \RTransform of $\rho^{\infty}_{\AMAT}(\lambda)$.
\end{enumerate}

\noindent

(See~\ref{sxn:tanaka_end}.)



\subsubsection{Step 1. Forming the Integral Transformation of ESD $(\rho_{\AMAT}^{\infty}(\lambda))$}
\label{sxn:tanaka_step1}
We first establish \EQN~\ref{eqn:ZD_step1}, in Steps $1.1-1.4$.
This is done by changing variables under a Unitary transformation, $\mathbf{W}\rightarrow\mathbf{\check{W}}$,
evaluating the resulting functional determinant,
and then taking the continuum limit of the ESD
$\tilde{\rho}_{\AMAT}(\lambda)\rightarrow\rho^{\infty}_{\AMAT}(\lambda)$.

\paragraph{Step 1.1}
%\nred{We need to check all the equations, I might have flipped the transpose in the second term with the source matrix from right to left.}\michael{I think I fixed this, check.}
To do so, let us first assume that \Teacher correlation matrix $\mathbf{X}$ and the source matrix $\mathbf{D}$
are simultaneously diagonalizable  
(i.e., their commutator is zero: $[\mathbf{X}, \mathbf{D}]=0$).
\michael{@charles: where precisely is this assumption used?}
\charles{Its used in Step 1.3 }
In this case, we may write the generating function $\ZD$ in \EQN~\ref{eqn:ZD0} as
%
\begin{align}
\label{eqn:Z-diag}
\ZD &= \int d\mu(\mathbf{W}) \exp\frac{\beta}{2}
 \bigg( 
\Trace{\mathbf{W}^{\top}\mathbf{U}^{\top}\mathbf{\Lambda}\mathbf{U}\mathbf{W}} 
- \Trace{\mathbf{W}\mathbf{V}^{\top}\mathbf{\Delta}\mathbf{V}\mathbf{W}^{\top}} 
\bigg)  ,
\end{align}
\michaeladdressed{How do we get this?  Is this just \ref{eqn:dmuA}, rewritten?  But we have changed $d\mathbf{W}$ to $d\mathbf{X}$ and changed $\beta$ to $1/2$?  Or does this use the assumption $[\mathbf{X}, \mathbf{D}]=0$?}
where we have defined
%
\begin{equation}
\label{eqn:diag-A-D}
    \AMAT_{2}=\mathbf{U}^{\top}\mathbf{\Lambda}\mathbf{U},\;\;
    \mathbf{D}=\mathbf{V}^{\top}\mathbf{\Delta}\mathbf{V}  ,
\end{equation}
%
%%where $\mathbf{U}, \mathbf{V}$ are Unitary matrices
%%with $\mathbf{U}$ is $(N\times N)$ and $\mathbf{V}$ is $(M\times M)$.
where $\mathbf{U}$ ($N\times N$) and $\mathbf{V}$ ($M\times M$) are Unitary matrices.
%
%% 
%% %
%% Using the Orthogonality properties of $\mathbf{U}$ and $\mathbf{V}$,
%% %
%% \begin{equation}\label{eqn:UU}
%%     \mathbf{U}^{T}\mathbf{U}=\mathbf{I},\;\;
%%     \mathbf{V}^{T}\mathbf{V}=\mathbf{I},
%% \end{equation}
Since $\mathbf{U}^{\top}\mathbf{U}=\mathbf{I}$ and $\mathbf{V}^{\top}\mathbf{V}=\mathbf{I}$,
%
we can insert these identities into $\ZD$ in \ref{eqn:Z-diag}, giving
%
\begin{align}\label{eqn:Z0-diag}
\ZD &= \int d\mu(\mathbf{W}) \exp\frac{\beta}{2}\times  \\ \nonumber
&\bigg(\Trace{(\mathbf{V}^{\top}\mathbf{V})\mathbf{W}^{\top}\mathbf{U}^{\top}\mathbf{\Lambda}\mathbf{U}\mathbf{W}(\mathbf{V}^{\top}\mathbf{V})} 
-\Trace{(\mathbf{U}^{\top} \mathbf{U})\mathbf{W}\mathbf{V}^{\top} \mathbf{\Delta} \mathbf{V}\mathbf{W}^{\top}(\mathbf{U}^{\top} \mathbf{U})}\bigg)  .
\end{align}
We can identify the reduced weight matrix $\mathbf{\check{W}}$ as
\begin{equation}
   \label{eqn:Wcheck}
   \mathbf{\check{W}}=\mathbf{U}\mathbf{W}\mathbf{V}^{\top},\;\;
   \mathbf{\check{W}}^{\top}=\mathbf{V}\mathbf{W}^{\top}\mathbf{U}^{\top}  ,
\end{equation}
Rearranging parentheses, this gives 
\begin{align}
\ZD &= \int d\mu(\mathbf{W}) \exp\frac{\beta}{2}\times  \\ \nonumber
&\bigg(\Trace{\mathbf{V}^T(\mathbf{V}\mathbf{W}^T\mathbf{U}^T)\mathbf{\Lambda}(\mathbf{U}\mathbf{W}\mathbf{V}^T)\mathbf{V}}  
-\Trace{ \mathbf{U}^{\top}(\mathbf{U} \mathbf{W}\mathbf{V}^{\top})\mathbf{\Delta}(\mathbf{V}\mathbf{W}^{\top} \mathbf{U}^{\top})\mathbf{U} }\bigg)  .
\end{align}
We can now express $\ZD$ in terms of $\mathbf{\check{W}}$ as
\begin{align}
\label{eqn:hciz-W-red}
\ZD &= \int d\mu(\mathbf{W})\exp\frac{\beta}{2}
 \bigg(\Trace{\mathbf{V}^{\top}\mathbf{\check{W}}^{\top}\mathbf{\Lambda}\mathbf{\check{W}}\mathbf{V}} 
 -\Trace{\mathbf{U}^{\top}\mathbf{\check{W}}\mathbf{\Delta}\mathbf{\check{W}}^{\top}\mathbf{U}}\bigg)  .
\end{align}
Since the Trace operator $\Trace{\cdot}$ is invariant to Unitary (Orthogonal) transformations, we can
now remove the
$\mathbf{U}$ and $\mathbf{V}$ terms, giving the simplified expression
for our generating function $\ZD$ in terms of
the two diagonal matrices $\mathbf{\Lambda}, \mathbf{\Delta}$, 
the reduced weight matrix $\mathbf{\check{W}}$, and
the Jacobian $J(\mathbf{\check{W}})$ transformation for $d\mu(\mathbf{W})\rightarrow d\mu(\mathbf{\check{W}})$, as:
\begin{align}
\label{eqn:hciz-W-red2}
    \ZD & =\int d\mu(\mathbf{\check{W}})J(\mathbf{\check{W}})\exp\frac{\beta}{2}
 \bigg( \Trace{\mathbf{\check{W}}^{\top}\mathbf{\Lambda}\mathbf{\check{W}}} 
       -\Trace{\mathbf{\check{W}}\mathbf{\Delta} \mathbf{\check{W}}^{\top}} \bigg)  .
\end{align}


\paragraph{Step 1.2}

\michael{@charles: we are switching gears here, Im missing something in the flow. Where do we get the next equation from?}
\charles{@michael: standard stuff.  Should we provide a reference ?  Can you find it ?}
We can now evaluate the
integral using the standard relation for the functional determinant for infinite-dimensional Gaussian integrals~\cite{EngelAndVanDenBroeck}

\begin{equation}
\label{eqn:hciz-det}
    \ZD=\left(\dfrac{2\pi}{\beta}\right)^{NM/2}\det\left(\mathbf{\Delta}-\mathbf{\Lambda}\right)^{-1/2}
\end{equation}
where the Jacobian is unity for the Unitary transformation.
\michaeladdressed{Is this $\mathbf{W}$ or $d\mathbf{\check{W}}$?  Also, this is a weight matrix, not an orthogonal matrix correct?  Meaning that the Jacobian being one is due to the Trace-Log condition?}
\charles{@michael:
The transformation $\mathbf{W} \mapsto \check{\mathbf{W}} = \mathbf{U}\,\mathbf{W}\,\mathbf{V}^{\top}$ is orthogonal, so $J(\check{\mathbf{W}})=1$.}

\begin{equation}\label{eqn:Jacobian}
    J(\mathbf{\check{W}})=1  .
\end{equation}
since $\mathbf{W} \mapsto \check{\mathbf{W}}$  is an orthogonal transfomation.
We now use the standard Trace-Log-Determinant relation~\cite{EngelAndVanDenBroeck}
\begin{equation}\label{eqn:tr-ln-det}
    \Trace{\ln\mathbf{M}}=\ln\det\mathbf{M}  .
\end{equation}
Let us insert $(\exp\ln)$ on the R.H.S. of \ref{eqn:hciz-det}, to obtain
\begin{align}
\nonumber
\ZD
  &=\exp\ln\bigg[\left(\dfrac{2\pi}{\beta}\right)^{NM/2}\det\left(\mathbf{\Delta}-\mathbf{\Lambda}\right)^{-1/2}\bigg] \\ 
\nonumber
  & =\exp\bigg[\left(\dfrac{NM}{2}\right)\ln\dfrac{2\pi}{\beta}-\dfrac{1}{2}\Trace{\ln\left(\mathbf{\Delta}-\mathbf{\Lambda}
\right)}\bigg] \\ 
\label{eqn:hciz-exp-ln}
  & =\exp\bigg[\dfrac{NM}{2}\ln\dfrac{2\pi}{\beta}-\dfrac{1}{2}\ln\det\left(\mathbf{\Delta}-\mathbf{\Lambda}\right)\bigg]  .
\end{align}


\paragraph{Step 1.3}
We now want to express
%the Log-Determinant, $\ln\det\left(\mathbf{\Delta}-\mathbf{\Lambda}\right)$,
the generating function $\ZD$ 
in \ref{eqn:hciz-exp-ln}
in terms of an integral over the continuous limiting spectral density
$\rho_{\AMAT}(\lambda)$ of the correlation matrix $\AMAT_{2}$.  

First, we express the Determinant of the matrix $\mathbf{\Delta}-\mathbf{\Lambda}$ in terms of discrete eigenvalues:
\michaeladdressed{I think so; are you asking if we got the $M$ and $N$s correct?}
\charles{check minus sign}
\begin{equation}
\label{eqn:det-discrete}
    \det\left(\mathbf{\Delta}-\mathbf{\Lambda}\right)^{-1/2}=\prod_{\mu=1}^{M}\prod_{i=1}^{N}\left(\DeltaMu-\lambda_{i}\right)^{-1/2}  .
\end{equation}
%
This gives the Log-Determinant in terms of the $M$ (non-zero)
eigenvalues of $\mathbf{D}$ and $\AMAT_{2}$, as
\begin{equation}
\label{eqn:ln-det-discrete}
    \ln\det\left(\mathbf{\Delta}-\mathbf{\Lambda}\right)^{-1/2}=-\dfrac{1}{2}\sum_{\mu=1}^{M}\sum_{i=1}^{N}\ln\left(\DeltaMu-\lambda_{i}\right)  .
\end{equation}
%
We can express %Define 
the ESD, $\tilde\rho_{\AMAT}(\lambda)$, of the 
Student Correlation 
matrix
$\AMAT_N$ in terms of the Dirac delta-function, $\delta(x)$, as
\begin{equation}
\label{eqn:rho-emp}
    \tilde\rho_{\AMAT}(\lambda)=\frac{1}{N}\sum_{i=1}^{N}\delta(\lambda-\lambda_{i})  .
\end{equation}
Using this, the \ExpectedValue of the Log-Determinant 
in \ref{eqn:ln-det-discrete}
can be expressed in terms of the ESD of
$\AMAT_N$ as
\begin{align}
\nonumber
\Expected[\AMAT_N]{\ln\det\left(\mathbf{\Delta}-\mathbf{\Lambda}\right)^{-1/2}}
   & = -\dfrac{1}{2}\sum_{\mu=1}^{M}\sum_{i=1}^{N}\int
       d\lambda\ln(\DeltaMu-\lambda)\delta(\lambda-\lambda_{i}) \\ 
\nonumber
   & = -\dfrac{1}{2}\sum_{\mu=1}^{M}\int
       d\lambda\ln(\DeltaMu-\lambda)\sum_{i=1}^{N}\delta(\lambda-\lambda_{i}) \\ 
\label{eqn:ln-det-rho}
   & = -\dfrac{1}{2}\sum_{\mu=1}^{M}\int
       d\lambda\ln(\DeltaMu-\lambda)
       N\tilde\rho_{\AMAT}(\lambda)  .
\end{align}

Let us insert this back into our
expression for the generating function,
\ref{eqn:hciz-exp-ln},   %% ~\ref{eqn:hciz-exp-ln3}, 
giving
$\EZDATWO$ in terms of the ESD $\tilde{\rho}_{\AMAT}$ as
\begin{equation}
\label{eqn:Z-rho}
    \EZDATWO=\exp\bigg\{\dfrac{N}{2}\big[M\ln\dfrac{2\pi}{\beta}-\sum_{\mu=1}^{M}\int
        d\lambda\ln(\DeltaMu-\lambda)\tilde{\rho}_{\AMAT}(\lambda)\big]\bigg\}  .
\end{equation}
\michael{BTW, I get a $1/\beta$ on the second term, when I derive it, so I think we missed a $\beta$ in the numerator somewhere.}
\charles{Yeah we gotta check all this very carefully}

We can now replace
the sum over the $N$ eigenvalues $\lambda_{i}$ with an integral over the limiting
ESD, $\rho(\lambda)$, to obtain
\begin{equation}
\label{eqn:rho-}
\rho^{\infty}_{\AMAT}(\lambda)=    \lim_{N\rightarrow\infty}\tilde\rho_{\AMAT}(\lambda)  .
\end{equation}
Observe that this effectively means that we are taking a large-$N$ limit, $N\gg 1$.
%
This lets us write the \ExpectedValue of the generating function $\ZD$
in \ref{eqn:Z-rho}
as
\begin{equation}
\label{eqn:Z-rho-2}
    \lim_{N\gg 1}\EZDATWO=\exp\bigg\{\dfrac{N}{2}\big[M\ln\dfrac{2\pi}{\beta}-
    \sum_{\mu=1}^{M}\int
        d\lambda\ln(\DeltaMu-\lambda)\rho^{\infty}_{\AMAT}(\lambda)\big]\bigg\}
\end{equation}
\michael{We just added an expectation. Is that correct? I think we missed something.}
\charles{No, this is correct.  But I obviously did not explain it well.  Will review}

\paragraph{Step 1.4}

Using the Self-Averaging Property,
\begin{equation}
   \ln \EZDATWO \simeq \Expected[\mathbf{A}_M]{\ln \ZD} ,
\end{equation}
It follows from \EQN~\ref{eqn:Z-rho-2}
that
\begin{equation}
   \lim_{N\gg 1} \ln \EZDATWO
   \simeq \dfrac{N M}{2}\ln\dfrac{2\pi}{\beta}
         -\dfrac{N}{2}\sum_{\mu=1}^{M}\int d\lambda\ln(\DeltaMu-\lambda)\rho^{\infty}_{\AMAT}(\lambda)  .
\end{equation}
%
The $N$-dependence now cancels out,
\michael{How? / Why? / Or because of what we do in the next step?}
\charles{@michael We took the log and the exp cancels out}
and we are left an approximate expression due to the remaining dependence of the continuum limiting density
$\rho^{\infty}_{\AMAT}(\lambda)$ (for $\AMAT=\mathbf{A}_{2}$)
%
\begin{equation}
\label{eqn:ln-Z-Nlim2}
    \lim_{N \gg 1}\dfrac{2}{N}\ln \EZDATWO
    = M\ln\dfrac{2\pi}{\beta}-\sum_{\mu=1}^{M}\int d\lambda\ln(\DeltaMu-\lambda)\rho^{\infty}_{\AMAT}(\lambda)  .
\end{equation}
\michael{We still have a factor of 2 issue}
\charles{Where ?  }
This completes the derivation of \EQN~\ref{eqn:ZD_step1}; 
we have an expression for the expected value of $\ZD$,
evaluated in the large-$N$ (continuum) limit.


\subsubsection{Step 2: The Saddle Point Approximation (SPA): Explicitly forming the Large Deviation Principle (LDP)}
\label{sxn:tanaka_step2}
We now evaluate $\EZDA$ in \EQN~\ref{eqn:ZD1} as $\EZDATWO$ 
to  establish \EQN~\ref{eqn:ZD_step2}, \charles{in Steps $2.1-2.6$}.

Using the LDP (and following similar approaches in spin glass theory \cite{PP95}),
below we will show that we can write the expected value of $\ZD$ 
in terms of $d\mu(\mathbf{X})$ now (which is equivalent to $d\mu(\AMATM)$)
and in the large-$N$ approximation, as
\begin{equation}
  \label{eqn:LDP}
 \lim_{N \gg 1} \EZDAONE=
  \int\exp\left(\ND\beta N \Trace{\GNORM(\mathbf{X})}-NI(\mathbf{X})+o(N)\right)d\mu(\mathbf{X})
\end{equation}
where $I(\mathbf{X})$ is \RateFunction, defined below,
and $\GNORM(\mathbf{X})$ is what we are eventually solving for.

\paragraph{Step 2.0} We start with the \emph{expected} \PartitionFunction
\begin{align}
  \label{eqn:avg_ZD}
  \EZDATWO
  &=
  \int d\mu(\AMAT)\int  d\mu(\WMAT)
      \exp\Bigl[
         \tfrac{\ND\beta}{2}\Trace{\WMAT^{\TR}\AMATN\,\WMAT}
        -\tfrac{\ND\beta}{2}\Trace{\WMAT\DMAT\WMAT^{\TR}}
      \Bigr].
\end{align}
The average over $\AMATN$ affects only the first exponential; applying the SPA, we
\textbf{define} a matrix function $\GFANCY$, depending solely on
$\XMAT=\frac1N\WMAT^{\TR}\WMAT$, by
\begin{align}
  \label{eqn:def_GFANCY}
  \int d\mu(\AMAT)\,
        \exp\Bigl[\tfrac{\ND\beta}{2}\Trace{\WMAT^{\TR}\AMATN\,\WMAT}\Bigr]
  &=
  \exp\Bigl[\tfrac{\ND\beta N}{2}\Trace{\GFANCY}\Bigr].
\end{align}
which will be valid in the large-$N$ approximation below.

We now note that given the duality of measures, we can assert 
\begin{equation}
\EZDA=\EZDAONE=\EZDATWO.
\end{equation}
This lets us insert \eqref{eqn:def_GFANCY} into \eqref{eqn:avg_ZD} and then write
\begin{align}
  \label{eqn:ZD_after_Aavg}
  \EZDA
  &=
  \int d\mu(\WMAT)
      \exp\Bigl[
         \tfrac{\ND\beta N}{2}\Trace{\GFANCY}
        -\tfrac{\ND\beta}{2}\Trace{\WMAT\DMAT\WMAT^{\TR}}
      \Bigr].
\end{align}

The now need to determine an explicit form for
$\GFANCY$. We introduce a new change of measure, 
 $d\mu(\mathbf{W})\rightarrow d\mu(\mathbf{X})$.
Then, we show this lets us express $\EZDA$ as $\EZDX$ and to express it using the LDP.
Next, we apply a SPA to solve for $\GMAX=max\;\GNORM$.
Importantly, we also show how to incorporate the inverse-Temperature $\ND\beta$,
which is new.

\paragraph{Step 2.1}
To define the transformation $d\mu(\mathbf{W})\rightarrow d\mu(\mathbf{X})$,  where (recall) $\mathbf{X}=\frac{1}{N}\mathbf{W}^{\top}\mathbf{W}$,
we use the (again) the integral representation of the Dirac delta-function $\delta(x)$:
\begin{equation}
  \label{eqn:dirac}
  \delta(x):=\frac{1}{2\pi}\int_{-\infty}^{\infty} e^{i\hat{x}x} d\hat{x}.
\end{equation}
%
This lets us express the transformation of measure $d\mu(\mathbf{W})\rightarrow d\mu(\mathbf{X})$
(approximately) as
\begin{align}
\nonumber
  d\mu(\mathbf{W}) &:= \delta(\frac{1}{2}\Trace{N\mathbf{X}-\mathbf{W}^{\top}\mathbf{W}}) d\mu(\mathbf{X}) \\ 
  &= \frac{1}{2\pi}\int_{-\infty}^{\infty} e^{i\frac{1}{2}
    \Trace{\hat{X}(N\mathbf{X}-\mathbf{W}^{\top}\mathbf{W})}
  }
  d\mu(\mathbf{X})d\mu(\hat{X} ),
\end{align}
where $\hat{X}$ is a scalar (or really a matrix of scalars),
and we have a $1/2$ term for mathematical consistency below.
\footnote{The full change of measure would require a delta function constraint
for each matrix element $X_{i,j}$, i.e., 
$\delta\left(\frac{1}{2}N\left(X_{i,j}-[\mathbf{W}^{\top}\mathbf{W}]_{i,j}\right)\right)$.
Here, we assume the Trace constraint is sufficient for our level of rigor.
}



\paragraph{Step 2.2}
Next, we take a Wick Rotation%
\footnote{The Wick rotation converts an oscillatory integral into an exponentially decaying one which should be well defined.  Technically, this is an analytic continuation which needs to be checked, but following standard practice in physics we will assume the resulting integral is analytic and therefore well defined and we will proceed onward. },
$i\XHAT\rightarrow -\XHAT$, so that the terms under the integral are all real (not complex), giving:
\begin{align}
  \label{eqn:dmuX}
  d\mu(\mathbf{W}) &= \mathcal{N}_{Wick}\int_{-i\infty}^{i\infty} e^{\frac{1}{2}\Trace{\hat{X}(N\mathbf{X}-\mathbf{W}^{\top}\mathbf{W})}} d\mu(\mathbf{X})d\mu(\hat{X} ).
\end{align}
where $d\mu(\hat{X})$ is a measure over $\tfrac{M(M-1)}{2}$ Lagrange multipliers, and the normalization is $\mathcal{N}_{Wick}=(\frac{1}{2\pi i})^{\tfrac{M(M-1)}{2}}$.

\paragraph{Step 2.3}
We now insert \ref{eqn:dmuX} into~\ref{eqn:ZD_after_Aavg}, which lets us
express $\EZDA$ as an integral over the \Teacher Correlation matrices.

\begin{align}
  \nonumber
  \EZDA&=
  \mathcal{N}_{Wick} \int_{\XMAT}  \int_{-i\infty}^{i\infty}
  e^{N\frac{\ND\beta}{2} \Trace{\GFANCY}+\frac{N}{2}\Trace{\hat{X}\mathbf{X}} }
  e^{-\frac{1}{2}\Trace{\hat{X}\mathbf{W}^{\top}\mathbf{W}}}
  e^{\frac{\ND\beta}{2}\Trace{\mathbf{W}\mathbf{D}\mathbf{W}^{\top}}}
  d\mu(\hat{X} )
  d\mu(\mathbf{\XMAT}) \\ 
  \nonumber
  &=
  \mathcal{N}_{Wick} \int_{\XMAT}  \int_{-i\infty}^{i\infty}
  e^{N\frac{\ND\beta}{2}\Trace{\GFANCY}+ \frac{N}{2}\Trace{\hat{X}\mathbf{X}}}
  e^{-\frac{1}{2}\Trace{\mathbf{W}\hat{X}\mathbf{W}^{\top}}+
  \frac{\ND\beta}{2}\Trace{\mathbf{W}\mathbf{D}\mathbf{W}^{\top}}}
  d\mu(\hat{X} )
  d\mu(\mathbf{\XMAT}) \\ 
  \label{eqn:ZD3}
    &=
  \mathcal{N}_{Wick} \int_{\XMAT}  \int_{-i\infty}^{i\infty}
  e^{N\frac{\ND\beta}{2}\Trace{\GFANCY}+
  \frac{N}{2}\Trace{\hat{X}\mathbf{X}}}
  e^{\frac{1}{2}\Trace{\mathbf{W}(\ND\beta\mathbf{D}-\hat{X})\mathbf{W}^{\top}}}
  d\mu(\hat{X} )
  d\mu(\mathbf{\XMAT})  .
\end{align}

\paragraph{Step 2.4}
We can now rearrange terms to make this expression look like the \EQN~\ref{eqn:LDP}
%Let us write
%express \EQN~\ref{eqn:ZD3} as the expected value $\EZDA$ in terms of 
%\michael{Huh?}
%\charles{SOME EXPLANATIONS NEEDED: }
%\begin{equation}
%  \EZDA=\int\exp\left(\ND\beta N\Trace{\GNORM}-NI(\mathbf{X})+o(N)\right)d\mu(\mathbf{X})
%\end{equation}


In Large Deviations Theory, the \RateFunction is defined by the Legendre Transform,
\begin{equation}
\label{eqn:rate-fun}
    \mathcal{I}(\mathbf{X})=\underset{\mathbf{\check{X}}}{\sup}
    \left[Tr\dfrac{1}{2}\mathbf{{X}}^{\top}\mathbf{\check{X}}-\ln\mathbb{M}(\mathbf{\check{X}})\right]  ,
\end{equation}
where $\mathbb{M}(\mathbf{\check{X}})$ is the \MomentGeneratingFunction,
$\ln\mathbb{M}(\mathbf{\check{X}})$, is the \CumulantGeneratingFunction, and
and $\mathbf{\check{X}}$ is a (matrix of) \emph{Lagrange Multiplier}(s).  $\mathbb{M}(\mathbf{\check{X}})$ is defined in terms of the (unnormalized) density $p(\mathbf{x})$ as
\begin{equation}
\label{eqn:rate_function}
   \mathbb{M}(\mathbf{\check{X}})=\exp\left(\frac{1}{2}\mathbf{x}^{\top}\mathbf{\check{X}}\mathbf{x}\right)  ,
  p(\mathbf{x})d\mathbf{x}
\end{equation}
which, in turn, is defined in terms of the source matrix $\mathbf{D}$,
\begin{equation}
\label{eqn:X_density}
p(\mathbf{x})=\exp\left(-\tfrac{1}{2}\mathbf{x}^{\top}\ND\beta\mathbf{D}\mathbf{x}\right)  .
\end{equation}
%
The moment generating function $ \mathbb{M}(\mathbf{\check{X}})$ is then given by
%
\begin{equation}
\mathbb{M}(\mathbf{\check{X}}) 
   = \int \exp\left(-\frac{1}{2}\mathbf{x}^T(\ND\beta\mathbf{D} - \mathbf{\check{X}})\mathbf{x}\right) d\mathbf{x} 
   = (2\pi)^{\frac{M}{2}} \Det{\ND\beta\mathbf{D} - \mathbf{\check{X}}}^{-\frac{1}{2}}  .
\end{equation}

\paragraph{Step 2.5}
The \SaddlePointApproximation (SPA) can be used to solve for $\mathcal{I}(\mathbf{\check{X}})$ 
by solving for the stationary conditions
\begin{equation}
  \frac{\partial}{\partial \mathbf{\check{X}}} I(\mathbf{X},\mathbf{\check{X}}) = 0  .
\end{equation}
%
First, let us compute $ \ln \mathbb{M}(\mathbf{\check{X}}) $ as:
%
\begin{equation}
\ln \mathbb{M}(\mathbf{\check{X}}) = \frac{M}{2} \ln(2\pi) - \frac{1}{2} \ln \Det{\ND\beta\mathbf{D} - \mathbf{\check{X}}}.
\end{equation}
\michael{Make det notation consistent throughout.}
\charles{OK lets double check that everywhere;}
%
Substituting this into the expression for the Legendre transform, we obtain:
%
\begin{equation}
I(\mathbf{X},\mathbf{\check{X}}) 
   = \sup_{\mathbf{\check{X}}} \left[\frac{1}{2} \Trace{\mathbf{X} \mathbf{\check{X}}} - \frac{M}{2} \ln(2\pi) + \frac{1}{2} \ln \Det{\ND\beta\mathbf{D} - \mathbf{\check{X}}} \right].
\end{equation}
%
The supremum of this expression is attained at the value of $\mathbf{\check{X}}$ that satisfies:
%
\begin{equation}
\frac{\partial}{\partial \mathbf{\check{X}}} \left[\frac{1}{2} \Trace{\mathbf{X} \mathbf{\check{X}}} + \frac{1}{2} \ln \Det{\ND\beta\mathbf{D} - \mathbf{\check{X}}} \right] = 0.
\end{equation}
%
Taking the derivative, we obtain
%
\begin{equation}
\frac{1}{2} \mathbf{X} + \frac{1}{2} (\ND\beta\mathbf{D} - \mathbf{\check{X}})^{-1} = 0,
\end{equation}
%
which simplifies to:
%
\begin{equation}
\mathbf{X} = (\ND\beta\mathbf{D} - \mathbf{\check{X}})^{-1} \quad \Rightarrow \quad \mathbf{\check{X}} = \ND\beta\mathbf{D} - \mathbf{X}^{-1}.
\end{equation}
%
Substituting $ \mathbf{\check{X}} = \ND\beta\mathbf{D} - \mathbf{X}^{-1} $ back into the expression for $ I(\mathbf{X})$, we obtain:
%
\begin{equation}
I(\mathbf{X}) = \frac{1}{2} \left[\Trace{\mathbf{X} (\ND\beta\mathbf{D} - \mathbf{X}^{-1})} - \frac{M}{2} \ln(2\pi) + \frac{1}{2} \ln \Det{\mathbf{X}^{-1}}\right].
\end{equation}
%
%
\begin{equation}
\Trace{\mathbf{X}\ND\beta\mathbf{D} - \mathbf{I}} = \Trace{\mathbf{X}\ND\beta\mathbf{D}} - N,
\end{equation}
\begin{equation}
\ln \Det{\mathbf{X}^{-1}} = -\ln \Det{\mathbf{X}},
\end{equation}
we get:
%
\begin{equation}
I(\mathbf{X}) = \frac{1}{2} \left[\Trace{\mathbf{X}\ND\beta\mathbf{D}} - \ln \Det{\mathbf{X}} - M - M\ln(2\pi)\right].
\end{equation}

Finally, we express $I(\mathbf{X})$ in the form:

\begin{equation}
I(\mathbf{X}) = \frac{1}{2} \left[ -M(1 + \ln(2\pi)) + \Trace{\mathbf{X}\ND\beta\mathbf{D}} - \ln \Det{\mathbf{X}} \right].
\end{equation}

\paragraph{Step 2.6}
\begin{equation}
  \ND\beta\GFANCY=\mathbb{M}(1+\ln 2\pi)+\ND\beta\Trace{\GNORM(\mathbf{X})} - \Trace{\mathbf{X}\ND\beta\mathbf{D}} +  \ln \Det{\mathbf{X}}.
\end{equation}

We restrict our solution to those where $\XMAT$ and $\ND\beta\mathbf{D}$ can be diagonalized simultaneously.
In particular, this lets us write
\begin{equation}
\Trace{\mathbf{X}\ND\beta\mathbf{D}} = \sum_{\mu=1}^{M}\ND\beta\delta_{\mu}\lambda_{\mu}   ,
\end{equation}
where $\ND\beta\delta_{\mu}$ and $\lambda_{\mu}$ denote the eigenvalues of $\XMAT$ and $\ND\beta\mathbf{D}$, resp.

We can now write the maximum value of $\GNORM$, $\GMAX$, as
\begin{equation}
\label{eqn:gmax_final}
\ND\beta\GMAX = M \left( 1 + \ln \frac{2\pi}{\ND\beta} \right) - \sum_{\mu=1}^{M} \min_{\ND\beta\delta_{\mu}} \left[\ND\beta\delta_{\mu}\lambda_{\mu}
- \ND\beta\GNORM(\lambda_{\mu}) + \ln \lambda_{\mu} \right]   .
\end{equation}

\subsubsection{Expressing the~\GEN~$(\GNORM(\lambda))$ as the Integrated~\RTransform~$(R(z))$ of the~\CorrelationMatrix~$(\AMAT)$}
\label{sxn:tanaka_end}
Having completed both steps, let us combine Eqns.~\ref{eqn:ZD_step1},~\ref{eqn:ln-Z-Nlim2}
with~\ref{eqn:ZD_step2} and~\ref{eqn:gmax_final}.
We follow the first arguments by Tanaka\cite{Tanaka2007} (which follows Cherrier\cite{Cherrier2003}).
\begin{align}
   M\ln(\dfrac{{2}\pi}{\beta})-\sum_{\mu=1}^{M}\int \ln(\beta\delta_{\mu}-\lambda)\rho^{\infty}_{\AMAT}(\lambda)d\lambda
      = M \left( 1 + \ln \frac{2\pi}{\beta} \right) - \sum_{\mu=1}^{M} \min_{\beta\delta_{\mu}} \left[\beta\delta_{\mu}\lambda_{\mu}
      - \beta\GNORM(\lambda_{\mu}) + \ln \lambda_{\mu} \right]   .
\end{align}
By cancelling the $\ln \frac{2\pi}{\beta}$ term from both sides, we obtain
\begin{align}
   -\sum_{\mu=1}^{M}\int \ln(\beta\delta_{\mu}-\lambda)\rho^{\infty}_{\AMAT}(\lambda)d\lambda
   =
   M - \sum_{\mu=1}^{M} \min_{\beta\delta_{\mu}} \left[\beta\delta_{\mu}\lambda_{\mu}
   - \beta\GNORM(\lambda_{\mu}) + \ln \lambda_{\mu} \right]   .
\end{align}
Since this is true for every $\mu$, we can solve this for any arbitrary eigenvalue $\lambda_{\mu}$.
\michael{There must be some other assumption that the $M$ term is spread out uniformly over eigenvalues?}
\charles{Yes. We assumed this earlier in A62.  Why dont you add some comments here}

Dropping the $\mu$ subscript, we have the following identity:
\begin{align}
\label{eqn:concave_id} 
 \min_{\delta} \left[\beta\delta\lambda - \beta\GNORM(\lambda) + \ln \lambda \right]
 = 1 -\int \ln(\beta\delta-\lambda)\rho^{\infty}_{\AMAT}(\lambda)d\lambda   .
\end{align}

We need to invert \ref{eqn:concave_id} in order to find $\beta\GNORM(\lambda)$.
If we choose the eigenvalues of $\mathbf{D}$ such that $\beta\delta_{\mu}>\lambda_{max}$ for all $\mu$, then 
this relation is concave and therefore invertible via a Legendre transform.  

This gives
\begin{equation}
\beta\GNORM(\lambda) = \beta\delta(\lambda) \lambda - \int \ln[\beta\delta(\lambda) - \lambda] \rho^{\infty}_{\AMAT}(\lambda) d\lambda - \ln \lambda - 1  ,
\end{equation}
where we need to  define  $ \beta\delta(\lambda)$, which (not to be confused with the Dirac delta-function), describes
the functional dependence between the eigenvalues of the source matrix $\DMAT$ and the \Student \CorrelationMatrix $\AMAT$.

$\GNORM(\lambda)$ is computed by minimizing over $\delta$, ensuring the relationship holds for the entire spectrum.
So let us take the derivative of $\beta\GNORM$ w/r.t $\lambda$. 
Term by term, this gives:
\begin{equation}
\dfrac{d}{d\lambda} \beta\delta(\lambda)\lambda = \beta\delta(\lambda) + \dfrac{d \beta\delta(\lambda)}{d\lambda} \lambda
\end{equation}
\begin{equation}
\dfrac{d}{d\lambda} \ln\lambda = \dfrac{1}{\lambda}
\end{equation}
\begin{align}
\dfrac{d}{d\lambda} \int \ln[\beta\delta(\lambda) - \lambda] \rho^{\infty}_{\AMAT}(\lambda) d\lambda
&= \int \dfrac{d}{d\lambda} \ln[\beta\delta(\lambda) - \lambda] \rho^{\infty}_{\AMAT}(\lambda) d\lambda \\ \nonumber
&= \int \dfrac{d \beta\delta(\lambda)}{d\lambda}\dfrac{ \rho^{\infty}_{\AMAT}(\lambda)}{\beta\delta(\lambda) - \lambda}d\lambda  \\ \nonumber
&=  \dfrac{d \beta\delta(\lambda)}{d\lambda}\int\dfrac{ \rho^{\infty}_{\AMAT}(\lambda)}{\beta\delta(\lambda) - \lambda}d\lambda
\end{align}

We can now simplify by  defining $\delta(\lambda)$ implicitly by the integral relation
\begin{equation}
\lambda = \int \frac{\rho^{\infty}_{\AMAT}(\lambda)}{\beta\delta(\lambda) - \lambda} d\lambda.
\end{equation}
Combining terms, this gives
\begin{equation}
\frac{d\beta\GNORM(\lambda)}{d\lambda} = \beta\delta(\lambda) - \frac{1}{\lambda},
\end{equation}

 Inverting the derivative, we obtain an integral equation for $\beta\GNORM(\lambda)$ 
\begin{equation}
\beta\GNORM(\lambda) = \int_0^\lambda \left(\beta\delta(z) - \frac{1}{z}\right) dz.
\end{equation}

Notice since  $\beta\delta(\lambda) \approx \frac{1}{\lambda}$ for $\lambda \ll 1$, then
as $\GNORM(0) = 0$ and we set the lower integrand to $0$ (for now).  Even though Tanaka’s original proof assumes an analytic continuation without branch cuts, a heavy-tailed spectrum merely shifts the lower limit of the $R$-transform integral, so the expression for $\beta\mathcal G(\lambda)$ continues to hold.

To further connect these to the \RTransform $R_{\AMAT}(z)$, we recall that the \CauchyStieltjes (or just \Cauchy See~\ref{eqn:Cz}) transform $\mathcal{C}_{\AMAT}(z)$  is given by:

\begin{equation}
\mathcal{C}_{\AMAT}(z) = \int \frac{\rho_{\AMAT}(\lambda)}{z - \lambda} d\lambda.
\end{equation}

The relationship between the \Cauchy transform and the \RTransform is  then expressed as:

\begin{equation}
\mathcal{C}_{\AMAT}\left(R_{\AMAT}(z) + \frac{1}{z}\right) = z,
\end{equation}

which implies:

\begin{equation}
\beta\GNORM(\lambda) = \int_0^{\lambda} R_{\AMAT}(z) dz.
\end{equation}

WLOG, as mentioned earlier, we can replace the lower bound on $\lambda$ from
$0\rightarrow\LambdaECSmin$ to obtain
\begin{equation}
\beta\GNORM(\lambda) = \int_{\LambdaECSmin}^{\lambda} \Re[R_{\AECS}(z]) dz.
\end{equation}
where $\LambdaECSmin$ corresponds to the start of the \EffectiveCorrelationSpace (\ECS),
and, notably,  take the \emph{Real} part of $R(z)$.
We take the Real part because the imaginary parts coming from the upper and lower lips of
the cut cancel.\footnote{
Alternatively, one might try to replace $R(z)$ by its modulus;
doing so breaks the Legendre relation $G' = R$ and spoils the
additivity of free cumulants.}










\subsubsection{Selecting \texorpdfstring{$\AMAT:=\AMAT_M$}{A:=A M} instead of 
\texorpdfstring{$\AMAT_N$}{A N}}
\label{sxn:tanaka_end}
In principle, we could have selected $\AMAT:=\AMAT_M=\tfrac{1}{N}\SMAT^{\top}\SMAT$  for the \Student Correlation matrix,
thereby avoiding the discussion on the \DualityOfMeasures altogether.
Doing this, however, would make $\AMAT$ $M\times M$, thereby
require defining the \SourceMatrix $\DMAT$ as an
$N \times N$ matrix, with presumably $N-M$ zero eigenvalues.
This would cause $\DMAT$ to violate the condition $\DeltaMu > \beta\lambda$
for all eigenvalues $\lambda$ of $\AMAT_M$.
In this case, it would be challenging to define the large-$N$ limit.


\subsection{The Inverse-MP (IMP) Model}
\label{sxn:IMP}
In this section, we rederive the integral $G(\lambda)[IMP]$ for the \InverseMP (IMP) model, focusing on the branch cut starting at $z = \kappa/2$ and extending to infinity. 
\michael{We derive, not rederive, right?  Section \ref{sxn:r_transforms} just points here.  }
This branch cut corresponds to the support of the ESD in this region. 
We will:
\begin{enumerate}
\item Explain the presence of the branch cut and its implications.
\item Show that $R(z)[IMP]$ becomes complex along this branch cut because the term under the square root becomes negative.
\item Perform the integral $G(\lambda)[IMP]$, showing all steps.
\item Compute the modulus $|G(\lambda)[IMP]| = \sqrt{ G(\lambda)[IMP]^* G(\lambda)[IMP] }$ to obtain a real-valued estimate, analogous to a probability estimate.
\end{enumerate}
\nred{Note: Below, we may want the Real part of $G(\lambda)[IMP]$ and not the modulus. Stil thinking on this.
  Also, we may want to add additional parameters to the IMP model to account for the dimension of the matrix.
This changes things  bit.}

\subsubsection{The Branch Cut in the IMP Model}

The R-transform for the IMP model is given by:
\begin{align}
\label{eqn:iw_r_transf}
R(z)[IMP] = \frac{\kappa - \sqrt{\kappa(\kappa - 2z)}}{z},
\end{align}
where $\kappa > 0$ is a parameter related to the dimensions of the random matrices under consideration.
\michael{Ref.}
The function $\sqrt{\kappa(\kappa - 2z)}$ introduces a branch point at $z = \kappa/2$ because the argument of the square root becomes zero at this point:
\begin{align}
\kappa - 2z = 0 \quad \Rightarrow \quad z = \frac{\kappa}{2}.
\end{align}
For $z > \kappa/2$, the argument $\kappa - 2z$ becomes negative, and thus the square root becomes imaginary. 
This leads to a branch cut starting at $z = \kappa/2$ and extending to $z = \infty$ along the real axis. 
This branch cut affects the analyticity of $R(z)[IMP]$, and it must be carefully considered in the integral $G(\lambda)[IMP]$.

\subsubsection{$R(z)[IMP]$ is Complex Along the Branch Cut}

For $z > \kappa/2$, we have:
\begin{align}
\kappa - 2z < 0 \quad \Rightarrow \quad \sqrt{\kappa(\kappa - 2z)} = \sqrt{-\kappa(2z - \kappa)} = i \sqrt{\kappa(2z - \kappa)}.
\end{align}
%
Therefore, $R(z)[IMP]$ becomes complex:
\begin{align}
R(z)[IMP] = \frac{\kappa - i \sqrt{\kappa(2z - \kappa)}}{z} = \frac{\kappa}{z} - i \frac{ \sqrt{ \kappa(2z - \kappa) } }{ z }.
\end{align}
%
This expression shows that $R(z)[IMP]$ has both real and imaginary parts when $z > \kappa/2$.

\subsubsection{Calculation of $G(\lambda)[IMP]$}

We aim to compute the integral:
\begin{align}
G(\lambda)[IMP] = \int_{z_0}^\lambda R(z)[IMP] , dz,
\end{align}
where $z_0 \geq \kappa/2$.


\paragraph{Integrating the Real Part.}

First, we consider the real part of $R(z)[IMP]$:
\begin{align}
\text{Re}[R(z)[IMP]] = \frac{\kappa}{z}.
\end{align}
%
The integral of this real part is:
\begin{align}
G_{\text{real}}(\lambda)[IMP] = \int_{z_0}^\lambda \frac{\kappa}{z} , dz = \kappa \left[ \ln z \right]_{z_0}^\lambda = \kappa \left( \ln \lambda - \ln z_0 \right).
\end{align}


\paragraph{Integrating the Imaginary Part.}

Next, consider the imaginary part:
\begin{align}
\text{Im}[R(z)[IMP]] = - \frac{ \sqrt{ \kappa(2z - \kappa) } }{ z }.
\end{align}
%
If we let $u = 2z - \kappa$, then:
\begin{align}
z = \frac{u + \kappa}{2}, \quad dz = \frac{du}{2}.
\end{align}
%
Substituting this into the imaginary part:
\begin{align}
\text{Im}[R(z)[IMP]] = - \frac{ \sqrt{ \kappa u } }{ \frac{u + \kappa}{2} } = - \frac{ 2 \sqrt{ \kappa u } }{ u + \kappa } ,
\end{align}
the integral becomes:
\begin{align}
G_{\text{imag}}(\lambda)[IMP] 
   = \int_{u_0}^{u_\lambda} - \frac{ 2 \sqrt{ \kappa u } }{ u + \kappa } \cdot \frac{du}{2}  
   = - \int_{u_0}^{u_\lambda} \frac{ \sqrt{ \kappa u } }{ u + \kappa } du  ,
\end{align}
where $u_0 = 2 z_0 - \kappa$ and $u_\lambda = 2 \lambda - \kappa$.
%
If we simplify the integrand:
\begin{align}
\sqrt{ \kappa u } = \sqrt{ \kappa } \sqrt{ u }, 
\end{align}
%
then the integral becomes:
\begin{align}
G_{\text{imag}}(\lambda)[IMP] = - \sqrt{ \kappa } \int_{u_0}^{u_\lambda} \frac{ \sqrt{ u } }{ u + \kappa } , du.
\end{align}
%
This integral can be evaluated using standard integral formulas. 
We will compute it step by step.


\paragraph{Evaluating the Integral.}

Consider the integral:
\begin{align}
I = \int \frac{ \sqrt{ u } }{ u + \kappa } , du.
\end{align}
%
We can use the following integral formula:
\begin{align}
\int \frac{ \sqrt{ u } }{ u + a } , du = 2 \sqrt{ u } - 2 a \tan^{-1} \left( \frac{ \sqrt{ u } }{ \sqrt{ a } } \right ) + C,
\end{align}
where $a > 0$ and $u > 0$.
%
Applying this formula, we get:
\begin{align}
I = 2 \sqrt{ u } - 2 \kappa \tan^{-1} \left( \frac{ \sqrt{ u } }{ \sqrt{ \kappa } } \right ) + C.
\end{align}
%
Therefore, the imaginary part of $G(\lambda)[IMP]$ is:
\begin{align}
\nonumber
G_{\text{imag}}(\lambda)[IMP] 
   &= - \sqrt{ \kappa } \left[ 2 \sqrt{ u } - 2 \kappa \tan^{-1} \left( \frac{ \sqrt{ u } }{ \sqrt{ \kappa } } \right ) \right ]{u_0}^{u\lambda} \\
\nonumber
   &= - \sqrt{ \kappa } \left( \left[ 2 \sqrt{ u_\lambda } - 2 \kappa \tan^{-1} \left( \frac{ \sqrt{ u_\lambda } }{ \sqrt{ \kappa } } \right ) \right ] - \left[ 2 \sqrt{ u_0 } - 2 \kappa \tan^{-1} \left( \frac{ \sqrt{ u_0 } }{ \sqrt{ \kappa } } \right ) \right ] \right ) \\
   &= - 2 \sqrt{ \kappa } \left( \sqrt{ u_\lambda } - \sqrt{ u_0 } \right ) + 2 \kappa^{ 3/2 } \left( \tan^{-1} \left( \frac{ \sqrt{ u_\lambda } }{ \sqrt{ \kappa } } \right ) - \tan^{-1} \left( \frac{ \sqrt{ u_0 } }{ \sqrt{ \kappa } } \right ) \right ).
\end{align}


\paragraph{Combining Real and Imaginary Parts.}

Combine the real and imaginary parts to obtain $G(\lambda)[IMP]$:
\begin{align}
G(\lambda)[IMP] = G_{\text{real}}(\lambda)[IMP] + i G_{\text{imag}}(\lambda)[IMP].
\end{align}
%
Substituting the expressions:
\begin{align}
\nonumber
G(\lambda)[IMP] 
   &= \kappa \left( \ln \lambda - \ln z_0 \right )   \\
   &\quad + i \left( - 2 \sqrt{ \kappa } \left( \sqrt{ u_\lambda } - \sqrt{ u_0 } \right ) + 2 \kappa^{ 3/2 } \left( \tan^{-1} \left( \frac{ \sqrt{ u_\lambda } }{ \sqrt{ \kappa } } \right ) - \tan^{-1} \left( \frac{ \sqrt{ u_0 } }{ \sqrt{ \kappa } } \right ) \right ) \right ).
\end{align}
%
Recall that $u = 2z - \kappa$, so:
\begin{align}
\sqrt{ u } = \sqrt{ 2z - \kappa }.
\end{align}
%
Therefore, we can write $G(\lambda)[IMP]$ as:
\begin{align}
\nonumber
G(\lambda)[IMP] 
   &= \kappa \ln \left( \frac{ \lambda }{ z_0 } \right ) - 2 i \sqrt{ \kappa } \left( \sqrt{ 2 \lambda - \kappa } - \sqrt{ 2 z_0 - \kappa } \right ) \\
   &\quad + 2 i \kappa^{ 3/2 } \left( \tan^{-1} \left( \frac{ \sqrt{ 2 \lambda - \kappa } }{ \sqrt{ \kappa } } \right ) - \tan^{-1} \left( \frac{ \sqrt{ 2 z_0 - \kappa } }{ \sqrt{ \kappa } } \right ) \right ).
\end{align}


\subsubsection{Computing the Modulus $|G(\lambda)[IMP]|$}

To obtain a real-valued estimate, we compute the modulus of $G(\lambda)[IMP]$:
\begin{align}
| G(\lambda)[IMP] | = \sqrt{ \left( \text{Re}[ G(\lambda)[IMP] ] \right )^2 + \left( \text{Im}[ G(\lambda)[IMP] ] \right )^2 }.
\end{align}


\paragraph{Calculating the Real Part Square.}

The real part is:
\begin{align}
\text{Re}[ G(\lambda)[IMP] ] = \kappa \ln \left( \frac{ \lambda }{ z_0 } \right ).
\end{align}
%
Therefore,
\begin{align}
\left( \text{Re}[ G(\lambda)[IMP] ] \right )^2 = \kappa^2 \left( \ln \left( \frac{ \lambda }{ z_0 } \right ) \right )^2.
\end{align}


\paragraph{Calculating the Imaginary Part Square.}

The imaginary part is:
\begin{align}
\text{Im}[ G(\lambda)[IMP] ] &= - 2 \sqrt{ \kappa } \left( \sqrt{ 2 \lambda - \kappa } - \sqrt{ 2 z_0 - \kappa } \right ) \
&\quad + 2 \kappa^{ 3/2 } \left( \tan^{-1} \left( \frac{ \sqrt{ 2 \lambda - \kappa } }{ \sqrt{ \kappa } } \right ) - \tan^{-1} \left( \frac{ \sqrt{ 2 z_0 - \kappa } }{ \sqrt{ \kappa } } \right ) \right ).
\end{align}
%
Let’s denote:
\begin{align}
A &= - 2 \sqrt{ \kappa } \left( \sqrt{ 2 \lambda - \kappa } - \sqrt{ 2 z_0 - \kappa } \right ), \
B &= 2 \kappa^{ 3/2 } \left( \tan^{-1} \left( \frac{ \sqrt{ 2 \lambda - \kappa } }{ \sqrt{ \kappa } } \right ) - \tan^{-1} \left( \frac{ \sqrt{ 2 z_0 - \kappa } }{ \sqrt{ \kappa } } \right ) \right ).
\end{align}
%
Then,
\begin{align}
\left( \text{Im}[ G(\lambda)[IMP] ] \right )^2 = (A + B)^2 = A^2 + 2AB + B^2.
\end{align}


\paragraph{Computing the Modulus.}

The modulus is:
\begin{align}
| G(\lambda)[IMP] | = \sqrt{ \left( \kappa \ln \left( \frac{ \lambda }{ z_0 } \right ) \right )^2 + \left( A + B \right )^2 }.
\end{align}


\paragraph{Interpretation.}

While the expression for $| G(\lambda)[IMP] |$ appears complex, it encapsulates the cumulative effect of both the real and imaginary components of $G(\lambda)[IMP]$. This modulus provides a real-valued estimate that is meaningful in the context of probability estimates.


\subsubsection{Summary}

By integrating $R(z)[IMP]$ directly, including its complex components, we have obtained an explicit expression for $G(\lambda)[IMP]$ as a complex function. 
Computing the modulus $| G(\lambda)[IMP] |$ gives us a real-valued function that accurately captures the contribution of the tail of the ESD in the IMP model.
%
This approach accounts for the complex nature of $R(z)[IMP]$ along the branch cut $z > \kappa/2$, and it provides a meaningful point estimate for further analysis.
\michaeladdressed{By that, I assume we mean that we will use it in our derivation.}








\end{document}
