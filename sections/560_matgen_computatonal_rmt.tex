\subsection{Computational Random Matrix Analysis}
\label{sxn:comp_rmt}
\nred{The presence of the branch cut at the start of the ECS complicates this section.
  We may need to remove it entirely as it is probably deeply wrong
  and save it for a future work.  Not super happy about that but time is limitted.
  I will think some more on it
  }
The \RTransform is the generating function for the \emph{\FreeCumulants} of RMT.  Formally, one can define $R(z)$ as a series expansion in $z$,
\begin{equation}
  \label{eqn:Rz_expansion}
  R(z) := \kappa_1 + \kappa_2 z + \kappa_3 z^2 + \ldots 
\end{equation}
where the coefficients $\kappa_{k}$ are the free cumulants, which can be expressed
in terms of the matrix moments $m_{k}$\cite{FreeCumulants}, defined (here) as
\begin{equation}
  \label{eqn:mk_defn}
  m_{k}:=\Trace{\XECS^{k}}=\sum_{i=1}^{\MECS}(\LambdaECS)^{k}
\end{equation}
where $\LambdaECS_{k}$ is the k-th eigenvalue of the effective correlation matrix $\mathbf\XECS$,
which  has been mean-centered and normalized by its standard deviation.

The free cumulants are defined recursively as
\begin{equation}
  \label{eqn:kappa_defn}
  \kappa_k := m_k - \sum_{\text{partitions of } n} \prod_{\text{blocks } B} m_{|B|} 
\end{equation}
\charles{finish explanation}

The first 5 \emph{\Cumulants} are, explicitly,
\begin{align}
  \label{eqn:kappa_defn_2}
  k_1 = m_1 \\ \nonumber
  k_2 = m_2 - m_1^2 \\ \nonumber
  k_3 = m_3 - 3 m_2 m_1 + 2 m_1^3 \\ \nonumber
  k_4 = m_4 - 4 m_3 m_1 - 2 m_2^2 + 10 m_2 m_1^2 - 5 m_1^4 \\ \nonumber 
  k_5 = m_5 - 5 m_4 m_1 + 15 m_3 m_1^2 + 15 m_2^2 m_1 - 35 m_2 m_1^3 - 5 m_3 m_2 + 14 m_1^5
\end{align}

Using these definitions, we can estimate the \LayerQuality matrix $\QT$ for our experimental models
(in Section~\ref{sxn:empirical} by computing  $\GN$ for the effective correlation space
(i.e. the tail of the layer ESD), however it is defined.  That is, we use

\begin{align}
  \label{eqn:G_lambda_series}
\GNI=\kappa_{1}\frac{\LambdaECS}{\MECS}+\frac{\kappa_{2}}{2}\left(\dfrac{\LambdaECS}{\MECS}\right)^{2}+\cdots
\end{align}
Note that we evaluate $\GN$ using eigenvalues normalized with $\tfrac{1}{\MECS}$.
\nred{although this may not be obvious the way we have written things so far}
\charles{We may not need to specify the $\tfrac{1}{\MECS}$ term anymore because I think the code now uses
  the properly normalized $\XECS$, but I need to check carefully}
\charles{Mention experimental section in XXX}



