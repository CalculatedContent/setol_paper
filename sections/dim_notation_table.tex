\begin{table}[H]
\centering
\begin{tabular}{l|l|l}
\toprule
 \textbf{Definition} & \textbf{Vector} & \textbf{Matrix} \\
\midrule
 Total number of data samples used in training. & $\ND$ & $\ND$ \\
 Number of features per training sample (Input dimension). & $m$ & $M$ \\
 Dimension of layer output (Output dimension)  & $1$ & $N$ \\
 Number of free parameters (in $R$) & $1$ & $M(M-1)/2$ \\
 Energy scaling & $\ND$ & $\ND\times N\times M$ \\
\bottomrule
\end{tabular}
\caption{In the ST \Perceptron \emph{vector} model, lowercase $m$ is the dimension of the weight vector (total parameters), which is also the number of features per sample. In the vector case, there is one free parameter -- the overlap $R$ (or angle $\theta$) between student and teacher. In the \SETOL \emph{matrix} model,  uppercase $N$ and $M$ are the input and output dimensions of the weight matrix, and the
matrix the overlap $\OVERLAP$, being an $M\times M$ symmetric matrix, has $M(M-1)/2$ free parameters.
The Energy scales as $\ND$ in the ST \Perceptron model, and as $\ND\times N\times M$ in the \SETOL matrix model.
%The total number of degrees of freedom is defined as the number of training examples $\ND$ times the number of free parameters.
}
\label{tab:dim_notation}
\end{table}

