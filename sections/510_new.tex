%\subsection{Basic Elements of the \SemiEmpirical Theory}
%\label{sxn:matgen_basic_elements_concise}

\begin{enumerate}[label=5.\arabic*]
\item
\textbf{Matrix Generalization of the ST Model.}
Section~\ref{sxn:matgen_mlp3} generalizes
classical \STATMECH vector-based ST model of Section~\ref{sxn:SMOG_main-student_teacher}
to obtain a \LayerQuality for a single layer in a NN.
It starts by first formulating the learning problem for
the  NN generalization accuracy or quality, $\Q^{NN}$,
of a  3-layer MLP (MLP3).
We then replace vectors with $N \times M$ matrices $\SVEC,\TVEC\rightarrow\SMAT,\TMAT$,
and obtain and expression for the NN \SelfOverlap $\ETA(\SMAT,\TMAT,\XI)$,
which then gives a matrix-generalized overlap operator
$\OVERLAP:=\langle\ETA(\SMAT,\TMAT,\XI)\rangle_{\AVGNDXI}=\tfrac{1}{N}\SMAT^{\top}\TMAT$.
This can be related to a single-layer matrix-generalization of the ST \AnnealedHamiltonian, 
presented in Appendix~\ref{sxn:summary_sst92}, $\HANHT:=M-\OVERLAP$,
where, importantly, the scalar overlap $R$ is now a matrix $\OVERLAP$ of $M$ adjustable parameters.

\item
\textbf{The \LayerQualitySquared $\QT$}
Section~\ref{sxn:matgen_quality_hciz} presents the expression for NN \LayerQualitySquared $\QT$.
Following the ST analogy, we define a \ThermalAverage over possible \Student weight matrices $\SMAT$
for the matrix overlap, giving $\Q^{NN}_{L}:=\THRMAVGIZ{\HANHT} =\THRMAVGIZ{\OVERLAP}$,
For technical reasons, however, we actually seek the (approximate)  \LayerQualitySquared, $\QT\approx\Q^{2}$,
defined as $\QT:=\THRMAVGIZ{\OLAPTOLAP}$.
To evaluate $\QT$, rather than sampling all random \Student matrices $\SMAT$ directly,
we switch measures to the \Student correlation matrices 
$\AMAT_{2} = \tfrac{1}{N}\SMAT\SMAT^{\top}$.
Importantly, we argue that the measures $d\mu(\AMAT_{1})\leftrightarrow d\mu(\AMAT_{2})$,
can be interchanged for our purposes, making them effectively equivalent.
This reparametrization leads us to an integral of the HCIZ type (as in \EQN~\ref{eqn:hciz_prelim})
which, as shown by Tanaka \cite{Tanaka2007, Tanaka2008}.

Then, we introduce the \EffectiveCorrelationSpace (\ECS), and two key approxmations,
the \TRACELOG Condition and the \IndependentFluctuationApproximation (\IFA).
The \TRACELOG condition states that the determinant of the (effective) \Student correlation matrix is unity, $\Det{\AECS}=1$.
Critcially, this condition can be tested empirically by assuming the (effective) \Teacher correlation matrix
also follows the \TRACELOG condition, $\Det{\XECS}=1$, and this is a key result of this work.
Finally, we impose the \IFA (described below) because it is necessary for the final result.

\item
\textbf{The Large-$N$ limit}
Section~\ref{sxn:matgen_evaluation_hciz} presents the core result,
(as in \EQN~\ref{eqn:QT_result}),
a closed-form or semi-analytic expressions for the \LayerQualitySquared $\QT$
formed in the large-$N$ limit.
Restricted to the \ECS, and under  the \TRACELOG condition and the \IFA, our
HCIZ integral for $\QT$ becomes tractable at large-$N$, givin an expression that can be parameterized
in terms of $\MECS$ eigenvalues $\LambdaECS$ of the \Teacher correlation matrix 
restricted to the \ECS $\XECS$.
In doing this, the $\MECS$ \Teacher eigenvalues are treated as experimental observables, and 
become the effective \SemiEmpirical parameters (i.e, $\alpha$, $\lambda_{max}$) of our \SETOL.

\item
\textbf{Modeling the \HeavyTailed \RTransform}
Section~\ref{sxn:r_transforms} presents several  models of different \RTransforms.
Evaluating $\QT$ requires evaluating selecting an \RTransform $R(z)$ for the Teacher ESD,
and also ensure that it is analytic and single-valued on the domain of interest-- the \ECS and/or tail of the ESD.
We examine three possible modles for $R(z)$: \emph{(i)} the \emph{Spikes-only model},
\emph{(ii)} the \emph{Inverse Wishart} (IW) model, 
and \emph{(iii)} the  \LevyWigner (LW) model.
First, as a trivial case, the tail of ESD can be treated a collection of spikes,
and the ESD is simply as a sum of Dirac delta functions; in this case,
$\QT$ becomes a \red{``Tail Norm'' or ``Trace Norm'' ; check this}
When the layer is \Ideal, i.e., $\alpha\sim 2$ and $\Det{\XECS}\sim 1$,
one can use  \emph{Inverse Wishart} (IW) model. 
As required, the IW \RTransform contains a \emph{branch cut} in the complex plane
which aligns with the start of the \ECS /\PowerLaw  tail.
Fnally, Using the \LevyWigner model, one can (at least formally) derive the \HTSR \ALPHAHAT metric.

\end{enumerate}

\vspace*{1em}

These core elements form a bridge between well-established empirical properties of large-scale NNs 
and a tractable ST-based theory. In the subsequent sections, we formalize the key steps: 
\emph{(i)} setting up the matrix-based ST problem, \emph{(ii)} defining our HCIZ integrals 
over restricted correlation matrices (\ECS), and \emph{(iii)} analyzing the resulting 
\emph{Layer Quality} (or \emph{Quality-Squared}) expressions in the large-$N$ limit.
