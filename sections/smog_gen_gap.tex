\paragraph{Generalization Gap vs. Model Quality}
\label{sxn:SMOG_main-model_quality}

We should distinguish between what is typically done in the \SLT literature versus the \STATMECH approaches.
In \SLT, one is typically interested in modeling the \emph{\GeneralizationGap}.
The \GeneralizationGap quantifies the difference between a models performance on training data versus unseen test data:
\begin{align}
  \label{eqn:gen_gap}
  \mathcal{E}^{emp}_{gap}:= \TTE[\DXtrain]- \TGE[\DXtest]
\end{align}
In contrast, in \STATMECH approaches, one considers the \ModelGeneralizationError directly,
which is sometimes called the \ModelQuality in the \SLT literature.
Model quality is an indication of the models accuracy, precision, recall, or any other relevant metric based on the task at hand.
%Here, we mean that the \ModelGeneralizationError is a measure of the \ModelQuality on such OOD data.

While related, in developing an analytic theory, the \GeneralizationGap and
the \ModelQuality (or \ModelGeneralizationError) require conceptually different approaches.
This is because the \GeneralizationGap depends on a specific realization of the training data,
whereas our \ModelGeneralizationError will be formulated on a random training data set
(and then corrected later with empirical data).
In this sense, any theory of the \GeneralizationGap  requires a formalism where the
predicted model error is \Quenched to the training data, which is not what we want.
In contrast, the \ModelGeneralizationError  will be formulated using the \AnnealedApproximation (AA),
and is therefore both conceptually and mathematically simpler.
Notably, the \HTSR model \Quality metric trend well when compared to the \ModelGeneralizationError ,
whereas methods from the \SLT literature tend to only describe the \GeneralizationGap well.\cite{YTHx23_KDD}
