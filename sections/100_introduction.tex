
\section{Introduction}
\label{sxn:intro}
Deep Neural Networks (DNNs)—models in the field of Artificial Intelligence (AI)—have driven remarkable advances in multiple fields of science and engineering. AlphaFold has made significant progress in solving the protein folding problem.\cite{AlphaFold} Notably, the 2024 Nobel Prize in Physics was awarded to Hopfield and Hinton for developing early approaches to AI using techniques from \emph{Statistical Mechanics} (\STATMECH), and Jumper and Hassabis, along with Baker, received the 2024 Nobel Prize in Chemistry for their contributions to AlphaFold and computational protein design.\cite{Nobel2024Physics, Nobel2024Chemistry} Self-driving cars now roam the streets of major metropolitan cities like San Francisco. Large Language Models (LLMs) like ChatGPT have gained worldwide attention and initiated serious conversations about the possibility of creating an Artificial General Intelligence (AGI). Clearly, not a single area of science or engineering has ignored these remarkable advances in the field of AI and Neural Networks (NNs).

Despite this remarkable progress in a research field spanning over 50 years, developing, training, and maintaining such complex models require staggering capital resources, limiting their development to only the largest and best-funded organizations. While many such entities have open-sourced some of their largest models (such as Llama and Falcon), using these models requires assuming they have been trained optimally, without significant defects that could limit, skew, or even invalidate their use downstream. Moreover, testing such models can be very expensive and complex to interpret.

Because training and evaluating NNs is so hard, significant issues can manifest in many obvious and non-obvious ways. A primary research goal is to improve the efficiency and reduce the cost of training large NNs. A less known but critical issue arises in many industrial settings, specifically “selecting the best models to test.” This arises in industries such as ad-click prediction, search relevance, quantitative trading, and more. Frequently, one has several seemingly equally good models to choose from, but testing the model can be very expensive, time-consuming, and even risky to the business. Recently, researchers and practitioners have started to fine-tune such large open-source models using techniques such as LoRA and QLoRA. Such methods allow one to adapt a large, open-source NN to a small dataset, and very cheaply. However, in fine-tuning, one could unwittingly overfit the model to the small dataset, degrading its performance for its intended use. Despite these and many other problems, theory remains well behind practice, and there is an increasingly pressing need to develop \emph{practical predictive theory} to both improve the training of these very large NN models and to design new methods to make their use more reliable.

Before dicussing these methods, however, let us explain \emph{What is a SemiEmpirical Theory}

\subsection{Statistical Mechanics (\STATMECH) vs. Statistical Learning Theory (\SLT)}

Historically, there have been two competing theoretical frameworks for understanding NNs:
\emph{\StatisticalMechanics} (\STATMECH)~\cite{Eng01, EB01_BOOK, Gardner_1988,SST90, SST92, LTS90, Solla2023}; and 
\emph{\StatisticalLearningTheory} (\SLT)~\cite{Vapnik98}. 
\begin{itemize}
\item
\textbf{\StatisticalMechanics (\STATMECH).}
This framework has been foundational to the early development of NN models, such as the Hopfield Associative Memory (HAM)~\cite{Hop82}, Boltzmann Machines~\cite{AHS85},~\cite{HinSej86_relearn}, etc.
\STATMECH~has also been used to build early theories of learning, such as the \StudentTeacher model for the \Perceptron \GeneralizationError~\cite{Eng01,EB01_BOOK}, the Gardner model~\cite{Gardner_1988}, and many others.
Notably, the HAM was based on an idea by Little, who observed that, in a simple model, long-term memories are stored in the eigenvectors of transfer matrix~\cite{Lit74}.
(This general idea, but in a broader sense, is central to our approach below.)
Moreover, \STATMECH~predicts that NNs exhibit phase behavior.
This has recently been rediscovered as the Double Descent phenomenon~\cite{BHMM19,loog2020}, but it was known in \STATMECH long before it's recent rediscovery~\cite{Opper01}.
However, unlike other applied physics theories (e.g., \SemiEmpirical methods in quantum chemistry), \STATMECH only offers qualitative analogies, failing to provide testable quantitative predictions
about large, modern NN models.\cite{roberts2022principles}
\item
\textbf{\StatisticalLearningTheory (\SLT).}
\SLT and related approaches (VC theory, PAC bounds theory, etc.) have been developed within the context of traditional computational learning problems~\cite{Vapnik98}, and 
they are based on analyzing the convergence of frequencies to probabilities (over problem classes, etc.).
It was recognized early on, however, that they could not be directly applied to NNs~\cite{VLL94}.
Moreover, \SLT cannot even reproduce quantitative properties of learning curves~\cite{WRB93,DKST96} (whereas \STATMECH~is very successful at this~\cite{SST92}).
\SLT also failed to predict the ``Double Descent'' phenomenon~\cite{BHMM19}.
More recently, it has been shown that in practical settings
%, when trying to predict test accuracies,
\SLT can give vacuous~\cite{DR17_nonvacuous_TR} or even opposite results
to those actually observed~\cite{MM21a_simpsons_TR}.
\end{itemize}

\noindent
Technically, \SLT focuses on obtaining bounds on a model’s worst-case behavior, while \STATMECH seeks a probabilistic understanding of typical behaviors across different states or configurations.
Unfortunately, neither of these general theoretical frameworks has proven particularly useful to NN practitioners.
\SETOL combines insights from both.
Rather than being purely phenomenological like the \HTSR~approach, \SETOL is derived from first-principles, and in the form of a \SemiEmpirical~theory.
As such, \SETOL offers a practical, \SemiEmpirical framework that bridges rigorous theoretical modeling and empirical observations for modern NNs.




\subsection{Heavy-Tailed Self-Regularization (\HTSR)}

%A key aspect of a \SemiEmpirical theory involves the empirical data that are used semi-empirically within a broader theoretical framework.
%\charles{This needs reworked.   The framework is effcetive hamilitonians, many body theory, and stat mech.
%making it braoder dillutes the paper and the science. }
%For this, we need ``observables'' that are empirically-measurable.
%For nuclear structure, this involved \michael{XXX};
%for quantum chemistry, this involved \michael{XXX}.
%See Table~\ref{tab:table-of-analogies} for a summary.


%\michaeladdressed{For DNNs, one might be tempted to parameterize the theory in terms of the training data, algorithm learning rates, gradient information, etc.
%However, none of these are ``observable'' to the model user.
%Instead, our \SemiEmpirical theory will be formulated in terms of weight matrices of trained DNN models}
%\charles{This is incorrect. Of course these are observables}

%\footnote{We will assume that the user has access to a full pre-trained model. The basic \SemiEmpirical approach is still applicable, but would need to %be modified appropriately, under more limited access models such as only API access.}
%and in particular in terms of the Empirical Spectral Distribution (ESD), i.e., the eigenvalues,%
%\footnote{We will be somewhat cavalier about the terminological distinction between eigenvalues and singular values.}
%of weight matrices.
%For this, we will use results from \emph{\HeavyTailedSelfRegularization} (\HTSR) theory~\cite{MM19_HTSR_ICML,MM20_SDM,MM18_TR_JMLRversion}.

\HTSR theory is an approach that combines ideas from \STATMECH~with those of \emph{\HeavyTailed} \emph{\RandomMatrixTheory} (\RMT),
providing eigenvalue-based quality metrics that correlate with model quality (i.e., out-of-sample performance).
\michael{We need to be pretty pedantic about model quality versus OOS performance versus generalization, etc.}
\HTSR theory posits that well-trained models have extracted subtle correlations from the training data, and that these correlations manifest themselves in the \SHAPE and \SCALE of the eigenvalues of the layer weight matrices $\mathbf{W}$. 
In particular, if one computes the empirical distribution of the eigenvalues, $\lambda_i$, of an individual  $N \times M$ weight matrix, $\mathbf{W}$, then this density, $\rho^{emp}(\lambda)$, which is an ESD, is Heavy-Tailed (HT) and can be well-fit to a \emph{\PowerLaw} (PL), i.e., $\rho(\lambda)\sim\lambda^{-\alpha}$, with exponent $\alpha$.
\HTSR~theory provides a \emph{\Phenomenology} for qualitatively-distinct phases of learning~\cite{MM18_TR_JMLRversion}.
It can, however, also be used to define \emph{Layer-level Quality metrics} and \emph{Model-level Quality metrics}: e.g., the \ALPHA~$(\alpha)$ and \ALPHAHAT~$(\ALPHAHATEQN)$ PL metrics, described below.

Not needing any training data, \HTSR~theory has many practical uses.
It can be directly applicable to large, open-source  models where the training and test data may not be available.
Model quality metrics can be used, e.g., to predict trends in the quality of SOTA models in computer vision (CV)~\cite{MM20a_trends_NatComm} and natural language processing (NLP)~\cite{YTHx22_TR,YTHx23_KDD}, both during and after training, and without needing access to the model test or training data.
Layer quality metrics can be used to diagnose potential internal problems in a given model, or (say) to accelerate training by providing optimal layer-wise learning rates \cite{NEURIPS2023_CHM} or pruning rations \cite{alphapruning_NEURIPS2024}.
%
Most notably, the \HTSR theory provides \emph{Universal} \LayerQuality metrics encapsulated in what apears to be a critical exponent, $\alpha=2$, that is empirically associated with optimal or \IdealLearning. Moreover, as argued below, the
value $\alpha=2$ appears to define a phase boundary between a generalization and overfitting, analogous
to the phase boundaries seen in \STATMECH theories of NN learning. 

These results both motivate the search for a first prinples understanding of the \HTSR theory, 
and suggests a path for developing a practical predictive theory of Deep Learning.
For this, however, we need to go beyond the \Phenomenology provided by \HTSR theory, to relate it to some sort of (at least semi-rigorous/semi-empirical) derivations based on the \STATMECH theory of learning, and drawing
upon previous success (in Quantum Chemistry) in developing a first principles \SemiEmpirical theory. 


\subsection{What is a Semi-Empirical Theory?}

Historically, one of the most well known \emph{\SemiEmpirical} methods comes from \NuclearPhysics.
The \SemiEmpirical Mass Formula, dating back to 1935, is based on the heuristic Liquid Drop Model of the nucleus,
and it was used to predict experimentally-observed binding energies of nucleons. 
This model describes nuclear fission, and it was central to its development of the atomic bomb:
\begin{quote}
  Prior to WWII, \NuclearPhysics was a phenomenological science, which relied upon experimental data and descriptive
  models~\cite{Negele05}.
\end{quote}
In the Post-war era, the epistemological nature of nuclear theory changed,
as it saw the development of \SemiEmpirical shell models of the nucleus.
These models were formulated with rigor (in the physics sense)
but also relied on heuristic assumptions and experimental data for accurate predictions.
They  captured  the structure of atomic nuclei
and could accurately describe various nuclear properties\cite{Ivanenko1932, GoeppertMayer1949, Jensen1949}.
The shell models, analogous to the electronic shell structure of atoms,
represented a shift toward a more rigorous understanding of nuclear phenomena.

About this time, \RMT~itself was also introduced by \emph{\Wigner}~\cite{Wigner55}
to model the statistical patterns of the nuclear energy spectra of 
strongly interacting heavy nuclei.
These patterns were universal, independent of the specific nucleus,
suggesting that a probabilistic approach would be fruitful.
In the following decades, \RMT saw many advances, including the development of
the Marchenko-Patur model\cite{MarchenkoPastur1967},
and numerous other applications in physics\cite{Guhr1998}.
By the 1990s, \RMT~was further expanded when \emph{Zee} introduced the \emph{Blue Function},
and reinterpreted the \emph{R-transform} as a self-energy within the
framework of many-body / quantum field theory (QFT)~\cite{Zee1996}.
Also, so-called \HCIZtext integrals, integrals over random matrices,
were being used both to model disordered electronic spectra\cite{SchultenRMT},
and, later, the behavior of spin glass models\cite{Bouchaud1998,Cherrier_2003}.

Returning to the 1950s, and prior to the development of highly accurate, modern,
computational \emph{ab initio} theories of \QuantumChemistry,%\cite{ChemistryNobel1988},
theoretical chemists introduced the \SemiEmpirical PPP method
%to interpret chemical experiments and make
%predictions about
for
conjugated polyenes~\cite{PariserParr53}%,
The PPP model recast the electronic structure problem as an \emph{\EffectiveHamiltonian}
for the $\pi$-electrons.
\footnote{The PPP model resembles the later developed tight-binding model of condensed matter physics\cite{Hubbard1963}}
For many years this and related \SemiEmpirical methods
worked remarkably well, even better than the existing \emph{ab initio}
theories\cite{Dewar1975,Ridley1973,Stewart1990,Warshel1976}
Most importantly, these methods could be \emph{fit}
on a broad set of empirical molecular data, and then applied to molecules not in the original training set.

Around the same time, \emph{Löwdin} first formalized the concept of the \EffectiveHamiltonian,
which allowed the reduction of complex many-body problems to simplified
\emph{Effective Potentials} that still captured the essential physics.
Then in the late 1960s Brandow 
developed an \EffectiveHamiltonian theory of
nuclear structure, 
leveraging the \emph{Linked Cluster Theorem} (LCT) (see \cite{Hubbard1959}) and quantum mechanical many-body theory
to describe the highly correlated effective interactions in a reduced model space.
\footnote{Note also that the LCT shows that the log partition function $(i.e., \ln Z)$ can be expressed
a sum of connected diagrams, which is very similar to our result below, which expresses the
log partition function here as a sum of matrix cumulants from RMT.}

Like modern NNs, these \SemiEmpirical methods of \QuantumChemistry
worked well beyond their apparent range of validity,
generalizing very well to out-of-distibution (OOD) data.  This led to the
search for a \SemiEmpirical \emph{Theory} to explain the
remarkable performance of these phenomenological methods.
Building on Brandow’s many-body formalism, Freed and collaborators
\cite{freed1977, Freed1983}
developed an \emph{ab initio} \EffectiveHamiltonian
Theory of \SemiEmpirical methods  to explain the remarkable success of the \SemiEmpirical methods.
Specifically, the values of the PPP empirical parameters could be directly computed
effective interactions, including both renormalized self-energies and higher-order terms.
 Somewhat later, in the 90s,
 Martin et. al.\cite{MartinFreed1996, Martin1996, Martin1996_CPL, Martin1998}
 extended and applied this \EffectiveHamiltonian theory
 and demonstrated  the \Universality of the \SemiEmpirical PPP parameters numerically.
 Indeed, it is this \Universality that enabled the for-a-time inexplicable
 ODD performance of these methods.
 Crucially, this decades long line of work established a comprehensive
 analytic and numerical \emph{Theory} of \SemiEmpirical methods.
 That is, a framework that confirmed the empirically observed \Universality,
 provided theoretical justification for this,
 and  enabled systematic improvements of the methods using numerical techniques.
  %bridging the gap between computationally expensive ab initio approaches and purely empirical techniques. This dual capability of providing explanatory power and enabling refinements made their contributions foundational to the evolution of computational \QuantumChemistry.

 Finally, it is important to mention the \EffectiveHamiltonian approach provided by the Wilson \emph{\RenormalizationGroup}
 (RG).\cite{NobelPrizeRG,PhysRevLett.69.800}
 The RG approach provides a powerful framework for studying strongly correlated systems across different scales,
 enabling the construction of an \EffectiveHamiltonian by \emph{integrating out} weakly-correlated degrees of freedom  in a \ScaleInvariant way.
 It is particularly suited for critical points and phase boundaries--
 such the phase boundary between generalization and memorization in spin glass models of neural networks--
 and, importantly,
 predicts the  exisitence of Universal Power Law (PL) exponents .
 
 \paragraph{Relevance to Deep Learning}
 
 In this sense, \SemiEmpirical theories of \NuclearPhysics and \QuantumChemistry,
 (as well as the \RenormalizationGroup approach),  seem particularly appropriate
  for Deep Learning.
  DNN models are complex black boxes that defy statistical descriptions  they are commonly pre-trained on a large set of data; and than applied to new data sets in new domains via transfer learning.
  Most recently, the inexplicable success of transfer-learning is seen
in the GPT (Generative Pre-trained Transformer) models\cite{Radford2018},
and motivated early work by Jumper et. al. on protein folding\cite{JKS16_TR}

In contrast, these \SemiEmpirical approaches differ from more
recently developed theoretical approaches to deep learning, which are typically based on SLT, rather than \STATMECH~\cite{Roberts2021}.
In particular, there have recently appeared several theories of deep learning, formulated using ideas from \RMT.
However, regarding realistic models, it has been explicitly stated that
``These networks are however too complex in general for developing a rigorous theoretical analysis on the spectral behavior~\cite{LBNx17_TR}.
Even in recent work applying \RMT~to NNs, it has been noted
``\emph{that we make no claim about trained weights, only random weights}''~\cite{Yang2021}.
The weight matrices of a trained NN, however, are clearly \emph{not} simply random matrices---since they encode the specific correlations from the training data.


%

\subsection{A Semi-Empirical Theory of Learning (\SETOL)}

We propose \SETOL, a \SemiEmpirical Theory for Deep Learning Neural Networks (NNs),
as both a theoretical foundation for \HTSR \Phenomenology
and a novel framework for predicting the properties of complex NN models.
This unified framework offers a deeper understanding of DNN generalization
through a \SemiEmpirical approach inspired by many-body physics,
combined with a classic \STATMECH model for NN generalization.
Specifically, \SETOL combines theoretical and empirical insights to evaluate \ModelQuality,
showing that the weightwatcher layer \HTSR PL metrics (\ALPHA and \ALPHAHAT)
can be derived using a phenomenological \EffectiveHamiltonian approach.
This approach expressses the \HTSR \LayerQuality in terms of the RMT matrix cumulants
of the layer weight matrix $\WMAT$,
and is governed by a \ScaleInvariant transformation equivalent
to a single step of an exact \RenormalizationGroup (RG) transformation.
Here, we derive this from first principles, requiring no previous knowledge of statistical physics.

The \SETOL approach unifies the \HTSR principles with
a broader theoretical framework for layer analysis.
The \HTSR theory identifies \Universality (e.g., $\alpha=2$) as a hallmark of the best-trained DNN layers,
and, here, our\SETOL introduces the closely related \emph{Trace Log Condition}, a \ScaleInvariant or
\VolumePreserving transformation that reflects an underlying \emph{Conservation Principle}.
Together, these principles form the theoretical foundation for deriving \HTSR \LayerQuality metrics from first principles.
By leveraging techniques from \STATMECH and modern \RMT, \SETOL offers a rigorous framework
to connect empirical observations with theoretical predictions, advancing our understanding of generalization
in neural networks.

\begin{itemize}
\item
  \textbf{Derivation of the HTSR Layer Quality metrics $\ALPHA$ and  $\ALPHAHAT$}
  The \SETOL approach takes as input the
  \EmpiricalSpectralDensity (ESD) of the layers
  of trained NN, and  derives an expression for the approximate \emph{\AverageGeneralizationAccuracy}
  of a multi-layer NN, We call this approximation the \emph{\ModelQuality}, denoted $\Q^{NN}$
  This \ModelQuality is  expressed as product of individual \LayerQuality terms, $\Q^{NN}_{L}$,
  which themselves can  then 
  be directly related to the \HTSR Power Law (PL) empirical $\ALPHA$ ($\alpha$)
  and $\ALPHAHAT$  ($\ALPHAHATEQN=\ALPHAHATLONG$) metrics.

  In particular, the \LayerQualitySquared, $\QT\approx[\Q^{NN}_{L}]^{2}$, is
  expressed the logarithm of an \HCIZtext integral, the \ThermalAverage of an \EffectivePotential
  for a matrix-generalized form the Linear Student-Teacher model of classical \STATMECH. This \HCIZtext
  integral evaluates into the sum of integrated \RTransforms
  from \RMT, or, equivalently, as a sum of integrated matrix cumulants.
  From this, the \HTSR $\ALPHAHAT$ metric can be derived in the special case of \IdealLearning.
  \footnote{The \SETOL approach to the \HTSR theory resembles
  in spirit the derivation of the \SemiEmpirical PPP models using
  the \EffectiveHamiltonian theory, where each phenomenological parameter is associated with a renormalized
  effective interaction, expressed as a sum of linked diagrams or clusters.\cite{Martin1996, Martin1998}}

   \item 
     \textbf{Discovery of a Mathematical Condition for Ideal Learning.}
     By \IdealLearning, we mean that the specific NN layer has optimally converged, capturing as
     much of the information as possible in the training data without overfitting to any part of it.
     In defining this, and deriving our results, we have discovered (and are proposing) a new condition
     for \IdealLearning, which is associated with the \Universality of the \HTSR theory
   \begin{itemize}
      \item 
        \textbf{\HTSR Condition for Ideal Learning.}
        This \HTSR theory states that a NN layer is \Ideal  when the ESD can be well fit to a
        Power Law (PL) distrubtion, with PL exponent $\alpha = 2$. Importantly, 
        this appears to be Universal property of all well trained NNs, independent of the training data,
        model architecture, and training procedure.
      \item 
        \textbf{\SETOL TRACE-LOG Condition  for Ideal Learning.}
        The \SETOL condition for \IdealLearning states that the 
        dominant eigencomponents associated with the ESD 
        of layer form a reduced-rank \emph{Effective Correlation Space} (\ECS) that satisfies
        a new kind of Conservation Principle
        or \emph{\VolumePreservingTransformation} such that the largest eigenvalues $\LambdaECS_{i}$ of the~\ECS satisfy
        the condition  $\ln \prod \LambdaECS_{i} = \sum \ln \LambdaECS_{i} = 0$.  
        This is called the \emph{TraceLog Condition}.
   \end{itemize}

   The \HTSR Condition has been proposed and analyzed previously~\cite{MM18_TR_JMLRversion,MM20a_trends_NatComm,YTHx23_KDD}; but
   the TRACE-LOG Condition is new, based on our \SETOL theory.
   When these two conditions align, we propose the NN layer is in the \Ideal state.

   \item 
   \textbf{Experimental Validation.} 
   We present detailed experimental results on a simple model, along with observations on large-scale pretrained NNs, to demonstrate that the \HTSR conditions for ideal learning $(\alpha = 2)$ are experimentally aligned with the independent \SETOL condition for ideal learning
   $(\Det{\XECS}=1)$. 
   See Section.~\ref{sxn:empirical-trace_log}.
   Our primary objective here is not to demonstrate performance improvements on SOTA NNs---this has been previously established \cite{NEURIPS2023_CHM}. 
   \michael{@charles: why that ref? don't we want Nat Comm and Yaoqing's KDD paper?}
   Instead, our aim is to \textbf{validate the theoretical assumptions} of \SETOL, test the \textbf{predictions of the \SETOL framework}, and examine the \textbf{new, independent learning conditions} we discovered---on a model that is sufficiently simple that we can evaluate and stress test the theory.

   \item 
   \textbf{Observations on Overfit Layers ($\alpha < 2$).} 
   Being a \SemiEmpirical theory, \SETOL can also identify violations of it's assumptions.
   For example, when empirical results show $\alpha < 2$ for a single layer, the layer's ESD falls into the \HTSR \VeryHeavyTailed (VHT) Universality class.
   (See Section~\ref{sxn:hysteresis_effect}.)
   When this happens, the layer may be slightly overfit to the training data, resulting in \textbf{suboptimal performance} and potentially even exhibiting \textbf{hysteresis-like effects} (memory effects)---that we observe empirically.
   These effects indicate that overfit layers may retain memory-like behavior, affecting learning dynamics and generalization.
\end{itemize}


%%%  We derive the an analytic expression for a general \LayerQuality metric that validates the \HTSR~approah
%%%  %Building on new old and new techniques from \STATMECH~and \RMT,
%%%%
%%%%  as well as previously-reported empirical results~\cite{MM18_TR_JMLRversion,MM20a_trends_NatComm}, to present a \SemiEmpirical the%ory for and derivation of the $\hat{\alpha}$ metric from the phenomenological \HTSR~Theory.
%%%%$\alpha$ is the average of fitted power law exponents over layers in a given model;
%%%%$\hat{\alpha}$ is the weighted average, where one weights by the log spectral norm of the corresponding layer; 
%%%%smaller values of $\alpha$ in the regime (a HT \Universality class~\cite{MM18_TR_JMLRversion}) of $(2,4)$ correspond to better-trained models; and
%%%%$\hat{\alpha}$ is strongly predictive of model quality, across a wide range of SOTA CV and NLP models~\cite{MM20a_trends_NatComm,YTHx22_TR,YTHx23_KDD}.
%%%\item
%%%In the course of our derivation, we discovered a new and independent mathematical pre-condition for \Ideal learning for a given trained layer in a NN: \\
%%%the \HTSR~condition: $\alpha \sim 2$; and \\
%%%the \SETOL~conditions:  \TRACELOG: $\ln\prod\lambda_{i}=\sum\ln\lambda_{i}=0$ for the eigenvalues in the tail of the ESD: $\lambda_{i}\ge\lambda_{max}$. \\
%%%%That is, in SETOL, we consider change  the measure of random layer weight matrices
%%%%#$d\mu(\mathbf{W})\rightarrow d\mu(\mathbf{X}^{\EFF})$, where $\mathbf{X}:=\mathbf{W}^{T}\mathbf{W}$.
%%%%When we restrict $\mathbf{X}$ to the \EffectiveCorrelationSpace $\mathbf{X}^{\EFF}$,
%%%%5that only contains the generalizating components of the NN model, this leads to
%%%%the new condition for \Ideal learning, $\mbox{det}(X^{\EFF}) = 1$.
%%%%While completely independent mathematically,
%%%%it turns out, these two conditions empirically coincide both on a simple model
%%%%and for many large-scale pretreind models.
%%%\nred{
%%%XXX.  NEED SOME MORE SETUP TO SAY WHAT $X$ IS, ETC.
%%%XXX.  EXPLAIN THE IDEAS HERE IN A MORE SELF CONTAINED WAY.
%%%XXX.  WE SHOULD SAY SOMETHING ABOUT CONNECTION IN TERMS OF NECESSARY OR SUFFICIENT, E.G., DO WE GET IDEAL LEARNING IF EW SATISFY ONE OR THE OTHER OR BOTH OR NEITHER (PROBABLY BOTH?).}
%%%\chris{I would describe it as $\alpha=2$ and $\mbox{detX} =1$ are both necessary for {\em \IdealLearning}, but \IdealLearning 
%%%is {\em not} necessary for high accuracy. \IdealLearning doesnt mean the model learns the concept perfectly -- it could 
%%%be unrealizeable -- but it does mean that in some sense the {\em maximum amount of information has been encoded in the 
%%%tail by SGD... Nevertheless, they correlate well with high accuracy, even when one or the 
%%%other is relaxed.}}
%%%\charles{agree to some extent. We built steam enegines before we understood thermodynamics, and they worked..but sometimes they blew up.
%%%  Theory is tool for engineers to build systems in a consistent and reliable way. }
%%%\item
%%%We present detailed experimental results on a very simple model,
%%%as well as observations on large scale pretrained NNs, to show that
%%%the \HTSR~theory conditions for \Ideal learning ($\alpha=2$) are
%%%experimentally aligned with the independent SETOL condition for \Ideal learning
%%%($\mbox{det}(X^{\EFF}) = 1$).
%%%Our goal is not to show that this approach performs well on SOTA NNs, as that has been previously established~\cite{MM20a_trends_NatComm,YTHx23_KDD}; insetad our goal is to test the assumptions of the theory, test the predictions of the theory, and text the new independent conditions we discovered.
%%%\item
%%%  We argue that when we empirically observe that $\alpha < 2$ \emph{for a single layer}, i.e., the layer ESD is in the
%%%  \HTSR~Very Heavy Tailed (VHT) Universality class, than
%%%the layer appears to be slightly overfit to the training data, corresponding to suboptimal performance,
%%%and can even exhibit hysteresis-like (i.e., memory) effects, 
%%%that we argue we observe empirically.
%%%\end{itemize}
%%%
%%%\noindent
%%%In the remainder of this introduction, we place our main results in a broader context.
%%%
%%%\charles{Can we just remove this paragraph?}
%%%\nred{
%%%Technically, \SLT~seeks bounds on the worst-case behavior of a model, whereas \STATMEC~seeks \Typical-case behavior~\cite{DKST96,EB01_BOOK}.
%%%Operationally, this difference usually amounts to how one deals with the specifies of limiting behavior, in the sense that certain limits are more amenable to one style of analysis or the other~\cite{MM17_TR,Liam_21_JSM_JRNL}.
%%%In the VC approach, one typically takes the so-called \emph{VC limit}, in order to ensure appropriate concentration of measurable quantities of interest~\cite{Vapnik98}. 
%%%More recent \SLT~models consider the limit of an infinitely wide network~\cite{Yang2021}.
%%%%\michael{What is the ref~\cite{YangMS5}?  Is it~\cite{Yang2021}?}
%%%In \STATMECH, one employs the so-called \ThermodyamnicLimit, where, e.g., the number of data points and the model complexity diverge together.
%%%If you want to Read The Fundamental Material, than here it is:~\cite{SST92,WRB93,DKST96,EB01_BOOK}.
%%%}
%%%



%
\subsection{Outline}

Here is an outline of the sections.

Section~\ref{sxn:htsr}: \HeavyTailedSelfRegularization Theory

Section~\ref{sxn:setol}: Overview of \SETOL %Spuriously Heavy Tailed ESDs

Section~\ref{sxn:SMOG_main}: \StatisticalMechanics of Generalization

%Section~\ref{sxn:SETOL}: SETOL: A \SemiEmpirical Theory of (Deep) Learning

Section~\ref{sxn:matgen}: Matrix Generalization of the \StudentTeacher Model

Section~\ref{sxn:empirical}: Empirical Studies

Section~\ref{sxn:discussion}: Discussion and Conclusion

Appendix~\ref{sxn:appendix}: Appendix




