\nred{REWORKED THIS SECTION; stil in progress}
\charles{If we include this section, we should cite the specific results from the old SST paper, including a plot
and we should cite our older paper.   but, importantly, we should prep the read for the empirical results in
section 6 on correlation traps and glassy dynamics, and we should discuss $F$ become non-extensive and/or non-analytic}
\subsection{Bad Training Data and the Spin Glass Phase}
\label{sxn:SMOG_main-spin_glass}

When training data is of low quality---whether due to mislabeling, inherent noise, or a generally unrealizable task---neural networks exhibit behavior that diverges sharply from what \SLT\ would predict under cleaner data assumptions. In \SLT, one typically studies the \GeneralizationError as a function of model complexity and data distribution, often assuming that data is realizable. By contrast, a \STATMECH-based analysis highlights that, when data itself is contradictory or highly random, the network can enter a \emph{spin glass--like phase}, marked by frustration and disorder.

\paragraph{From Annealed to Quenched: The Role of Disorder.}
In the \StudentTeacher\ model, the \AnnealedApproximation (AA) supposes that one can simply average over data disorder. However, with \emph{bad} training data---random or conflicting labels---that assumption breaks down. A \emph{quenched} treatment, which holds the disorder fixed while averaging over model parameters, shows that the perceptron's loss landscape fragments into many local minima, much like a spin glass in physics, and characterized bv \emph{Frustration}.

Frustration can arise when facing mislabeled or contradictory training examples, leading to:
\begin{itemize}
\item Multiple ``almost-satisfactory'' weight matrics coexist, each misclassifying a different subset of examples.
\item The abundance of local minima is characteristic of a \emph{spin glass} phase: no single global minimum stands out; the model can get ``stuck'' in a metastable configuration.
\end{itemize}

In such disordered regimes, the AA often underestimates the difficulty of learning.
In contrast, a fully \emph{quenched} theoretical analysis captures
how small random variations in data can drastically alter the learned weights $\WVEC$':
\begin{itemize}
\item \emph{Annealed}: Increasing the number of training examples always decreases training error, i.e. $(N \gg 1) \rightarrow (\DETOT\approx 0)$, and, also the generalization or test error.  \nred{comment on DD?}
\item \emph{Quenched}: In a spin glass regime, contradictory data can trap the model; adding more data may not improve (or can even worsen) generalization.
\end{itemize}

\paragraph{Practical Considerations}

Such \emph{spin glass} behavior highlights the pivotal role of data quality. Even if the network has vast capacity, it cannot reliably generalize when the dataset is fundamentally inconsistent. Instead, the model remains in a ``frozen'' disordered state:
\begin{itemize}
\item \emph{Overfitting}: The perceptron might merely memorize certain random labels, rather than extracting meaningful structure.
\item \emph{Instability}: Small perturbations in the data can dramatically change the final weights, reflecting a glassy landscape of solutions.
\item \emph{Phase Transitions}: As the fraction of contradictory (or noise-corrupted) data increases, one observes a shift from a ``learnable'' phase to a frustrated, spin-glass--like phase.
\end{itemize}

Hence, this spin glass analogy clarifies why purely annealed approaches can fail for mislabeled or unrealizable tasks: sufficiently disordered data can lock the model into a \emph{glassy} configuration, where it nominally ``fits'' parts of the data but loses coherent generalization capability. Detecting these heavy-tailed indicators ($\alpha < 2$) connects insights from \STATMECH\ to modern deep learning challenges.
