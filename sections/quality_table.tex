\begin{table}[H]
\centering
%\small
  \begin{tabular}{@{} l  p{0.50\linewidth}  l  c @{}}
  \toprule
\textbf{Symbol} 
  & \textbf{Definition \& Formula} 
  & & \textbf{Eq.\ \#} \\
\midrule
$\epsilon$  
  & \EffectivePotential (\LargeN, high-T)
  & 
    $\epsilon(\WVEC) = \langle \DETOPXI\rangle_{\AVGNDXI}$
  & \ref{eqn:epsl} \\[1ex]

$\eta$
  & \SelfOverlap\ (average accuracy):
  & $\eta(\WVEC) = 1 - \epsilon(\WVEC)$
  & \ref{eqn:def_eta} \\[1ex]

$\bar{F}_{\ND}$
  & Average Free Energy:
  & $\beta\,\bar{F}_{\ND} = -\langle\ln Z_{\ND}\rangle_{\AVGNDXI}$
  & \ref{eqn:mm_f_bar} \\[1ex]
$\Gamma_{\bar Q}$
  & Quality‐Generating Function:
  & $\Gamma_{\bar Q} = 1 - \bar{F}_{\ND}$
  & \ref{eqn:GammaBar} \\
$\bar Q$
  & Model \Quality\ (average self‐overlap):
  & $\bar Q = \THRMAVGw{\eta(\WVEC)}$
  & \ref{eqn:model_qualities} \\[1ex]
\bottomrule
\end{tabular}
\caption{Summary of the main intensive (average, per‐parameter) quantities.  Here $n$ is the number of free parameters.  The average model
  \Quality~$\bar Q$ is the model’s average accuracy (one minus the error), and the Quality‐Generating function $\Gamma_{\bar Q}$ plays the same role as the free energy $\bar F$ but with an opposite sign convention.}
\label{tab:intensive_quantities}
\end{table}
