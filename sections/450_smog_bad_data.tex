
\subsection{Bad Training Data and the Spin Glass Phase}
\label{sxn:SMOG_main-spin_glass}

When training data is of low quality—whether due to mislabeling, inherent noise, or a generally unrealizable task—neural networks exhibit behavior fundamentally different from predictions by \StatisticalLearningTheory (\SLT). In \SLT, the focus is often on the \GeneralizationError as a function of model complexity and data distribution. However, classic \STATMECH analysis shows that when data is mislabeled or the problem is fundamentally unsolvable, the behavior of the neural network changes, revealing a notable divergence between unquenched (annealed) and quenched analyses.

In such cases, the neural network perceptron layer may enter what is known as the spin glass phase. This phase signifies a highly disordered state in which the network struggles to find stable, consistent solutions, akin to the frustrated interactions seen in spin glass systems in physics.
Analogously, the \SETOL approach posits that, under these challenging conditions, the NN layer operates in a distinct regime,
particularly when the \HTSR parameter  $\alpha < 2$ and the \TRACELOG condition $\mbox{detX} < 0$.
These two independent, complementary conditions both places the layer within the \emph{\VeryHeavyTailed} (VHT) Universality class,
seems empirically to reflect instability and overfitting, and presumably on specific data instances or features.

This phenomenon diverges sharply from traditional learning theories, underscoring the role of data quality in neural network (NN)
behavior and offering a unique perspective for understanding overfitting and generalization in deep learning systems.


