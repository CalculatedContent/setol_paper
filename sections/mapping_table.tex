
\begin{table}[t]
  \raggedright
\hspace*{-0.5cm}% Adjust this value as needed
\renewcommand{\arraystretch}{1.25} % Increase line spacing in table
\begin{tabular}{|c|c|c|c|}
  \hline
  Quantity & Traditional \SMOG & \makecell{\LinearPerceptron \\ in Traditional \SMOG} & \makecell{Matrix Generalization \\ for \SETOL} \\ \hline

  Total \DataModel Error 
    & $\DETOPXI$ (\ref{eqn:detox})
    & $\DETOPSTL$ (\ref{eqn:deriveSTerror}) 
    & $\DETOPNN$ (\ref{eqn:DE2}) \\ \hline

   Annealed Hamiltonian
    & $\HANHT=\EPSLw$ (\ref{eqn:epsl}) 
    & $\GANHTR=\EPSLSTx=1-R$ (\ref{eqn:e0}) 
  & $\GANMATHT = \IM-\OVERLAP$ (\ref{eqn:GANHTmatR}) \\

  (Data-Averaged Error) 
    & (AA, at high-T) 
    & (and at large-N) 
    & (only for a layer)  \\ \hline

    \SelfOverlap 
    & $\ETAw = 1-\EPSLw$ 
    & $\ETA(\SVEC,\TVEC)=\tfrac{1}{N}\SVEC^{\top}\TVEC$ (\ref{eqn:eta_vec_avg_def})
    & $\ETA(\SMAT,\TMAT)=\tfrac{1}{N}\SMAT^{\top}\TMAT$ (\ref{eqn:eta_mat_avg_def})  \\ \hline
    \hline

  \ModelQuality 
    & $\Q:=1-\AVGGE$ 
    & $\Q^{ST}:=1-\AVGGE^{ST}$ (\ref{eqn:model_qualities})
   & $\Q^{NN}:=1-\AVGGE^{NN}$  (\ref{eqn:model_qualities})\\ 

  in terms of \LayerQuality
    & 
    & 
   & $\Q^{NN}:=\prod_{L} \Q^{NN}_{L}$ \\ \hline
\end{tabular}
\caption{Summary of key quantities compared across traditional \SMOG models,  the \Student-\Teacher (ST) \LinearPerceptron--in the \AnnealedApproximation
(AA) and at high-Temperature (high-T) and at large-$N$, and the matrix-generalized forms as the starting point to frame \SETOL.
The Total \DataModel Error represents the difference between the model and its labels for the ST model between
the \Student and \Teacher predictions.
The \AnnealedHamiltonian is the Energy function for this Error after it is averaged over the model for the training data
(an $n$ or $N$-dimensional i.i.d. Gaussian model, i.e.,  $\NDXIn$ or $\NDXI$).
In the AA, the \AnnealedHamiltonian is equal to the \EffectivePotential.  For the ST model,  this is one minus the overlap, $(1-R)$;
for the \SETOL, this is  the ($M$-dimensional) identity minus the overlap operator/matrix, $\IM-\OVERLAP$.
The \SelfOverlap $\eta(\cdots)$ is used to describe the Accuracy (as opposed to the Error) for both the ST model and
its matrix-generalized form.
%Notice that $\eta(\XI)$, as defined,  has not yet been averaged over the model data $\XI^{N}$.
Finally, the different forms of the \Quality are defined.  Generally speaking, the \Quality $\Q$ is an approximation to some measure
of $1$ minus the \AverageGeneralizationError, $(\Q:=1-\AVGGE)$ (in the AA, at high-T, at large-N, and with whatever else
approximations are applied).
For the ST model, having just 1 layer, the \ModelQuality and the \LayerQuality are the same, and denoted $\Q^{ST}$.
For \SETOL, the \ModelQuality $\Q^{NN}$ is a product of individual \LayerQualities $\Q^{NN}_{L}$.
(Note that the  final \SETOL \LayerQuality $\Q$ is defined in terms of the \LayerQualitySquared $\QT$,
and the starting point for this is expressed with the \LayerQualitySquared Hamiltonian $\HBARE=\OLAPTOLAP$).
}
\label{table:quantities_general_vect_matrix}
\end{table}

\charles{@MIKE:
\paragraph{Other notation suggetions and issues}  Why do we use $\epsilon(R)$ for the Annealed Error Potential
and not the Hamiltonian directly ?
We can define a Hamiltonian for the \LayerQualitySquared $\QT$ a
\begin{align}
  \mathbf{H}_{\QT}=\OVERLAP^{T}\OVERLAP \nonumber
\end{align}
which gives
\begin{align}
  Z_{\QT}:=\int d\mu(\SMAT) e^{N\beta\operatorname{Tr}[\mathbf{H}_{\QT}(\SMAT)]}
\end{align}
And this will become the RG \EffectiveHamiltonian.

But we can not define an $\mathbf{H}$ for the Unsquared  $\Q$ using  $\sqrt{\OLAPTOLAP}$ because when write the \PartitionFunction
we always use $\beta\operatorname{Tr}[\mathbf{H}(\SMAT)]$ in the exponential
\begin{align}
  Z:=\int d\mu(\SMAT) e^{N\beta\operatorname{Tr}[\mathbf{H}(\SMAT)]}
\end{align}
and for $\Q$ the $\sqrt{\cdots}$ needs to appear outside the Trace

\paragraph{Quality Approximation}
Why do we work with the \LayerQualitySquared and not the \LayerQuality directly?
For our matrix-genearlized model, we need to map
the \AnnealedHamiltonian for the \LinearPerceptron to a matrix,
$\HANHT(\WVEC)\rightarrow\mathbf{H}(\WMAT)$ in such a way that the new
Hamiltonian operator (i.e., matrix) is symmetric.  At high-T, we have $\HANHT(\WVEC)=\EPSLR=1-R$.
But since we want to compute the \LayerQuality, not the error, we could
define a Hamiltonian operator for the \LayerQuality, $\mathbf{H}_{\Q}(\WVEC)=R$.
This is very convenient.
To make this symmetric, we can define the
\LayerQualitySquared Hamiltonian $\mathbf{H}_{\QT}(\WMAT)=\mathbf{R}^{\top}\mathbf{R}$,
use this to compute the \LayerQualitySquared, $\QT$, 
and take the square root and end of the calculation to give
$\Q\approx\sqrt{\QT}$.


Moving forward, to compute the \LayerQuality $\Q$, we approximate the \ThermalAverage of the square root if the
Overlap squared with
the square root of the \ThermalAverage of the Overlap squared
\begin{align}
  \Q := \THRMAVGIZ{\sqrt{\OLAPTOLAP}} \approx \sqrt{ \THRMAVGIZ{\OLAPTOLAP}} 
\end{align}
We will define the \LayerQualitySquared $\QT$,
\begin{align}
  \Q :=  \THRMAVGIZ{\OLAPTOLAP}
  \end{align}
and, moving forward, work with this directly
}
