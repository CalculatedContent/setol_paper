% Gtable.tex  —  final closed-form (or leading-order) Layer Quality \Q
\begin{table}[h!]
  \centering
  \renewcommand{\arraystretch}{2}
  \small
  \begin{tabular}{|c|p{2.5cm}|c|c|} % Removed the last 'p{4.5cm}' column
    \hline
    \textbf{Model} & \textbf{Tail Norm} & \textbf{WW Metric} & \textbf{$\log\mathcal{Q}$ (Log Quality)} \\ \hline\hline

    Bulk$+$Spikes (BS)
    & Frobenius Norm
    & N/A
    & $\displaystyle \log\left(\sum_{i=1}^{\MECS}\LambdaECS_{i}\right)$ \\ \hline

    Free Cauchy (FC)
    & Spectral Norm
    & \ALPHA $\;\;\alpha$
    & $\displaystyle \log\lambda_{max}\sim 1/\alpha$ \\ \hline
    
    L\'evy Wigner (LW)
    & Shatten Norm
    & \ALPHAHAT $\;\;\hat{\alpha}$
    & $\displaystyle (\alpha-1)\,\log\lambda_{\max}}$ \\ \hline
     \hline
    \InverseMP (MP)\footnotemark
    & ECS Boundary*
    & \ALPHAHAT $\;\;\hat{\alpha}$
    & $\displaystyle \alpha[\log\lambda_{\max}-\log\lambda_{\min}]$ \\ \hline

  \end{tabular}
  \caption{Closed-form or leading-order expressions for the log \LayerQuality
            $\log\mathcal{Q}$ derived from the integrated $R$–transform for each core
            tail-model, simplified to show the relation to the ~\WW~\ALPHA and ~\ALPHAHAT metrics.
            For each model, we interpret the final result.
            Bulk$+$Space (BS): Sums the $\LambdaECS_{i}$ in the \ECS, giving a Frobenius Tail norm.
            Free Cauchy (FC): Yields the Spectral Norm, $\lambda_{max}$. Since this scales as $1/\alpha$, this explaines the~\HTSR~\ALPHA metric as it shows  why smaller $\alpha$ yields higher $\mathcal{Q}$.
            L\'evy Wigner (LW): Yields a Shatten Norm, which can be approximated by~\ALPHAHAT, $\ALPHAHATEQN$,
            Being for the VHT Universality class, it implies that heavier tails ($\alpha\in(1,2)$) depress \LayerQuality.
            \InverseMP (IMP): Also gives ~\ALPHAHAT, and, importably, a branch cut that defines the \ECS Boundary
             }
  \label{tab:htsr_layer_quality}
\end{table}

\footnotetext{See Appendix~\ref{sxn:IW}, Eq.\,(A.7.25): $|G(\lambda)|\approx1.138\,\lambda^{0.539}$.}
