
\subsubsection{Aggressively Applying the Annealed Approximation}

In our development of the \SETOL approach, we apply the AA ``more aggressively than it is usually applied in \STATMECH.
By this, we mean that we will apply it more generally than just to averages of the data $x\in\mathcal{D}$.
Doing so amounts to applying Jensens inequality \emph{as an equality}, which allows will let us interchange integrals and logarithms at will:
\begin{align}
\label{eqn:Jensens}
\int\ln\cdots \leftrightarrow \ln\int\cdots
\end{align}

\noindent
As an example, let us consider what happens when we ``aggressively apply the AA to the \FreeEnergy $F$ (as opposed to the quenched average  $\mathbb{E}[F]_{x}$).
Using Eqns~\ref{eqn:Zu} and~\ref{eqn:F}, we obtain
\begin{align}
\label{eqn:ZtoF}
F %%:=       -\dfrac{1}{\beta}\ln Z_{w}
   =       -\dfrac{1}{\beta}\ln \int d\mu(w) e^{-\beta E(w,x)}  %%\\ \nonumber
   \approx -\dfrac{1}{\beta}\int d\mu(w) \ln e^{-\beta E(w,x)}  %%\\ \nonumber
   \approx \int d\mu(w) E(w,x)                                  %%\\ \nonumber
   \approx \langle E(w,x)\rangle_{w}
\end{align}
where the final average $\langle E(w,x)\rangle_{w}$ is now just a Direct Average \st{or Likelihood}, i.e., 
in this approximation , we treat $F$ as a Direct Average of the Energy $E(w,x)$.

Notice that to obtain $F$, we maginalized out the variable $w$ from $E(w,x)$.
We now see the intitution behind a \FreeEnergy: it lets us marginalize out, or ``free, a variable from an Energy function.
(Note, however, that the \FreeEnergy is generally more complex than \EQN~\ref{eqn:ZtoF}, since $F$ can also include entropic contributions.)

\michael{Lets sync, so we can minimize or modify the subsequent disucssion about high-temperature, since it is a bit confusing. Do we need it. I think we are saying that the aggressive applincation of AA and the first order contribution to the high-$T$ limit are the same, but we dont really say that explicitly.}

More accurately, we can obtain this relation  by taking a \emph{High Temperature} (or small $\beta$) limit.
At very high temperatures, i.e., $( T \rightarrow \infty $ or $ \beta \rightarrow 0 )$,
the exponential Boltzmann factor tends to $1$,  $(e^{-\beta E(w,x)} \rightarrow 1)$,
for all energy values $ E(w,x)$.
Thus, the energy distribution broadens, and all microstates become nearly equally probable.
\chris{Equivalently: Thus, the Boltzmann distribution approaches a Uniform distribution over microstates.}
Note that in the high-T limit,
\FreeEnergy $F$ 
does not strictly become the average energy, but rather,
the difference $F -\langle E \rangle_{w}^{\red{T}} $ is minimized.

We also note that in the high-\(T\) limit, the free energy \(F(\beta)\) can be approximated as:
\begin{equation}
 \label{eqn:Ftaylor}
F(\beta) \approx F(0) + \beta \left. \frac{\partial F}{\partial \beta} \right|_{\beta=0} + \dots
\end{equation}
Assuming that at high \(T\), \(F(\beta) \rightarrow F_0\) and no longer depends significantly on \(\beta\), \EQN~\ref{eqn:Ftaylor} simplifies, highlighting that the average energy is approximately given by:
\begin{align}
\label{eqn:Fu2}
\langle E \rangle_{w}^{\text{high-T}} \approx \frac{\partial \beta F_0}{\partial \beta} = F_0.
\end{align}

Also, in this limit, we also note that the \ThermalAverage $\langle E\rangle_{w}$ becomes
a uniform average over all possible states of the system, and one also has the relation:
\begin{equation}
\label{eqn:avgE}
\langle E \rangle_{w}^{\red{T}} \approx
\langle E \rangle_{w}:=
\int d\mu(w) \ E(w,x),\;\;\beta\;small
\end{equation}
and the result resembles that of applying the AA aggressively.
%%This can also be see by noting that in the high-T limit, we can assume $F\rightarrow F_{0}$ no longer depends on $\beta$,so that
%the relation in \EQN~\ref{eqn:Fu1} becomes
%\begin{align}
%\label{eqn:Fu2}
%\langle E \rangle_{w}^{\red{T}} \approx \dfrac{\partial \beta F_{0}}{\partial \beta} = F_{0}.
%\end{align}



\subsubsection{Overloading the Notation for Averages}

In summary, in the AA, when applied ``aggresively, as if in a high-Temperature limit, the following relations hold: 
\begin{equation}
\label{eqn:FEE}
\langle E \rangle_{w}^{\beta} \approx F_{an} \approx \langle E \rangle_{w}  ,
\end{equation}
where $F_{an}$ denotes specifically an \Annealed \FreeEnergy. 
\michael{What is $F_{an}$, the high-$T$ FE, or the first order contribution, which I think is the aggressive application of the AA.}
Moreover, if we apply the AA ``in reverse, we can effective convert a Direct Average or Likelihood into a what looks like a \FreeEnergy.
This is exactly what is done in Section~\ref{sxn:matgen}; see \EQN~\ref{eqn:logAvgOverlap0}.

From \EQN~\ref{eqn:FEE}, we notice that we can overload the bracket notation $\langle\cdots\rangle$ for averages; and we will consider multiple types of averages.
For averages over data ($\langle\cdots\rangle_{x}$, $\langle\cdots\rangle_{\DX}$, $\langle\cdots\rangle{\XI}$, etc.), we always mean a direct average over some specific measure
(like a Gaussian measure).
For averages over weights ($\langle\cdots\rangle_{w}$, $\langle\cdots\rangle_{\mathbf{W}}$, $\langle\cdots\rangle_{\mathbf{S}}$, etc.),
we will be converting these between Thermal Averages, Direct Averages/Likelihoods, and averages of random matrices (HCIZ Integrals).

We can convert between averages like this because we are less concerned with computing rigourous bounds than with framing the modelling assumptions correctly and effectively.
While obviously these are gross approximations, whatever errors these introduce will be compensated for when the final expression used is fit with empirical data (hence, a \SETOL).
