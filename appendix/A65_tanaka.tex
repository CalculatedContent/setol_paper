\subsubsection{Selecting $\AMAT:=\AMAT_M$ instead of $\AMAT_N$}
\label{sxn:tanaka_end}
In principle, we could have selected $\AMAT:=\AMAT_M=\tfrac{1}{N}\SMAT^{\top}\SMAT$  for the \Student Correlation matrix,
thereby avoiding the discussion on the \DualityOfMeasures altogether.
Doing this, however, would make $\AMAT$ $M\times M$, thereby
require defining the \SourceMatrix $\DMAT$ as an
$N \times N$ matrix, with presumably $N-M$ zero eigenvalues.
This would cause $\DMAT$ to violate the condition $\DeltaMu > \beta\lambda$
for all eigenvalues $\lambda$ of $\AMAT_M$.
In this case, it would be challenging to define the large-$N$ limit.
