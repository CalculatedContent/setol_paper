\subsubsection{Selecting \texorpdfstring{$\AMAT:=\AMATM$}{A:=A M} instead of 
\texorpdfstring{$\AMATN$}{A N}}
\label{sxn:tanaka_end}
In principle, we could have selected $\AMAT:=\AMATM=\tfrac{1}{N}\SMAT^{\top}\SMAT$  for the \Student Correlation matrix,
thereby avoiding the discussion on the \DualityOfMeasures altogether.
Doing this, however, would make $\AMAT$ $M\times M$, thereby
require defining the \SourceMatrix $\DMAT$ as an
$N \times N$ matrix, with presumably $N-M$ zero eigenvalues.
This would cause $\DMAT$ to violate the condition $\DeltaMu > \beta\lambda$
for all eigenvalues $\lambda$ of $\AMATM$.
In this case, it would be challenging to define the \LargeN limit in $N$.
