\subsubsection{Relating $\GNORM(\lambda)$ to the Integrated R-Transform $R(x)$}
\label{sxn:tanaka_end}
\nred{A64\_tanaka}
\\
Having completed both steps, let us combine Eqns.~\ref{eqn:ZD_step1} and~\ref{eqn:ZD_step2}, giving
\begin{align}
M\ln(\dfrac{\red{2}\pi}{\beta})-\sum_{\mu=1}^{M}\int \ln(\delta_{\mu}-\lambda)\rho^{\infty}_{\AMAT}(\lambda)d\lambda
=
M \left( 1 + \ln \frac{2\pi}{\beta} \right) - \sum_{\mu=1}^{M} \min_{\delta_{\mu}} \left[\delta_{\mu}\lambda_{\mu}
- \GNORM(\lambda_{\mu}) + \ln \lambda_{\mu} \right] 
\end{align}
The $\ln \frac{2\pi}{\beta}$ term cancels from both sides, giving
\begin{align}
-\sum_{\mu=1}^{M}\int \ln(\delta_{\mu}-\lambda)\rho^{\infty}_{\AMAT}(\lambda)d\lambda
=
M - \sum_{\mu=1}^{M} \min_{\delta_{\mu}} \left[\delta_{\mu}\lambda_{\mu}
- \GNORM(\lambda_{\mu}) + \ln \lambda_{\mu} \right] 
\end{align}
Since this is true for every $\mu$, we can solve this for any arbitrary eigenvalue $\lambda_{\mu}$.
Dropping the $\mu$ subscript, we have the following identity
\begin{align}
\int \ln(\delta-\lambda)\rho^{\infty}_{\AMAT}(\lambda)d\lambda
=
1 - \min_{\delta} \left[\delta\lambda - \GNORM(\lambda) + \ln \lambda \right] 
\end{align}

\charles{NEED TO FINISH THIS.  For now, you can gather the rest of this from the original papers by Tanaka.
I will finish this shortly.}

\paragraph{Defining the Source Matrix}
We now fix (the Lagrange Multipliers) $\mathbf{\Delta}$, which amounts to choosing the eigenvalues $\delta_{\mu}$ of the \SourceMatrix $\mathbf{D}$

Assume that
$\mathbf{X}$ and $\mathbf{D}]$ are simultaneously diagonalize-able  (i.e, the commutator between them is zero:
$[\mathbf{X}, \mathbf{D}]=0$)

\nred{MORE to convert here...also, I think we should lead with this}

Now introduce the
\emph{Inverse Legendre Transform}

\nred{more to do}

\paragraph{Cauchy Transform}

\begin{equation}\label{eqn:cauchy-1}
    \mathcal{C}_{\AMAT}=\int d\lambda\dfrac{\rho_{\AMAT}(\lambda)}{\delta-\lambda}=0
\end{equation}

\begin{equation}\label{eqn:cauchy-2}
    \underset{\delta}{\min}\left[\cdots\right]=q-\int
    d\lambda\dfrac{\rho_{\AMAT}(\lambda)}{\delta-\lambda}=0
\end{equation}

\begin{equation}\label{eqn:cauchy-3}
    q*=\int
    d\lambda\dfrac{\rho_{\AMAT}(\lambda)}{\delta-\lambda}=0
\end{equation}

\paragraph{R-Transform}
\begin{equation}\label{eqn:R-trans}
    R_{\AMAT}(z)=\mathcal{C}_{\AMAT}(z)-\dfrac{1}{z}
\end{equation}

\begin{equation}\label{eqn:R-trans-2}
    \mathcal{C}_{\AMAT}\left(R_{\AMAT}(z)-\dfrac{1}{z}\right)=z
\end{equation}

\begin{equation}\label{eqn:gen}
    \mathcal{G}(q)=\int^{q}dz R_{\AMAT}(z)
\end{equation}

\nred{Some extra stuff: maybe have
    this at the beginning, or in an appendix, to explain the intent of whts going on}


MISC
\michael{non-Gaussian tails relate to the rate function---get cumulant generating function, and express the results i.t.o. the cumulants---since express the integral over teachers i.t.o. a large deviation principle, with these last equations, 
\eqref{eqn:rate}, 
\eqref{eqn:legendre-transform},
\eqref{eqn:legendre-transform2}.  So get cumulant generating function, so we express generaliz i.t.o. cumulants.}


\nred{PLACE THIS:
Up until now, the source matrix $D$ has been a free parmater; now
we will show that $D$ should be chaosen such that its eigenvalues $\delta_{i}$ 
satisfy the following equation:

The eigenvalues $\delta_{i}$ of $\mathbf{D}$ will be constrained, as shown below,
using a \SaddlePointApproximation , 
giving, for each eigenvalue pair $(\delta_{i},\lambda_{i})$
\begin{equation}
\lambda_{i}=\int\dfrac{\rho_{\AMAT}(\lambda)}{\delta_{i}-\lambda}d\lambda,\;\;i\in[1,M]
\end{equation}
}



MISC
We can evaluate \EQN~\ref{eqn:hciz2}
using a \SaddlePointApproximation (SPA) such that the HCUIZ integral $\HCIZ$ becomes
\begin{equation} 
  \label{eqn:hciz3}
\HCIZ := \int d\mu(\mathbf{\AMAT})\left[\exp\left(\beta\Trace{\mathbf{W}^{T}\mathbf{\AMAT}\mathbf{W}}\right)\right]
     \simeq\exp\beta N Tr[\GNORM(\mathbf{X})]
\end{equation}

Upon taking the lnarithm, one has
\nred{Are we missing an $N$ here ?}
\begin{align}
\label{eqn:hciz4}
\IZG = \lim_{N\rightarrow \infty}\frac{1}{N} \ln\exp\beta\left[N \Trace{\GNORM(\mathbf{X})}\right] = \beta \Trace{\GNORM(\mathbf{X})}
\end{align}
where
\begin{equation}
\label{eqn:hciz-moment}
Tr[\GNORM(\mathbf{X})]= \sum_{i=1}^{M}\GNORM(\lambda_{i}) .
\end{equation}
%
We will find that $\GNORM(\lambda)$ is simply the integral of the \RTransform $R_{\AMAT}(z)$ of $\mathbf{\AMAT}$
of Random Matrix Theoryt (RMT).
%
\begin{equation}\label{eqn:rint}
    \GNORM(\lambda)=\int_{0}^{\lambda}R_{\AMAT}(z)dz.
\end{equation}
%


the Legendre Transform of the (Scaled) \CumulantGeneratingFunction for the measure $\mu(\mathbf{X})$.


