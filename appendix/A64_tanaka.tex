\subsubsection{Expressing the~\GEN~$(\GNORM(\lambda))$ as the Integrated~\RTransform~$(R(z))$ of the~\CorrelationMatrix~$(\AMAT)$}
\label{sxn:tanaka_end}
Having completed both steps, let us combine Eqns.~\ref{eqn:ZD_step1},~\ref{eqn:ln-Z-Nlim2}
with~\ref{eqn:ZD_step2} and~\ref{eqn:gmax_final}.
We follow the first arguments by Tanaka\cite{Tanaka2007} (which follows Cherrier\cite{Cherrier2003}).
\begin{align}
   M\ln(\dfrac{{2}\pi}{\beta})-\sum_{\mu=1}^{M}\int \ln(\beta\delta_{\mu}-\lambda)\rho^{\infty}_{\AMAT}(\lambda)d\lambda
      = M \left( 1 + \ln \frac{2\pi}{\beta} \right) - \sum_{\mu=1}^{M} \min_{\beta\delta_{\mu}} \left[\beta\delta_{\mu}\lambda_{\mu}
      - \beta\GNORM(\lambda_{\mu}) + \ln \lambda_{\mu} \right]   .
\end{align}
By cancelling the $\ln \frac{2\pi}{\beta}$ term from both sides, we obtain
\begin{align}
   -\sum_{\mu=1}^{M}\int \ln(\beta\delta_{\mu}-\lambda)\rho^{\infty}_{\AMAT}(\lambda)d\lambda
   =
   M - \sum_{\mu=1}^{M} \min_{\beta\delta_{\mu}} \left[\beta\delta_{\mu}\lambda_{\mu}
   - \beta\GNORM(\lambda_{\mu}) + \ln \lambda_{\mu} \right]   .
\end{align}
Since this is true for every $\mu$, we can solve this for any arbitrary eigenvalue $\lambda_{\mu}$.
\michael{There must be some other assumption that the $M$ term is spread out uniformly over eigenvalues?}
\charles{Yes. We assumed this earlier in A62.  Why dont you add some comments here}

Dropping the $\mu$ subscript, we have the following identity:
\begin{align}
\label{eqn:concave_id} 
 \min_{\delta} \left[\beta\delta\lambda - \beta\GNORM(\lambda) + \ln \lambda \right]
 = 1 -\int \ln(\beta\delta-\lambda)\rho^{\infty}_{\AMAT}(\lambda)d\lambda   .
\end{align}

We need to invert \ref{eqn:concave_id} in order to find $\beta\GNORM(\lambda)$.
If we choose the eigenvalues of $\mathbf{D}$ such that $\beta\delta_{\mu}>\lambda_{max}$ for all $\mu$, then 
this relation is concave and therefore invertible via a Legendre transform.  

This gives
\begin{equation}
\beta\GNORM(\lambda) = \beta\delta(\lambda) \lambda - \int \ln[\beta\delta(\lambda) - \lambda] \rho^{\infty}_{\AMAT}(\lambda) d\lambda - \ln \lambda - 1  ,
\end{equation}
where we need to  define  $ \beta\delta(\lambda)$, which (not to be confused with the Dirac delta-function), describes
the functional dependence between the eigenvalues of the source matrix $\DMAT$ and the \Student \CorrelationMatrix $\AMAT$.

$\GNORM(\lambda)$ is computed by minimizing over $\delta$, ensuring the relationship holds for the entire spectrum.
So let us take the derivative of $\beta\GNORM$ w/r.t $\lambda$. 
Term by term, this gives:
\begin{equation}
\dfrac{d}{d\lambda} \beta\delta(\lambda)\lambda = \beta\delta(\lambda) + \dfrac{d \beta\delta(\lambda)}{d\lambda} \lambda
\end{equation}
\begin{equation}
\dfrac{d}{d\lambda} \ln\lambda = \dfrac{1}{\lambda}
\end{equation}
\begin{align}
\dfrac{d}{d\lambda} \int \ln[\beta\delta(\lambda) - \lambda] \rho^{\infty}_{\AMAT}(\lambda) d\lambda
&= \int \dfrac{d}{d\lambda} \ln[\beta\delta(\lambda) - \lambda] \rho^{\infty}_{\AMAT}(\lambda) d\lambda \\ \nonumber
&= \int \dfrac{d \beta\delta(\lambda)}{d\lambda}\dfrac{ \rho^{\infty}_{\AMAT}(\lambda)}{\beta\delta(\lambda) - \lambda}d\lambda  \\ \nonumber
&=  \dfrac{d \beta\delta(\lambda)}{d\lambda}\int\dfrac{ \rho^{\infty}_{\AMAT}(\lambda)}{\beta\delta(\lambda) - \lambda}d\lambda
\end{align}

We can now simplify by  defining $\delta(\lambda)$ implicitly by the integral relation
\begin{equation}
\lambda = \int \frac{\rho^{\infty}_{\AMAT}(\lambda)}{\beta\delta(\lambda) - \lambda} d\lambda.
\end{equation}
Combining terms, this gives
\begin{equation}
\frac{d\beta\GNORM(\lambda)}{d\lambda} = \beta\delta(\lambda) - \frac{1}{\lambda},
\end{equation}

 Inverting the derivative, we obtain an integral equation for $\beta\GNORM(\lambda)$ 
\begin{equation}
\beta\GNORM(\lambda) = \int_0^\lambda \left(\beta\delta(z) - \frac{1}{z}\right) dz.
\end{equation}

Notice since  $\beta\delta(\lambda) \approx \frac{1}{\lambda}$ for $\lambda \ll 1$, then
as $\GNORM(0) = 0$ and we set the lower integrand to $0$ (for now).  Even though Tanaka’s original proof assumes an analytic continuation without branch cuts, a heavy-tailed spectrum merely shifts the lower limit of the $R$-transform integral, so the expression for $\beta\mathcal G(\lambda)$ continues to hold.

To further connect these to the \RTransform $R_{\AMAT}(z)$, we recall that the \CauchyStieltjes (or just \Cauchy See~\ref{eqn:Cz}) transform $\mathcal{C}_{\AMAT}(z)$  is given by:

\begin{equation}
\mathcal{C}_{\AMAT}(z) = \int \frac{\rho_{\AMAT}(\lambda)}{z - \lambda} d\lambda.
\end{equation}

The relationship between the \Cauchy transform and the \RTransform is  then expressed as:

\begin{equation}
\mathcal{C}_{\AMAT}\left(R_{\AMAT}(z) + \frac{1}{z}\right) = z,
\end{equation}

which implies:

\begin{equation}
\beta\GNORM(\lambda) = \int_0^{\lambda} R_{\AMAT}(z) dz.
\end{equation}

WLOG, as mentioned earlier, we can replace the lower bound on $\lambda$ from
$0\rightarrow\LambdaECSmin$ to obtain
\begin{equation}
\beta\GNORM(\lambda) = \int_{\LambdaECSmin}^{\lambda} \Re[R_{\AECS}(z]) dz.
\end{equation}
where $\LambdaECSmin$ corresponds to the start of the \EffectiveCorrelationSpace (\ECS),
and, notably,  take the \emph{Real} part of $R(z)$.
We take the Real part because the imaginary parts coming from the upper and lower lips of
the cut cancel.\footnote{
Alternatively, one might try to replace $R(z)$ by its modulus;
doing so breaks the Legendre relation $G' = R$ and spoils the
additivity of free cumulants.}









