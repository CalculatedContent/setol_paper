\subsection{Laurent Series of $R(z)[IW]$ at $z = 0$}
\label{sxn:IW}

In this section, we derive the Laurent series of $R(z)[IW]$, as given in Eqn.~\ref{eqn:IW_Lseries}.

\subsubsection*{1. Starting Expression}

The R-Transform for the Inverse Wishart (IW) model is given by:
\begin{equation}
R(z)[IW] = \frac{\kappa - \sqrt{\kappa(\kappa - 2z)}}{z}
\end{equation}
where  $\kappa > 0$  is a parameter related to the dimensions of the matrix.

\subsubsection*{2. Simplify the Square Root Term}

First, we simplify the square root term to make it suitable for expansion:
\begin{align}
\sqrt{\kappa(\kappa - 2z)} &= \kappa \sqrt{1 - \frac{2z}{\kappa}}
\end{align}

\subsubsection*{3. Expand the Square Root Using the Binomial Series}

We use the binomial series expansion for  $\sqrt{1 - x}$ , valid for  $|x| < 1$:
\begin{align}
\sqrt{1 - x} &= 1 - \frac{1}{2} x - \frac{1}{8} x^2 - \frac{1}{16} x^3 - \frac{5}{128} x^4 - \cdots
\end{align}
Setting \( x = \dfrac{2z}{\kappa} \), we have:
\begin{align}
\sqrt{1 - \frac{2z}{\kappa}} &= 1 - \frac{1}{2} \left( \frac{2z}{\kappa} \right) - \frac{1}{8} \left( \frac{2z}{\kappa} \right)^2 - \frac{1}{16} \left( \frac{2z}{\kappa} \right)^3 - \frac{5}{128} \left( \frac{2z}{\kappa} \right)^4 - \cdots \
&= 1 - \frac{z}{\kappa} - \frac{z^2}{2\kappa^2} - \frac{z^3}{2\kappa^3} - \frac{5z^4}{8\kappa^4} - \cdots
\end{align}

\subsubsection*{4. Multiply Back by  $\kappa$ }

Recall that:
\begin{align}
\sqrt{\kappa(\kappa - 2z)} &= \kappa \left( 1 - \frac{z}{\kappa} - \frac{z^2}{2\kappa^2} - \frac{z^3}{2\kappa^3} - \frac{5z^4}{8\kappa^4} - \cdots \right) \
&= \kappa - z - \frac{z^2}{2\kappa} - \frac{z^3}{2\kappa^2} - \frac{5z^4}{8\kappa^3} - \cdots
\end{align}

\subsubsection*{5. Compute  $\kappa - \sqrt{\kappa(\kappa - 2z)}$ }

Subtracting the square root from  $\kappa$:
\begin{align}
\kappa - \sqrt{\kappa(\kappa - 2z)} &= \kappa - \left( \kappa - z - \frac{z^2}{2\kappa} - \frac{z^3}{2\kappa^2} - \frac{5z^4}{8\kappa^3} - \cdots \right) \
&= z + \frac{z^2}{2\kappa} + \frac{z^3}{2\kappa^2} + \frac{5z^4}{8\kappa^3} + \cdots
\end{align}

\subsubsection*{6. Divide by  z  to Obtain  R(z)[IW] }

Now, dividing by  $z$:
\begin{align}
R(z)[IW] &= \frac{\kappa - \sqrt{\kappa(\kappa - 2z)}}{z} \
&= \frac{ z + \dfrac{z^2}{2\kappa} + \dfrac{z^3}{2\kappa^2} + \dfrac{5z^4}{8\kappa^3} + \cdots }{ z } \
&= 1 + \frac{z}{2\kappa} + \frac{z^2}{2\kappa^2} + \frac{5z^3}{8\kappa^3} + \cdots
\end{align}

\subsubsection*{7. Summary of the Laurent Series}

Thus, the Laurent series expansion of  $R(z)[IW]$  around  $z=0$ is:
\begin{align}
R(z)[IW] &= 1 + \frac{z}{2\kappa} + \frac{z^2}{2\kappa^2} + \frac{5z^3}{8\kappa^3} + \frac{7z^4}{8\kappa^4} + \cdots
\end{align}
The coefficient of the  $z^4$  term can be obtained by continuing the expansion:
\begin{align}
\text{Coefficient of } z^4: \quad \frac{7z^4}{8\kappa^4}
\end{align}

\subsubsection*{8. Implications}

The series expansion shows that  $R(z)[IW]$   is analytic at  $z=0$ , as it can be represented by a power series in  $z$  with non-negative integer powers. This confirms that the apparent singularity at $z=0$  is removable.

Each term in the series corresponds to higher-order contributions to  $R(z)[IW]$ ,
with coefficients decreasing as $\dfrac{1}{\kappa^n}$ for higher powers  n .
This indicates that for large  $\kappa$,
the higher-order terms become negligible, and  $R(z)[IW]$  is well-approximated by the leading terms.

\subsubsection*{9. Conclusion}

We have derived the Laurent series of  $R(z)[IW]$  at  $z=0$  and shown that:
\begin{align}
R(z)[IW] &= 1 + \frac{z}{2\kappa} + \frac{z^2}{2\kappa^2} + \frac{5z^3}{8\kappa^3} + \frac{7z^4}{8\kappa^4} + \cdots
\end{align}
This expansion confirms the analyticity of
$R(z)[IW]$  at $z=0$  and provides a practical tool for approximations  near this point.

\nred{Now we have a problem, because I think we want the branch cut $z>\kappa/2$ so $\kappa$ has to be very small;
  need to think about this}
