\subsection{The Inverse-Wishart Model}
\label{sxn:IW}

\nred{THIS SECTION is probably only valid for the branch cut near $z=0$}
\\
\\
In this section, we derive the Laurent series of $R(z)[IW]$ around the branch cut from $z = \kappa/2$ to $z = \infty$. This branch cut corresponds to the tail of the empirical spectral density (ESD), which is critical for evaluating the integral $G(\lambda)$. Additionally, we confirm the analyticity of $R(z)[IW]$ in this region and compute the integral.

\subsubsection{Laurent Series of $R(z)[IW]$ for the Branch Cut $z = \kappa/2$ to $z = \infty$}
\paragraph{1. Starting Expression}

The R-transform for the Inverse Wishart (IW) model is:
\begin{align}
R(z)[IW] = \frac{\kappa - \sqrt{\kappa(\kappa - 2z)}}{z},
\end{align}
where $\kappa > 0$ is a parameter related to the dimensions of the matrix.

The square-root term introduces a branch cut starting at $z = \kappa/2$, as the argument of the square root becomes zero at this point. For $z > \kappa/2$, the square root is real and positive, corresponding to the tail of the ESD.

\paragraph{2. Simplify the Square Root Term}

To facilitate the series expansion, rewrite the square root term:
\begin{align}
\sqrt{\kappa(\kappa - 2z)} = \sqrt{\kappa} \sqrt{\kappa - 2z}.
\end{align}

Express the argument of the second square root in terms of $\kappa$:
\begin{align}
\sqrt{\kappa - 2z} = \sqrt{\kappa} \sqrt{1 - \frac{2z}{\kappa}}.
\end{align}

Expand the second square root term using the binomial series for $x = \frac{2z}{\kappa}$:
\begin{align}
\sqrt{1 - x} &= 1 - \frac{1}{2}x - \frac{1}{8}x^2 - \frac{1}{16}x^3 - \frac{5}{128}x^4 - \cdots, \quad |x| < 1.
\end{align}

Substitute \( x = \frac{2z}{\kappa} \):
\begin{align}
\sqrt{\kappa - 2z} &= \sqrt{\kappa} \left( 1 - \frac{z}{\kappa} - \frac{z^2}{\kappa^2} - \frac{z^3}{2\kappa^3} - \frac{5z^4}{8\kappa^4} - \cdots \right).
\end{align}

Multiply back by $\sqrt{\kappa}$:
\begin{align}
\sqrt{\kappa(\kappa - 2z)} &= \kappa \left( 1 - \frac{z}{\kappa} - \frac{z^2}{\kappa^2} - \frac{z^3}{2\kappa^3} - \frac{5z^4}{8\kappa^4} - \cdots \right).
\end{align}

\paragraph{3. Expand the Numerator}

The numerator of $R(z)[IW]$ is:
\begin{align}
\kappa - \sqrt{\kappa(\kappa - 2z)}.
\end{align}

Substitute the expansion for $\sqrt{\kappa(\kappa - 2z)}$:
\begin{align}
\kappa - \sqrt{\kappa(\kappa - 2z)} &= \kappa - \kappa \left( 1 - \frac{z}{\kappa} - \frac{z^2}{\kappa^2} - \frac{z^3}{2\kappa^3} - \frac{5z^4}{8\kappa^4} - \cdots \right) \\
&= z + \frac{z^2}{\kappa} + \frac{z^3}{2\kappa^2} + \frac{5z^4}{8\kappa^3} + \cdots.
\end{align}

\paragraph{4. Divide by $z$ to Obtain $R(z)[IW]$}

The expression for $R(z)[IW]$ is:
\begin{align}
R(z)[IW] &= \frac{\kappa - \sqrt{\kappa(\kappa - 2z)}}{z}.
\end{align}

Substitute the expanded numerator:
\begin{align}
R(z)[IW] &= \frac{z + \frac{z^2}{\kappa} + \frac{z^3}{2\kappa^2} + \frac{5z^4}{8\kappa^3} + \cdots}{z}.
\end{align}

Simplify:
\begin{align}
R(z)[IW] &= 1 + \frac{z}{\kappa} + \frac{z^2}{2\kappa^2} + \frac{5z^3}{8\kappa^3} + \cdots.
\end{align}

\paragraph{5. Integral for $G(\lambda)$}

The integral for $G(\lambda)$ is defined as:
\begin{align}
G(\lambda) = \int_{\kappa/2}^\lambda R(z)[IW] \, dz.
\end{align}

Substitute the Laurent series for $R(z)[IW]$:
\begin{align}
G(\lambda) &= \int_{\kappa/2}^\lambda \left(1 + \frac{z}{\kappa} + \frac{z^2}{2\kappa^2} + \frac{5z^3}{8\kappa^3} + \cdots\right) dz.
\end{align}

Evaluate term by term:
\begin{align}
G(\lambda) &= (\lambda - \kappa/2) + \frac{\lambda^2 - (\kappa/2)^2}{2\kappa} + \frac{\lambda^3 - (\kappa/2)^3}{6\kappa^2} + \frac{5}{32\kappa^3} \left(\lambda^4 - \left(\frac{\kappa}{2}\right)^4\right) + \cdots.
\end{align}


\subsubsection{ Comparison of Laurent Series and Direct Integration Results}

We now compare the result obtained from the Laurent series expansion of $R(z)[IW]$ with the result obtained from the direct integration of $R(z)$ along the branch cut. This comparison demonstrates the consistency of our derivations.

\paragraph{1. Result from the Laurent Series Expansion}

Using the Laurent series expansion for $R(z)[IW]$:
\begin{align}
R(z)[IW] = 1 + \frac{z}{\kappa} + \frac{z^2}{2\kappa^2} + \frac{5z^3}{8\kappa^3} + \cdots,
\end{align}
the integral $G(\lambda)$ along the branch cut from $z = \kappa/2$ to $z = \lambda$ was computed term by term:
\begin{align}
G(\lambda) &= (\lambda - \kappa/2) + \frac{\lambda^2 - (\kappa/2)^2}{2\kappa} + \frac{\lambda^3 - (\kappa/2)^3}{6\kappa^2} + \cdots.
\end{align}

For the first few terms, this result simplifies to:
\begin{align}
G(\lambda) = \lambda - \frac{\kappa}{2} + \frac{\lambda^2}{4\kappa} - \frac{\kappa}{16} + \cdots.
\end{align}

\paragraph{2. Result from Direct Integration}

From the direct integration of $R(z)[IW]$, split into two terms:
\begin{align}
R(z)[IW] = \frac{\kappa}{z} - \frac{\sqrt{\kappa(\kappa - 2z)}}{z},
\end{align}
we obtained:
\begin{align}
G(\lambda) &= (\lambda - \kappa/2) + \frac{\lambda^2}{4\kappa} - \frac{\kappa}{16} + \cdots.
\end{align}

\paragraph{3. Consistency of Results}

Both the Laurent series expansion and the direct integration give identical results for $G(\lambda)$:
\begin{align}
G(\lambda) = \lambda - \frac{\kappa}{2} + \frac{\lambda^2}{4\kappa} - \frac{\kappa}{16} + \cdots.
\end{align}

This agreement verifies that the Laurent series expansion correctly represents $R(z)[IW]$ near the branch cut, and the integral computed term by term using the series is consistent with the direct integration.

\paragraph{4. Implications for Analyticity}

This consistency also confirms that $R(z)[IW]$ is analytic along the branch cut for $z > \kappa/2$ and that the branch cut correctly captures the tail behavior of the ESD. The series and direct integration are equivalent, providing confidence in both approaches for computing $G(\lambda)$.

\paragraph{5. Final Expression for $G(\lambda)$}

The final expression for $G(\lambda)$, valid for $z > \kappa/2$, is:
\begin{align}
G(\lambda) = \lambda - \frac{\kappa}{2} + \frac{\lambda^2}{4\kappa} - \frac{\kappa}{16} + \cdots.
\end{align}
This result can be used to describe the contribution of the tail of the ESD to the R-transform.
