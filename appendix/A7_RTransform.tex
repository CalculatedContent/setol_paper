%--------------------------------------------------------------------
% R–transform existence for power-law spectra
%--------------------------------------------------------------------
\subsection{Existence of the Free $R$–Transform for Power-Law Spectra}
\label{sxn:RTransformExists}

In this subsection, we examine when the \RTransform $R(z)$ does and does not exist
for power law tails.

%--------------------------------------------------------------------
\subsubsection{Analyticity criterion}
\label{sxn:RTransformExists:criterion}

Recall the Cauchy–Stieltjes transform defined in Eq.~\eqref{eqn:Cz} of
Sec.~\ref{sxn:r_transforms:elementary_rmt},
\begin{equation}
\label{eqn:RTransformExists_Gdef}
G_\mu(z)=\int_{\mathbb R}\frac{\rho(\lambda)}{z-\lambda}\,
         \mathrm d\lambda,
\qquad
z\in\mathbb C\setminus\operatorname{supp}\mu .
\end{equation}
If $G_\mu(z)$ is \emph{holomorphic} at $z=\infty$ one may invert the map
$w=G_\mu(z)$ in a neighbourhood of $w=0$ and define
\begin{equation}
\label{eqn:RTransformExists_Rdef}
R_\mu(z)=G_\mu^{-1}(z)-\frac{1}{w}.
\end{equation}
Obstructions to the existence of $R_\mu$ therefore coincide with
non-analytic terms in the Laurent series of $G_\mu$ about $z=\infty$.
These can be easily removed, however, by considering that $\rho(\lambda)$ is always strictly bounded from above by $\LambdaMax$. 
%--------------------------------------------------------------------
\subsubsection{Model and notation}
\label{sxn:RTransformExists:model}

We first model a \emph{bare} power-law tail regularised only by a hard
infrared cut-off $\LambdaMin>0$:
\begin{equation}
\label{eqn:RTransformExists_density_bare}
\rho_{\alpha}(\lambda)=
(\alpha-1)\,\LambdaMin^{\alpha-1}\,
\lambda^{-\alpha},
\qquad
\lambda\ge\LambdaMin,
\qquad
\alpha\in\{2,3,4\}.
\end{equation}
Normalization is immediate:
\begin{equation}
\label{eqn:RTransformExists_norm}
\int_{\LambdaMin}^{\infty}\rho_{\alpha}(\lambda)\,
      \mathrm d\lambda = 1.
\end{equation}

The choice $\alpha\in\{2,3,4\}$ mirrors the Heavy-Tailed exponents most
frequently observed in neural-network weight and Hessian spectra.

%--------------------------------------------------------------------
\subsubsection{Stieltjes (Green’s) transform}
\label{sxn:RTransformExists:greens}

For $z\in\mathbb C\setminus[0,\infty)$
\begin{equation}
\label{eqn:RTransformExists_Gbare_def}
G_{\alpha}(z)=
\int_{\LambdaMin}^{\infty}
\frac{\rho_{\alpha}(\lambda)}{z-\lambda}\,
\mathrm d\lambda .
\end{equation}
Carrying out the integration yields

\begin{equation}
\label{eqn:RTransformExists_G2}
G_{(2)}(z)=
\frac{1}{z}+
\frac{\LambdaMin\,\ln\!\bigl(1-\frac{z}{\LambdaMin}\bigr)}{z^{2}},
\end{equation}

\begin{equation}
\label{eqn:RTransformExists_G3}
G_{(3)}(z)=
\frac{1}{z}+
\frac{2\LambdaMin}{z^{2}}+
\frac{2\LambdaMin^{2}\,
      \ln\!\bigl(1-\frac{z}{\LambdaMin}\bigr)}{z^{3}},
\end{equation}

\begin{equation}
\label{eqn:RTransformExists_G4}
G_{(4)}(z)=
\frac{1}{z}+
\frac{3\LambdaMin}{2\,z^{2}}+
\frac{3\LambdaMin^{2}}{z^{3}}+
\frac{3\LambdaMin^{3}\,
      \ln\!\bigl(1-\frac{z}{\LambdaMin}\bigr)}{z^{4}}.
\end{equation}

The logarithmic pieces carry the entire heavy-tail fingerprint; their
placement in the expansion dictates whether $R(z)$ will be available.

%--------------------------------------------------------------------
\subsubsection{Moments and free cumulants}
\label{sxn:RTransformExists:moments}

Algebraic moments exist only up to order $\alpha-2$:
\begin{equation}
\label{eqn:RTransformExists_mk}
m_{k}=
\int_{\lambda_0}^{\infty}\lambda^{k}\rho_{\alpha}(\lambda)\,
      \mathrm d\lambda
=
\frac{\alpha-1}{\alpha-k-1}\,
\LambdaMin^{k},
\qquad
k<\alpha-1 .
\end{equation}
Hence
\begin{equation}
\label{eqn:RTransformExists_m1m2}
m_{1}=\frac{\alpha-1}{\alpha-2}\,\LambdaMin,
\qquad
m_{2}=\frac{\alpha-1}{\alpha-3}\,\LambdaMin^{2}
\quad$alpha>3).
\end{equation}
The first two free cumulants are $\kappa_{1}=m_{1}$ and
$\kappa_{2}=m_{2}-m_{1}^{2}$.

%--------------------------------------------------------------------
\subsubsection{R-transform for the bare tail}
\label{sxn:RTransformExists:Rbare}

Define $w=G_{\alpha}(z)$ and solve locally for $z(z)$.  
Using the finite cumulants one obtains

\begin{itemize}
\item \textbf{$\alpha=2$:}  
  the logarithm appears at order $z^{-2}$; $G(z)$ is \emph{not} analytic at
  infinity and the inversion fails.  
  \emph{Conclusion: no $R$–transform.  Truly $1/\lambda^{2}$ tails are
  outside the remit of free addition.}

\item \textbf{$\alpha=3$:}
  \begin{equation}
  \label{eqn:RTransformExists_R3_final}
  R_{(3)}(z)=2\,\LambdaMin
  \quad$text{constant}).
  \end{equation}
  Only a zeroth-order free cumulant survives, but that is \emph{enough}
  for free convolution.

\item \textbf{$\alpha=4$:}
  \begin{equation}
  \label{eqn:RTransformExists_R4_final}
  R_{(4)}(z)=\frac{3}{2}\,\LambdaMin+
             \frac{3}{4}\,\LambdaMin^{2}\,w.
  \end{equation}
  Here the series truncates after the linear term; higher cumulants
  diverge.
\end{itemize}

These three cases show explicitly how \emph{incremental} changes in the
tail exponent alter the analytic status of $G(z)$ and hence of $R(z)$. Since the  higher-order free cumulants diverge, the \RTransform cannot be a finite polynomial. It must either be an infinite series (where the terms beyond a certain point don't vanish) or, more strongly, exhibit non-analytic behavior (like the logarithmic terms present in $G(z)$) because its Taylor series coefficients (the cumulants) become infinite. Fortunately, $R(z)$ can be defined if we ensure that $\rho(\lambda)$ has compact support.
%--------------------------------------------------------------------
\subsubsection{Truncated $\alpha=2$ power law}
\label{sxn:RTransformExists:trunc2}

Introduce a cut-off $\LambdaMax>\LambdaMin$ and set
\begin{equation}
\label{eqn:RTransformExists_density_trunc2}
\rho_{\mathrm{tr}}(\lambda)=
C\,\lambda^{-2},
\qquad
\LambdaMin\le\lambda\le\LambdaMax,
\qquad
C=\frac{1}{\LambdaMin^{-1}-\LambdaMax^{-1}} .
\end{equation}
Exact integration gives
\begin{equation}
\label{eqn:RTransformExists_Gtrunc_exact}
G_{\mathrm{tr}}(z)=
C\Biggl[
\frac{\log\LambdaMax-\log(\lambda-z)
      -\log\LambdaMin+\log\LambdaMin-z)}{z^{2}}
-\frac{1}{\LambdaMax\,z}+\frac{1}{\LambdaMin\,z}
\Biggr],
\end{equation}
valid for $z\in\mathbb C\setminus[\LambdaMin,\LambdaMax]$.

\paragraph{Expansion at $z=\infty$.}
Using
$\log(\lambda-z)=\log z+\log\!\bigl(1-\LambdaMax/z\bigr)$ and
$\log(\LambdaMin-z)=\log z+\log\!\bigl(1-\LambdaMin/z\bigr)$,
the two $\log z$ terms cancel and we find the regular series
\begin{equation}
\label{eqn:RTransformExists_Gtrunc_Laurent}
G_{\mathrm{tr}}(z)=
\frac{1}{z}+\frac{m_{1}}{z^{2}}+\frac{m_{2}}{z^{3}}+\dots,
\qquad
z\to\infty,
\end{equation}
with finite moments of all orders.  Consequently
\begin{equation}
\label{eqn:RTransformExists_Rtrunc2}
R_{\mathrm{tr}}(z)=G_{\mathrm{tr}}^{-1}(z)-\frac{1}{w}
\end{equation}
is analytic for $|w|$ small.

\textbf{Interpretation.}  
A truncation at any physically reasonable $\LambdaMax$—for instance the
largest finite eigenvalue observed in a data set—instantly restores full
analyticity at infinity.  
From the point of view of free probability the system now behaves as if it
had \emph{all} moments, even though the raw tail is still $1/\lambda^{2}$
within $[\LambdaMin,\LambdaMax]$.

%--------------------------------------------------------------------
\subsubsection{Key points and implications for \SETOL}
\label{sxn:RTransformExists:summary}

\begin{enumerate}
\item Cutting the tail at $\LambdaMax=\LambdaECSmax$ \emph{removes} the non-analytic
  $\log z/z^{2}$ obstruction and turns the free-probability machinery back
  on.
\item Any model density $\rho(\lambda)$ with compact support has $G(z)$ analytic at
  $z=\infty$; hence its $R$–transform equals the usual free-cumulant
  series and is available for algebraic manipulation.
\item In all theoretical derivations and numerical experiments in
  \SETOL\ we therefore \textbf{model empirical spectra as effectively as \emph{truncated}
  power laws} (i.e with finite bounds, not necessarily exponentially truncated).
  This choice is both empirically justified (no spectrum is truly
  infinite) and mathematically essential: it guarantees that $R(z)$
  \emph{always} exists.
\end{enumerate}


\subsubsection{Explicit $R$–transforms for the truncated tail}
\label{sxn:RTransformExists:explicit}

We can also provide exact expressions for the leading terms (free cumulants) in $R(z)$ for $alpha=2,3,4$.
The compact support $[\LambdaMin,\LambdaMax]=$[\LambdaECSmin,\LambdaECSmax]$ guarantees that every
algebraic moment (and free cumulant) is finite.   Hence the Voiculescu series 

\begin{equation}
R_{\alpha}^{\mathrm{tr}}(z)=
\sum_{n=1}^{\infty}\kappa_{n}\,z^{\,n-1},
\qquad |z|\text{ small},
\end{equation}

\noindent converges for every $\alpha>1$.  
Below we list the free cumulants $\kappa_{1}$, $\kappa_{2}$ and the resulting
\(R(z)$ for the three exponents most relevant to \SETOL.
Higher cumulants follow from the recursion in
Eqs.~\eqref{eqn:RTransformExists_mk}–\eqref{eqn:RTransformExists_m1m2} and
need not be written out.

---

\paragraph{$\alpha = 2$.}

\begin{equation}
C_{2}=\frac{\LambdaMin\LambdaMax}{\LambdaMax-\LambdaMin},
\end{equation}

\begin{equation}
\kappa_{1}=C_{2}\,\log\!\left\frac{\LambdaMax}{\LambdaMin},
\end{equation}

\begin{equation}
\kappa_{2}=C_{2}\,\LambdaMax-\LambdaMin)-\kappa_{1}^{2},
\end{equation}

\begin{equation}
R_{(2)}^{\mathrm{tr}}(z)=
\kappa_{1}+\kappa_{2}\,z+\mathcal{O}\!\left(z^{2}\right).
\end{equation}


\paragraph{$\alpha = 3$.}

\begin{equation}
C_{3}=\frac{2\,\LambdaMin^{2}\LambdaMax^{2}}{\LambdaMax^{2}-\LambdaMin^{2}},
\end{equation}

\begin{equation}
\kappa_{1}=\frac{2\,\LambdaMax\LambdaMin}{\LambdaMax+\LambdaMin},
\end{equation}

\begin{equation}
\kappa_{2}=C_{3}\,\log\!\left\frac{\LambdaMax}{\LambdaMin}
-\kappa_{1}^{2},
\end{equation}

\begin{equation}
R_{(3)}^{\mathrm{tr}}(z)=
\kappa_{1}+\kappa_{2}\,z+\mathcal{O}\!\left(z^{2}\right).
\end{equation}


\paragraph{$\alpha = 4$.}

\begin{equation}
C_{4}=\frac{3\,\LambdaMin^{3}\LambdaMax^{3}}{\LambdaMax^{3}-\LambdaMin^{3}},
\end{equation}

\begin{equation}
\kappa_{1}=
\frac{3\,\LambdaMax\LambdaMin\,(\LambdaMax^{2}-\LambdaMin^{2})}
     {2\,(\LambdaMax^{3}-\LambdaMin^{3})},
\end{equation}

\begin{equation}
\kappa_{2}=C_{4}\,
\left(\frac{1}{\LambdaMin}-\frac{1}{\LambdaMax}\right)
-\kappa_{1}^{2},
\end{equation}

\begin{equation}
R_{(4)}^{\mathrm{tr}}(z)=
\kappa_{1}+\kappa_{2}\,z+\mathcal{O}\!\left(z^{2}\right).
\end{equation}

---

\textbf{Interpretation.}  
$\kappa_{1}$ fixes the mean scale of the heavy tail;
$\kappa_{2}$ sets its leading spread.  
Because both depend only on the empirical cut‑offs
\(\LambdaMin$ and \(\LambdaMax$, the two–term truncation already delivers
an accurate $R$–transform for free‑probability manipulations inside the
\SETOL\ framework.