\subsubsection{Step 2: The Saddle Point Approximation (SPA): Explicitly forming the Large Deviation Principle (LDP)}
\label{sxn:tanaka_step2}
We now evaluate $\EZDA$ in \EQN~\ref{eqn:ZD1} as $\EZDATWO$ 
to  establish \EQN~\ref{eqn:ZD_step2}, \charles{in Steps $2.1-2.6$}.

Using the LDP (and following similar approaches in spin glass theory \cite{PP95}),
below we will show that we can write the expected value of $\ZD$ 
in terms of $d\mu(\mathbf{X})$ now (which is equivalent to $d\mu(\AMAT_{1})$)
and in the large-$N$ approximation, as
\begin{equation}
  \label{eqn:LDP}
 \lim_{N \gg 1} \EZDAONE=
  \int\exp\left(\beta N \Trace{\GNORM(\mathbf{X})}-NI(\mathbf{X})+o(N)\right)d\mu(\mathbf{X})
\end{equation}
where $I(\mathbf{X})$ is  \RateFunction, defined below,
and  $\GNORM(\mathbf{X})$ is what we are eventually solving for.

We first introduce a new change of measure, 
 $d\mu(\mathbf{W})\rightarrow d\mu(\mathbf{X})$.
Then, we show this lets us express $\EZDAONE$ as $\EZDX$ and to express it using the LDP.
Next, we apply a SPA to solve for $\GMAX$
Importantly, we also show how to incorporate the inverse-Temperature $\beta$,
which is new.
\\

\paragraph{Step 2.1}
To define the transformation $d\mu(\mathbf{W})\rightarrow d\mu(\mathbf{X})$,  where (recall) $\mathbf{X}=\frac{1}{N}\mathbf{W}^{\top}\mathbf{W}$,
we use the (again) the integral representation of the Dirac delta-function $\delta(x)$:
\begin{equation}
  \label{eqn:dirac}
  \delta(x):=\frac{1}{2\pi}\int_{-\infty}^{\infty} e^{i\hat{x}x} d\hat{x}.
\end{equation}
%
This lets us express the transformation of measure $d\mu(\mathbf{W})\rightarrow d\mu(\mathbf{X})$
(approximately) as
\begin{align}
\nonumber
  d\mu(\mathbf{W}) &:= \delta(\frac{1}{2}\Trace{N\mathbf{X}-\mathbf{W}^{\top}\mathbf{W}}) d\mu(\mathbf{X}) \\ 
  &= \frac{1}{2\pi}\int_{-\infty}^{\infty} e^{i\frac{1}{2}
    \Trace{\hat{X}(N\mathbf{X}-\mathbf{W}^{\top}\mathbf{W})}
  }
  d\mu(\mathbf{X})d\hat{X}  ,
\end{align}
where $\hat{X}$ is a scalar (or really matrix of scalars),
and we have a $1/2$ term for mathematical consistency below.
\footnote{The full change of measure would require a delta function constraint
for each matrix element $X_{i,j}$, i.e., 
$\delta\left(\frac{1}{2}N\left(X_{i,j}-[\mathbf{W}^{\top}\mathbf{W}]_{i,j}\right)\right)$.
Here, we assume the Trace constraint is sufficient for our level of rigor.
}



\paragraph{Step 2.2}
Next, we take a Wick Rotation, $i\XHAT\rightarrow \XHAT$, so that the terms under the integral are all real (not complex), giving:
\begin{align}
  \label{eqn:dmuX}
  d\mu(\mathbf{W}) &= \frac{1}{2\pi}\int_{-i\infty}^{i\infty} e^{\frac{1}{2}\Trace{\hat{X}(N\mathbf{X}-\mathbf{W}^{\top}\mathbf{W})}} d\mu(\mathbf{X})d\hat{X}  .
\end{align}

\paragraph{Step 2.3}
We now insert \ref{eqn:dmuX} into \ref{eqn:ZD0}, which lets
express $\ZD$ as an integral over the \Teacher Correlation matrices
\begin{align}
  \nonumber
  \ZD&=
  \frac{1}{2\pi} \int_{\XMAT}  \int_{-i\infty}^{i\infty}
  e^{N\frac{\beta}{2} \Trace{\GFANCY}+\Trace{\hat{X}\mathbf{X}} }
  e^{-\frac{1}{2}\Trace{\hat{X}\mathbf{W}^{\top}\mathbf{W}}}
  e^{\frac{\beta}{2}\Trace{\mathbf{W}\mathbf{D}\mathbf{W}^{\top}}}
  d\hat{X}  
  d\mu(\mathbf{\XMAT}) \\ 
  \nonumber
  &=
  \frac{1}{2\pi} \int_{\XMAT}  \int_{-i\infty}^{i\infty}
  e^{N\frac{\beta}{2}\Trace{\GFANCY}+ \frac{1}{2}\Trace{\hat{X}\mathbf{X}}}
  e^{-\frac{1}{2}\Trace{\mathbf{W}\hat{X}\mathbf{W}^{\top}}+
  \frac{\beta}{2}\Trace{\mathbf{W}\mathbf{D}\mathbf{W}^{\top}}}
  d\hat{X}  
  d\mu(\mathbf{\XMAT}) \\ 
  \label{eqn:ZD3}
    &=
  \frac{1}{2\pi} \int_{\XMAT}  \int_{-i\infty}^{i\infty}
  e^{N\frac{\beta}{2}\Trace{\GFANCY}+
  \frac{1}{2}\Trace{\hat{X}\mathbf{X}}}
  e^{\frac{1}{2}\Trace{\mathbf{W}(\beta\mathbf{D}-\hat{X})\mathbf{W}^{\top}}}
  d\hat{X}  
  d\mu(\mathbf{\XMAT})  .
\end{align}

\nred{ABOVE may be missing a $1/2$}
\paragraph{Step 2.4}
We can now rearrange terms to make this expression look like the \EQN~\ref{eqn:LDP}
%Let us write
%express \EQN~\ref{eqn:ZD3} as the expected value $\EZDA$ in terms of 
%\michael{Huh?}
%\charles{SOME EXPLANATIONS NEEDED: }
%\begin{equation}
%  \EZDA=\int\exp\left(\beta N\Trace{\GNORM}-NI(\mathbf{X})+o(N)\right)d\mu(\mathbf{X})
%\end{equation}


In Large Deviations Theory, the \RateFunction is defined by the Legendre Transform,
\begin{equation}
\label{eqn:rate-fun}
    \mathcal{I}(\mathbf{X})=\underset{\mathbf{\check{X}}}{\sup}
    \left[Tr\dfrac{1}{2}\mathbf{{X}}^{\top}\mathbf{\check{X}}-\ln\mathbb{M}(\mathbf{\check{X}})\right]  ,
\end{equation}
where $\mathbb{M}(\mathbf{\check{X}})$ is the \MomentGeneratingFunction,
$\ln\mathbb{M}(\mathbf{\check{X}})$, is the \CumulantGeneratingFunction, and
and $\mathbf{\check{X}}$ is a (matrix of) \emph{Lagrange Multiplier}(s).  $\mathbb{M}(\mathbf{\check{X}})$ is defined in terms of the (unnormalized) density $p(\mathbf{x})$ as
\begin{equation}
\label{eqn:rate_function}
   \mathbb{M}(\mathbf{\check{X}})=\exp\left(\frac{1}{2}\mathbf{x}^{\top}\mathbf{\check{X}}\mathbf{x}\right)  ,
  p(\mathbf{x})d\mathbf{x}
\end{equation}
which, in term, is defined in terms of the source matrix $\mathbf{D}$,
\begin{equation}
\label{eqn:X_density}
p(\mathbf{x})=\exp\left(-\tfrac{1}{2}\mathbf{x}^{\top}\beta\mathbf{D}\mathbf{x}\right)  .
\end{equation}
%
The moment generating function $ \mathbb{M}(\mathbf{\check{X}})$ is then given by
%
\begin{equation}
\mathbb{M}(\mathbf{\check{X}}) 
   = \int \exp\left(-\frac{1}{2}\mathbf{x}^T(\beta\mathbf{D} - \mathbf{\check{X}})\mathbf{x}\right) d\mathbf{x} 
   = (2\pi)^{\frac{M}{2}} \Det{\beta\mathbf{D} - \mathbf{\check{X}}}^{-\frac{1}{2}}  .
\end{equation}

\paragraph{Step 2.5}
The \SaddlePointApproximation (SPA) can be used to solve for $\mathcal{I}(\mathbf{\check{X}})$ 
by solving for the stationary conditions
\begin{equation}
  \frac{\partial}{\partial \mathbf{\check{X}}} I(\mathbf{X},\mathbf{\check{X}}) = 0  .
\end{equation}
%
First, let us compute $ \ln \mathbb{M}(\mathbf{\check{X}}) $ as:
%
\begin{equation}
\ln \mathbb{M}(\mathbf{\check{X}}) = \frac{M}{2} \ln(2\pi) - \frac{1}{2} \ln \Det{\beta\mathbf{D} - \mathbf{\check{X}}}.
\end{equation}
\michael{Make det notation consistent throughout.}
\charles{OK lets double check that everywhere;}
%
Substituting this into the expression for the Legendre transform, we obtain:
%
\begin{equation}
I(\mathbf{X},\mathbf{\check{X}}) 
   = \sup_{\mathbf{\check{X}}} \left[\frac{1}{2} \Trace{\mathbf{X} \mathbf{\check{X}}} - \frac{M}{2} \ln(2\pi) + \frac{1}{2} \ln \Det{\beta\mathbf{D} - \mathbf{\check{X}}} \right].
\end{equation}
%
The supremum of this expression is attained at the value of $\mathbf{\check{X}}$ that satisfies:
%
\begin{equation}
\frac{\partial}{\partial \mathbf{\check{X}}} \left[\frac{1}{2} \Trace{\mathbf{X} \mathbf{\check{X}}} + \frac{1}{2} \ln \Det{\beta\mathbf{D} - \mathbf{\check{X}}} \right] = 0.
\end{equation}
%
Taking the derivative, we obtain
%
\begin{equation}
\frac{1}{2} \mathbf{X} + \frac{1}{2} (\beta\mathbf{D} - \mathbf{\check{X}})^{-1} = 0,
\end{equation}
%
which simplifies to:
%
\begin{equation}
\mathbf{X} = (\beta\mathbf{D} - \mathbf{\check{X}})^{-1} \quad \Rightarrow \quad \mathbf{\check{X}} = \beta\mathbf{D} - \mathbf{X}^{-1}.
\end{equation}
%
Substituting $ \mathbf{\check{X}} = \beta\mathbf{D} - \mathbf{X}^{-1} $ back into the expression for $ I(\mathbf{X})$, we obtain:
%
\begin{equation}
I(\mathbf{X}) = \frac{1}{2} \left[\Trace{\mathbf{X} (\beta\mathbf{D} - \mathbf{X}^{-1})} - \frac{M}{2} \ln(2\pi) + \frac{1}{2} \ln \Det{\mathbf{X}^{-1}}\right].
\end{equation}
%
%
\begin{equation}
\Trace{\mathbf{X}\beta\mathbf{D} - \mathbf{I}} = \Trace{\mathbf{X}\beta\mathbf{D}} - N,
\end{equation}
\begin{equation}
\ln \Det{\mathbf{X}^{-1}} = -\ln \Det{\mathbf{X}},
\end{equation}
we get:
%
\begin{equation}
I(\mathbf{X}) = \frac{1}{2} \left[\Trace{\mathbf{X}\beta\mathbf{D}} - \ln \Det{\mathbf{X}} - M - M\ln(2\pi)\right].
\end{equation}

Finally, we express $I(\mathbf{X})$ in the form:

\begin{equation}
I(\mathbf{X}) = \frac{1}{2} \left[ -M(1 + \ln(2\pi)) + \Trace{\mathbf{X}\beta\mathbf{D}} - \ln \Det{\mathbf{X}} \right].
\end{equation}

\nred{THE DERIVATION ABOVE FOR $\GFANCY$ may have some TYPOS: CHECK}
\paragraph{Step 2.6}
\begin{equation}
  \beta\GFANCY=\mathbb{M}(1+\ln 2\pi)+\beta\Trace{\GNORM(\mathbf{X})} - \Trace{\mathbf{X}\beta\mathbf{D}} +  \ln \Det{\mathbf{X}}.
\end{equation}

We restrict our solution to those where $\XMAT$ and $\beta\mathbf{D}$ can be diagonalized simultaneously.
In particular, this lets us write
\begin{equation}
\Trace{\mathbf{X}\beta\mathbf{D}} = \sum_{\mu=1}^{M}\beta\delta_{\mu}\lambda_{\mu}   ,
\end{equation}
where $\beta\delta_{\mu}$ and $\lambda_{\mu}$ denote the eigenvalues of $\XMAT$ and $\beta\mathbf{D}$, resp.

We can now write the maximum value of $\GNORM$, $\GMAX$, as
\begin{equation}
\label{eqn:gmax_final}
\beta\GMAX = M \left( 1 + \ln \frac{2\pi}{\beta} \right) - \sum_{\mu=1}^{M} \min_{\beta\delta_{\mu}} \left[\beta\delta_{\mu}\lambda_{\mu}
- \beta\GNORM(\lambda_{\mu}) + \ln \lambda_{\mu} \right]   .
\end{equation}

