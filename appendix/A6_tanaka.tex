\subsection{Tanaka's Result}
\label{sxn:tanaka}

In this section, we will rederive the result by Tanaka~\cite{Tanaka2007,Tanaka2008} that we use in our main derivation,
and, importantly, explain how to address the missing Temperature term.
For completeness, we restate it here using the notation of the main text:
%Tanaka states the as have the following HCIZ (Harish-Chandra/Itzykson-Zuber) integral~\cite{BP2001}
\begin{equation}
  \label{eqn:hciz}
  \lim_{N \gg 1} \frac{1}{N} \ln 
\underbrace{
  \Expected[\mathbf{\AMAT}]{
    \exp\left(\frac{\ND\beta}{2}
    \Trace{\mathbf{W}^{\top} \AMAT_N \mathbf{W}}
    \right)
  }
 }_{\text{HCIZ Integral}}
  = \frac{\ND\beta}{2} \sum_{i=1}^{M} \GNORM(\lambda_{i})
\end{equation}
where 
$\mathbf{W}$ is the $N\times M$ \Teacher weight matrix, 
$\AMAT=\AMAT_N$ is the $N\times N$ \Student (correlation) matrix, 
but $\ND\beta$ is now the inverse-Temperature (because we are working with real matrices),
and we have added a $\tfrac{1}{2}$ (which will be clear later).
$\GNORM(\lambda)$ is a complex analytic function of the eigenvalues $\lambda$ of (the \Teacher Correlation matrix) $\XMAT$, 
whose functional form will depend on the structure of the limiting form of (the \Student) ESD $\rho_{\AMAT}^{\infty}(\lambda)$.
We may also write it as $\GNORM(\XMAT)$ below.
We call it perhaps somewhat imprecisely a  \emph{\GEN} because the final results for the \LayerQuality  $\Q$ will take the form of a \emph{Tail norm} in many cases.
\michael{I think calling $\GNORM(\lambda)$ a Norm \GeneratingFunction will be confusing, unless the thing it generates is precisely a norm, which I think it is not.}
\michael{Also, I think the mathbb is confusing on the $G$ here, but especially on the $Z$ below, since it makes it look like integers, real numbers, etc., and not a function.
}
\charles{Its hard to find different but similar notation for new things.  Suggestions ?}

\michael{
We refer to Tanaka in at least three sets of different letters: 
$\mathbf{W}$ in \ref{eqn:hciz}; 
$\AMAT$ and $\BMAT$ around \ref{eqn:izgin_def,eqn:hciz_tanaka}; and
$\TMAT$ in \ref{eqn:tanaka_result}.
But this equation here is also referred to as HCIZ, and we refer to HCIZ with different sets of letters.
I am in the process of cleaning this up.
}

To apply this result, we note that
while the term $\ND\beta$ is just a constant in~\cite{Tanaka2008}
($1$ or $2$, depending on whether the random matrix is real or complex),
it is not actually inverse Temperature $\ND\beta=\tfrac{1}{T}$ in the original derivation.
Still, we seek a final result that is linear in $\ND\beta=\tfrac{1}{T}$,
so that we can easily evaluate $\QT$ in the high-T limit, i.e.
$\QT
=\tfrac{\partial}{\partial N}\tfrac{1}{\ND\beta}\IZGINF
=\tfrac{\partial}{\partial \ND\beta}\tfrac{1}{N}\IZGINF$
(see \ref{eqn:IZG_QT}).
We can introduce $\ND\beta=\tfrac{1}{T}$ by
simply changing the scale of $\AMAT_N$ since the final result is a sum of \RTransforms, which by definition
are linear, i.e., $\GNORM(\ND\beta\lambda)=\ND\beta \GNORM(\lambda)$, however, it is instructive
to rederive the final result, with $\ND\beta$ explicitly included.
\michaeladdressed{@charles: what is the point of this comment?  Is it just that if temperature is constant, then the derivative is simple, since we dont need the chain rule.  Also, this comment is about applying this result; do we need it for the derivation?}

%%Also, notice there is an extra $\tfrac{1}{2}$ term in the exponential; this term will factor out later, but it is added here to make it easier to see how to apply the \emph{Large Deviation Principle} (LDP); this not essential but is nice to have
%%\nred{I may remove this}
%%\michael{I would say include the 1/2, for expositional clarity; it helped me.}

\paragraph{Notation.}

We start by rewriting the Tanaka result, \EQN~(\ref{eqn:hciz}),
in our notation for the expected value $\Expected[\AMAT]{\cdots}$ operator, as follows:
\begin{equation}
\label{eqn:hciz2}
  \tfrac{1}{2\red{N}}\IZGINF = 
  \red{\frac{1}{N}}\lim_{N\gg 1} \ln \underbrace{ \int d\mu(\mathbf{\AMAT})\left[\exp\left(\frac{\ND\beta}{2}\Trace{\mathbf{W}^{\top}\AMATN\mathbf{W}}\right)\right] }_{\mbox{HCIZ Integral}} 
  = \ND\beta \cancel{N}\tfrac{1}{2}\sum_{i=1}^{M}\GNORM(\lambda_{i})   .
\end{equation}
where we have added a $\tfrac{1}{2}$ for technical convenience (to make the connection with the LDP, below).
If we denote the internal HCIZ integral as 
%%MM%% $\HCIZ$,  which
%%MM%% denoting the \PartitionFunction for our matrix generalization of the ST model.
%%MM%% This gives
\begin{equation}
\label{eqn:hciz_def}
  \HCIZ := \int d\mu(\mathbf{\AMAT})\left[\exp\left(\frac{\ND\beta}{2}\Trace{\mathbf{W}^{\top}\AMATN\mathbf{W}}\right)\right]  ,
 \end{equation}
then it holds that \nred{I think we need $1/N$ below and we need to justify it}
\begin{equation}
  \label{eqn:hciz_def2}
  \IZG :=  \red{\frac{1}{N}}\ln\HCIZ  ,
\end{equation}
from which it follows that %%MM%% or, equivalently,
\begin{equation}
\label{eqn:hciz_def3}
  \IZGINF := \red{\frac{1}{N}}\lim_{N \gg 1} \ln\HCIZ  .
\end{equation}

The SPA approximates the \PartitionFunction $\HCIZ$, which is now an HCIZ integral,  by its peak value.
For this, $\GNORM(\lambda)$ itself must either not explicitly depend on $N$ and/or at least not grow faster than $N$.
%\nred{For that reason, I think we need to normalize the eigenvalue as $\lambda/N$, or rather $\lambda/M$, or maybe $\lambda/\MECS$, as with the normalization constrain on $\XECS$.}

The trick here is we can choose an \RTransform of $\mathbf{\AMAT}$
that is a simple analytic expression based on the observed
empirical spectral density (ESD) of the $\mathbf{X}$.
And this can readily be done for the ESDs for a wide range of layer weight matrices
observed in modern DNNs because their ESDs are \HeavyTailed \PowerLaw\cite{MM19_HTSR_ICML}.
We can then readily express the \Quality $\Q$ of the \Teacher
layer in a simple functional form, (i.e  an approximate Shatten Norm).

Importantly, the matrices $\mathbf{X}$  and $\mathbf{\AMAT}$ must be well approximated
by low rank matrices since the derivation in Tanaka requires this.  Fortunately,
this appears to be generally true for the layers in very well trained DNNs,
which is what allows us to apply this withing the~\ECS.
In fact, technically we need to integrate over $d\mu(\AECS)$; this is straightforward
as this simply changes the lower bound on the integral from $0\rightarrow\LambdaECSmin$,
both above and in the subsequent derivation.
\red{For now, we omit this detail; I may add it in later}

Finally, we note that $\GNORM(\mathbf{X})$ is kind of \emph{Generalized Norm} because 
it can be evaluated as a sum over a function of the $M$ eigenvalues $\lambda_{\mu}$ of the \Teacher
correlation matrix $\mathbf{X}=\frac{1}{N}\mathbf{W}^{\top}\mathbf{W}$.
$\GNORM(\mathbf{X})$ will turn out to be an expression similar to the Frobenius Norm or the
Shatten Norm of $\mathbf{X}$, depending on the functional form we choose to represent the
limiting form of the \Student ESD, $\rho_{\AMAT}^{\infty}(\lambda)$ and the associated \RTransform $R(z)$, and various approximations made thereafter,

\michael{Those three paragraphs are more like comments, so shouldnt be here in the appendix. Put at the right spot in the main text.}
\charles{@michael: where ? }

\subsubsection{Setup and Outline}
\label{sxn:tanaka_setup}

\noindent
To evaluate \ref{eqn:hciz2},
we %%MM%% In setting up the problem for this paper, we 
want to integrate over all \Student Correlation matrices $\mathbf{\AMAT}$
that ``resemble the \Teacher Correlation matrix $\mathbf{X}$.  
To %%MM%% So, first, we need to 
formalize this idea,
we need to %%MM%% which requires that we 
define the measure over ``all desired $\mathbf{\AMAT}$, $d\mu(\mathbf{\AMAT})$, 
in 
terms of the actual $M$ eigenvalues, $\left\{ \lambda_{i} \right\}_{i=1}^{M}$, of the \Teacher.


\paragraph{Randomness assumption.}
For real weights $\mathbf{W}$,  we assume an \emph{orthogonally invariant} ensemble,
$d\mu(\mathbf W)=d\mu(\mathbf U\mathbf W\mathbf U^{\top})$ for all $\mathbf U\in O(M)$,
mirroring the isotropic Gaussian initialisation widely used in neural networks.
Crucially, Tanaka’s large-$N$ analysis shows the resulting HCIZ exponent depends only on the eigenvalue spectrum, so the final integrated–$R$ expression should remain applicable even when full rotational invariance is later broken in training.

%%\paragraph{Representing $d\mu(\AMAT)$ with $d\mu(\WMAT)$ using the source matrix $\mathbf{D}$.}
\paragraph{Using a source matrix $\mathbf{D}$ to represent $d\mu(\AMAT)$ with $d\mu(\WMAT)$.}

We consider all matrices $\mathbf{\AMAT}$ with the same limiting spectral density, $\rho_{\AMAT}^{\infty}(\lambda)$,
as the limiting (\emph{empirical}) ESD of the \Teacher.
That is, we want $\rho_{\AMAT}^{\infty}(\lambda)=\rho^{\infty}_{\WMAT}(\lambda)$, where $\TMAT=\WMAT$.
%\nred{Comment on the nature of the randomness required to be invariant under unitary transformations.}
Of course, there are infinitely many weight matrices $\mathbf{W}$ with the same $M$ eigenvalues, $\left\{ \lambda_{i} \right\}_{i=1}^{M}$, as the \Teacher.
Let us specify these matrices with the measure $d\mu(\mathbf{W})$.
Doing this lets us then write the measure $d\mu(\mathbf{\AMAT})$ in terms of $d\mu(\mathbf{W})$ as:
\begin{equation}
\label{eqn:dmuA}
d\mu(\mathbf{\AMAT}) 
   := e^{- \frac{\beta}{2} Tr[\mathbf{W}\mathbf{D}\mathbf{W}^{\top}]} d\mu(\mathbf{W})  ,
\end{equation}
where $\mathbf{D}$ is some $M \times M$ matrix, called the \SourceMatrix, to be specified below,
and the $\tfrac{1}{2}$ here as well.
Indeed, the key idea here will be to define $\mathbf{D}$ in such a way as to obtain the desired final result.
Notice also that we have added a $\beta$ term; this will be factored out later.
%Notice that we  define $\mathbf{\AMAT}$ through this change of measure, and 
%$\mathbf{\AMAT}$ should be an $M \times M$ matrix (i.e. $\mathbf{\AMAT}:=\tfrac{1}{N}\mathbf{S}^{T}\mathbf{S}$),
%but we could also let $\mathbf{\AMAT}$ be an $N \times N$ matrix, (i.e $\mathbf{\AMAT}:=\tfrac{1}{N}\mathbf{S}\mathbf{S}^{T}$),
%simply with $N-M$ zero eigenvalues.
%\paragraph{Unitary‐invariant randomness.}
%To ensure the ensemble contains \emph{no information beyond the fixed spectrum} $\{\lambda_i\}$, impose invariance under every orthogonal rotation:
%\begin{equation}
%  d\mu(\mathbf{W}) \;=\; d\mu\!\bigl(\mathbf{U}\mathbf{W}\mathbf{U}^{\top}\bigr),
%  \qquad
 % \forall\,\mathbf{U}\in O(M).
%\end{equation}
%Haar‐averaging over $\mathbf{U}$ renders all eigenvector choices equally likely, so the HCIZ integral collapses to a pure eigenvalue problem; this isotropy drives the saddle‐point/large‐deviation steps in Eqns.~\ref{eqn:ZD_step1}–\ref{eqn:ZD_step2}.


We can now represent the partition function
$\ZD$, by inserting \EQN~\ref{eqn:dmuA} into \EQN~\ref{eqn:hciz_def}.
$\ZD$ is now defined as an integral over all
possible (\Teacher) weight matrices $\mathbf{W}$
\begin{equation}
  \label{eqn:ZD0}
    \ZD=\int d\mu(\mathbf{W})\exp[\frac{\beta}{2}\left
    (\Trace{\mathbf{W}^{\top}\AMAT_{2}\mathbf{W}}-\Trace{\mathbf{W}\mathbf{D}\mathbf{W}^{\top}}  
    \right)]  ,
\end{equation}
%where the matrix $\mathbf{D}$ is the called the \SourceMatrix.
\michaeladdressed{@charles: What is this a ``modification of? I get it if it is just a regularized objective, with Lagrange parameters to be determined; but are we going to use it as a generating function or partition function of some effective physical system? Maybe a sentence or two saying why doing this modification makes sense. This is probably just going to be a vector of prices or lagrange parameters that control the constraint satisfaction?}
\michaeladdressed{Also, $Z(D)$ versus $\ZD$?}
Observe that this integral only converges when all the eigenvalues of $\DMAT$, 
$\left\{ \DeltaMu \right\}_{\mu=1}^{M}$, 
are larger than the maximum eigenvalue of $\mathbf{\AMAT}$, i.e., when $\DeltaMu >\lambda_{max}$, for $\mu\in[1,M]$
(although below this will become $\beta\DeltaMu >\lambda_{max}$).
Later, we will place $\mathbf{D}$ in diagonal form, and we will obtain an explicit expression for its $M$ eigenvalues in terms of the $M$ non-zero eigenvalues of $\mathbf{X}$.
The eigenvalues of $\mathbf{D}$ will turn out to Lagrange Multipliers, needed later.


\paragraph{The Saddle Point Approximation (SPA) and the Large Deviation Principle (LDP).}

To evaluate the large-$N$ case of $\IZG$ (see \ref{eqn:hciz_def2},~\ref{eqn:hciz_def3}), 
we assume that the distribution of possible \Teacher correlation matrices,
$\mu(\mathbf{X})$, satisfies a \emph{Large Deviation Principle (LDP)}.
A LDP applies to probability distributions that take an exponential form,
such that $\mu(\mathbf{X})=e^{-N I(\mathbf{X})}d\mu(\mathbf{X})$,
where  $I(\mathbf{X})$ is Entropy or Rate function $I(\mathbf{X})$.
\michael{REF.}
\michael{Maybe write out the general expression, citing that ref, so that it is clear how \ref{eqn:EZD} follows from it, given our setup.}
%\charles{(((
%Is this a resonable assumption ?
%While we model the ESD of $\mathbf{X}$ as a \PowerLaw (PL), it is really
%finite-size distrubtion, best modeled by a Truncated \PowerLaw (TPL),
%either cut off at the largest eigenvalue $\lambda^{max}$
%and/or with an exponential decay~\cite{YTHx22_TR}.
%)))}
In applying a LDP, we effectively restrict measure of student correlation matrices $\mathbf{\AMAT}$
to those most similar to the empirically observed \Teacher correlation matrix $\mathbf{X}$.
%
We expect the measure over all \Teacher correlation matrices
follows an LDP because the ESD is far from Gaussian,
the dominant generalizing components reside in the tail of the ESD,
and at finite-size the tail decays at worst as an exponentially
Truncated \PowerLaw (TPL).

%\nmove{ MOVE THIS BELOW:
%Using the LDP (and following similar approaches in spin glass theory \cite{PP95}),
%below we will show that we can write the expected value of $\ZD$ 
%in terms of $d\mu(\mathbf{X})$ now (as opposed to $d\mu(\mathbf{A})$)
%and in the large-$N$ approximation, as
%\begin{equation}
%  \label{eqn:EZD}
%  \EZDA=
%  \int\exp\left(\beta N Tr[\GNORM(\mathbf{X})]-NI(\mathbf{X})+o(N)\right)d\mu(\mathbf{X})
%\end{equation}
%where $I(\mathbf{X})$ is the \RateFunction, defined below.
%}
%\michael{@charles: Which previous equation from this section does this follow from?}
%\charles{None.  We defined it beliow}


\paragraph{Two steps to evaluate $\Expected[\AMAT]{\ZD}$ in the large-$N$ approximation.}
The goal is to start with \EQN~\ref{eqn:ZD0} and obtain two separate, equivalent
relations, Eqns.~\ref{eqn:ZD_step1} and ~\ref{eqn:ZD_step2}:
%this proceeds in the following two steps:
\begin{enumerate}
   \item
   \textbf{Obtaining an integral transform of $\rho^{\infty}_{\AMAT}(\lambda)$.}
   First, we expand and reduce \EQN~\ref{eqn:ZD0} and evaluate the expected value of
   $\EZDA=\EZDATWO$ in the large-$N$ limit by expressing the $\rho_{\AMAT}(\lambda)$
   for the $N\times N$ matrix $\AMAT=\mathbf{A}_{2}=\tfrac{1}{N}\SMAT\SMAT^{\top}$
   in the continuum representation, i.e., as ]
   $\rho^{emp}_{\AMAT}(\lambda)\rightarrow \rho^{\infty}_{\AMAT}(\lambda)$, to obtain:
   \begin{equation}
      \label{eqn:ZD_step1}
      \lim_{N\gg 1}\dfrac{1}{N}
      \ln\EZDATWO =M\ln(\dfrac{2\pi}{\beta})-\sum_{\mu=1}^{M}\int \ln(\delta_{\mu}-\lambda)\rho^{\infty}_{\AMAT}(\lambda)d\lambda  .
   \end{equation}
   This gives us an $\EZDATWO$ in terms of an integral transform $\rho^{\infty}_{\AMAT}(\lambda)$, which we can model.\footnote{This integral of $\rho^{\infty}_{\AMAT}(\lambda)$  is related to the \emph{Shannon Transform}, an integral transform from information theory that is useful when analyzing the mutual information or the capacity of a communication channel~\cite{Tanaka2007}. }
   \michaeladdressed{I dont think thats obvious, is it? I thought the point was that is is convex/concave when $\delta_{\mu} >\lambda_{max}$, in which case it is a Laplace(?) transformation (which we can then invert)? Or is this something else?}
   \item
   \textbf{Forming the \SaddlePointApproximation (SPA).}
   We evaluate \EQN~\ref{eqn:ZD0} as the expected value of $\EZDA=\EZDAONE$
   for the $M \times M$ matrix $\AMAT=\mathbf{A}_{1}=\tfrac{1}{N}\SMAT^{\top}\SMAT$
   (but explicitly in terms of $d\mu(\mathbf{X})$).
   Then, taking in the large-$N$ approximation using the SPA,
    (and which can be done implicitly using the LDP), we obtain
   \begin{equation}  
  \label{eqn:ZD1} 
  \lim_{N \gg 1} \EZDAONE\simeq\int  \exp\left(\beta N\Trace{\GFANCY}\right)d\mu(\mathbf{X}) \approx \exp(\beta N\GMAX)
\end{equation}
  where  $\GFANCY$ depends on $\GNORM(\mathbf{X})$, and $\GMAX=\sup_{\XMAT}\GFANCY$.
  We can then write
   \begin{equation}
      \label{eqn:ZD_step2}
      \lim_{N \gg 1}\dfrac{1}{N}\ln\EZDAONE \approx \beta\GMAX  ,
   \end{equation}

 \item
 \textbf{Finding the Inverse Legendre Transform.}
  To do this, we now equate
  \begin{equation}
  \lim_{N \gg 1}\frac{1}{N}\ln\EZDAONE=  \lim_{N \gg 1}\frac{1}{N}\ln\EZDATWO
  \end{equation}
 Then, we can form the
 %with a suitable choice for the source matrix $\mathbf{D}$, we can form the
 inverse Legendre transform  which we will let us relate $\GNORM(\lambda)$ in \EQN~\ref{eqn:hciz} to the integrated \RTransform of $\rho^{\infty}_{\AMAT}(\lambda)$.
\end{enumerate}

\noindent

(See~\ref{sxn:tanaka_end}.)



\subsubsection{Step 1. Forming the Integral Transformation of ESD 
\texorpdfstring{$(\rho_{\AMAT}^{\infty}(\lambda))$}{rho(lambda)}}
\label{sxn:tanaka_step1}
We first establish \EQN~\ref{eqn:ZD_step1}, in Steps $1.1-1.4$.
This is done by changing variables under a Unitary transformation, $\mathbf{W}\rightarrow\mathbf{\check{W}}$,
evaluating the resulting functional determinant,
and then taking the continuum limit of the ESD
$\tilde{\rho}_{\AMAT}(\lambda)\rightarrow\rho^{\infty}_{\AMAT}(\lambda)$.

\paragraph{Step 1.1}
%\nred{We need to check all the equations, I might have flipped the transpose in the second term with the source matrix from right to left.}\michael{I think I fixed this, check.}
To do so, let us first assume that \Teacher correlation matrix $\mathbf{X}$ and the source matrix $\mathbf{D}$
are simultaneously diagonalizable
(i.e., their commutator is zero: $[\mathbf{X}, \mathbf{D}]=0$).
In this case, we may write the generating function $\ZD$ in \EQN~\ref{eqn:ZD0} as
%
\begin{align}
\label{eqn:Z-diag}
\ZD &= \int d\mu(\mathbf{W}) \exp\frac{\ND\beta}{2}
 \bigg( 
\Trace{\mathbf{W}^{\top}\mathbf{U}^{\top}\mathbf{\Lambda}\mathbf{U}\mathbf{W}} 
- \Trace{\mathbf{W}\mathbf{V}^{\top}\mathbf{\Delta}\mathbf{V}\mathbf{W}^{\top}} 
\bigg)  ,
\end{align}
\michaeladdressed{How do we get this?  Is this just \ref{eqn:dmuA}, rewritten?  But we have changed $d\mathbf{W}$ to $d\mathbf{X}$ and changed $\ND\beta$ to $1/2$?  Or does this use the assumption $[\mathbf{X}, \mathbf{D}]=0$?}
where we have defined
%
\begin{equation}
\label{eqn:diag-A-D}
    \AMATN=\mathbf{U}^{\top}\mathbf{\Lambda}\mathbf{U},\;\;
    \mathbf{D}=\mathbf{V}^{\top}\mathbf{\Delta}\mathbf{V}  ,
\end{equation}
%
%%where $\mathbf{U}, \mathbf{V}$ are Unitary matrices
%%with $\mathbf{U}$ is $(N\times N)$ and $\mathbf{V}$ is $(M\times M)$.
where $\mathbf{U}$ ($N\times N$) and $\mathbf{V}$ ($M\times M$) are Unitary matrices.
%
%% 
%% %
%% Using the Orthogonality properties of $\mathbf{U}$ and $\mathbf{V}$,
%% %
%% \begin{equation}\label{eqn:UU}
%%     \mathbf{U}^{T}\mathbf{U}=\mathbf{I},\;\;
%%     \mathbf{V}^{T}\mathbf{V}=\mathbf{I},
%% \end{equation}
Since $\mathbf{U}^{\top}\mathbf{U}=\mathbf{I}$ and $\mathbf{V}^{\top}\mathbf{V}=\mathbf{I}$,
%
we can insert these identities into $\ZD$ in \ref{eqn:Z-diag}, giving
%
\begin{align}\label{eqn:Z0-diag}
\ZD &= \int d\mu(\mathbf{W}) \exp\frac{\ND\beta}{2}\times  \\ \nonumber
&\bigg(\Trace{(\mathbf{V}^{\top}\mathbf{V})\mathbf{W}^{\top}\mathbf{U}^{\top}\mathbf{\Lambda}\mathbf{U}\mathbf{W}(\mathbf{V}^{\top}\mathbf{V})} 
-\Trace{(\mathbf{U}^{\top} \mathbf{U})\mathbf{W}\mathbf{V}^{\top} \mathbf{\Delta} \mathbf{V}\mathbf{W}^{\top}(\mathbf{U}^{\top} \mathbf{U})}\bigg)  .
\end{align}
We can identify the reduced weight matrix $\mathbf{\check{W}}$ as
\begin{equation}
   \label{eqn:Wcheck}
   \mathbf{\check{W}}=\mathbf{U}\mathbf{W}\mathbf{V}^{\top},\;\;
   \mathbf{\check{W}}^{\top}=\mathbf{V}\mathbf{W}^{\top}\mathbf{U}^{\top}  ,
\end{equation}
Rearranging parentheses, this gives 
\begin{align}
\ZD &= \int d\mu(\mathbf{W}) \exp\frac{\ND\beta}{2}\times  \\ \nonumber
&\bigg(\Trace{\mathbf{V}^T(\mathbf{V}\mathbf{W}^T\mathbf{U}^T)\mathbf{\Lambda}(\mathbf{U}\mathbf{W}\mathbf{V}^T)\mathbf{V}}  
-\Trace{ \mathbf{U}^{\top}(\mathbf{U} \mathbf{W}\mathbf{V}^{\top})\mathbf{\Delta}(\mathbf{V}\mathbf{W}^{\top} \mathbf{U}^{\top})\mathbf{U} }\bigg)  .
\end{align}
We can now express $\ZD$ in terms of $\mathbf{\check{W}}$ as
\begin{align}
\label{eqn:hciz-W-red}
\ZD &= \int d\mu(\mathbf{W})\exp\frac{\ND\beta}{2}
 \bigg(\Trace{\mathbf{V}^{\top}\mathbf{\check{W}}^{\top}\mathbf{\Lambda}\mathbf{\check{W}}\mathbf{V}} 
 -\Trace{\mathbf{U}^{\top}\mathbf{\check{W}}\mathbf{\Delta}\mathbf{\check{W}}^{\top}\mathbf{U}}\bigg)  .
\end{align}
Since the Trace operator $\Trace{\cdot}$ is invariant to Unitary (Orthogonal) transformations, we can
now remove the
$\mathbf{U}$ and $\mathbf{V}$ terms, giving the simplified expression
for our generating function $\ZD$ in terms of
the two diagonal matrices $\mathbf{\Lambda}, \mathbf{\Delta}$, 
the reduced weight matrix $\mathbf{\check{W}}$, and
the Jacobian $J(\mathbf{\check{W}})$ transformation for $d\mu(\mathbf{W})\rightarrow d\mu(\mathbf{\check{W}})$, as:
\begin{align}
\label{eqn:hciz-W-red2}
    \ZD & =\int d\mu(\mathbf{\check{W}})J(\mathbf{\check{W}})\exp\frac{\ND\beta}{2}
 \bigg( \Trace{\mathbf{\check{W}}^{\top}\mathbf{\Lambda}\mathbf{\check{W}}} 
       -\Trace{\mathbf{\check{W}}\mathbf{\Delta} \mathbf{\check{W}}^{\top}} \bigg)  .
\end{align}


\paragraph{Step 1.2}

\michael{@charles: we are switching gears here, Im missing something in the flow. Where do we get the next equation from?}
\charles{@michael: standard stuff.  Should we provide a reference ?  Can you find it ?}
We can now evaluate the
integral using the standard relation for the functional determinant for infinite-dimensional Gaussian integrals~\cite{EngelAndVanDenBroeck}

\begin{equation}
\label{eqn:hciz-det}
    \ZD=\left(\dfrac{2\pi}{\ND\beta}\right)^{NM/2}\det\left(\mathbf{\Delta}-\mathbf{\Lambda}\right)^{-1/2}
\end{equation}
where the Jacobian is unity for the Unitary transformation.
\michaeladdressed{Is this $\mathbf{W}$ or $d\mathbf{\check{W}}$?  Also, this is a weight matrix, not an orthogonal matrix correct?  Meaning that the Jacobian being one is due to the Trace-Log condition?}
\charles{@michael:
The transformation $\mathbf{W} \mapsto \check{\mathbf{W}} = \mathbf{U}\,\mathbf{W}\,\mathbf{V}^{\top}$ is orthogonal, so $J(\check{\mathbf{W}})=1$.}

\begin{equation}\label{eqn:Jacobian}
    J(\mathbf{\check{W}})=1  .
\end{equation}
since $\mathbf{W} \mapsto \check{\mathbf{W}}$  is an orthogonal transfomation.
We now use the standard Trace-Log-Determinant relation~\cite{EngelAndVanDenBroeck}
\begin{equation}\label{eqn:tr-ln-det}
    \Trace{\ln\mathbf{M}}=\ln\det\mathbf{M}  .
\end{equation}
Let us insert $(\exp\ln)$ on the R.H.S. of \ref{eqn:hciz-det}, to obtain
\begin{align}
\nonumber
\ZD
  &=\exp\ln\bigg[\left(\dfrac{2\pi}{\ND\beta}\right)^{NM/2}\det\left(\mathbf{\Delta}-\mathbf{\Lambda}\right)^{-1/2}\bigg] \\ 
\nonumber
  & =\exp\bigg[\left(\dfrac{NM}{2}\right)\ln\dfrac{2\pi}{\ND\beta}-\dfrac{1}{2}\Trace{\ln\left(\mathbf{\Delta}-\mathbf{\Lambda}
\right)}\bigg] \\ 
\label{eqn:hciz-exp-ln}
  & =\exp\bigg[\dfrac{NM}{2}\ln\dfrac{2\pi}{\ND\beta}-\dfrac{1}{2}\ln\det\left(\mathbf{\Delta}-\mathbf{\Lambda}\right)\bigg]  .
\end{align}


\paragraph{Step 1.3}
We now want to express
%the Log-Determinant, $\ln\det\left(\mathbf{\Delta}-\mathbf{\Lambda}\right)$,
the generating function $\ZD$ 
in \ref{eqn:hciz-exp-ln}
in terms of an integral over the continuous limiting spectral density
$\rho_{\AMAT}(\lambda)$ of the correlation matrix $\AMATN$.  

First, we express the Determinant of the matrix $\mathbf{\Delta}-\mathbf{\Lambda}$ in terms of discrete eigenvalues:
\michaeladdressed{I think so; are you asking if we got the $M$ and $N$s correct?}
\charles{check minus sign}
\begin{equation}
\label{eqn:det-discrete}
    \det\left(\mathbf{\Delta}-\mathbf{\Lambda}\right)^{-1/2}=\prod_{\mu=1}^{M}\prod_{i=1}^{N}\left(\DeltaMu-\lambda_{i}\right)^{-1/2}  .
\end{equation}
%
This gives the Log-Determinant in terms of the $M$ (non-zero)
eigenvalues of $\mathbf{D}$ and $\AMATN$, as
\begin{equation}
\label{eqn:ln-det-discrete}
    \ln\det\left(\mathbf{\Delta}-\mathbf{\Lambda}\right)^{-1/2}=-\dfrac{1}{2}\sum_{\mu=1}^{M}\sum_{i=1}^{N}\ln\left(\DeltaMu-\lambda_{i}\right)  .
\end{equation}
%
We can express %Define 
the ESD, $\tilde\rho_{\AMAT}(\lambda)$, of the 
Student Correlation 
matrix
$\AMAT_N$ in terms of the Dirac delta-function, $\delta(x)$, as
\begin{equation}
\label{eqn:rho-emp}
    \tilde\rho_{\AMAT}(\lambda)=\frac{1}{N}\sum_{i=1}^{N}\delta(\lambda-\lambda_{i})  .
\end{equation}
Using this, the \ExpectedValue of the Log-Determinant 
in \ref{eqn:ln-det-discrete}
can be expressed in terms of the ESD of
$\AMAT_N$ as
\begin{align}
\nonumber
\Expected[\AMAT_N]{\ln\det\left(\mathbf{\Delta}-\mathbf{\Lambda}\right)^{-1/2}}
   & = -\dfrac{1}{2}\sum_{\mu=1}^{M}\sum_{i=1}^{N}\int
       d\lambda\ln(\DeltaMu-\lambda)\delta(\lambda-\lambda_{i}) \\ 
\nonumber
   & = -\dfrac{1}{2}\sum_{\mu=1}^{M}\int
       d\lambda\ln(\DeltaMu-\lambda)\sum_{i=1}^{N}\delta(\lambda-\lambda_{i}) \\ 
\label{eqn:ln-det-rho}
   & = -\dfrac{1}{2}\sum_{\mu=1}^{M}\int
       d\lambda\ln(\DeltaMu-\lambda)
       N\tilde\rho_{\AMAT}(\lambda)  .
\end{align}

Let us insert this back into our
expression for the generating function,
\ref{eqn:hciz-exp-ln},   %% ~\ref{eqn:hciz-exp-ln3}, 
giving
$\EZDATWO$ in terms of the ESD $\tilde{\rho}_{\AMAT}$ as
\begin{equation}
\label{eqn:Z-rho}
    \EZDATWO=\exp\bigg\{\dfrac{N}{2}\big[M\ln\dfrac{2\pi}{\ND\beta}-\sum_{\mu=1}^{M}\int
        d\lambda\ln(\DeltaMu-\lambda)\tilde{\rho}_{\AMAT}(\lambda)\big]\bigg\}  .
\end{equation}
\michael{BTW, I get a $1/\ND\beta$ on the second term, when I derive it, so I think we missed a $\ND\beta$ in the numerator somewhere.}
\charles{Yeah we gotta check all this very carefully}

We can now replace
the sum over the $N$ eigenvalues $\lambda_{i}$ with an integral over the limiting
ESD, $\rho(\lambda)$, to obtain
\begin{equation}
\label{eqn:rho-}
\rho^{\infty}_{\AMAT}(\lambda)=    \lim_{N\rightarrow\infty}\tilde\rho_{\AMAT}(\lambda)  .
\end{equation}
Observe that this effectively means that we are taking a \LargeN limit in $N$, $N\gg 1$.
%
This lets us write the \ExpectedValue of the generating function $\ZD$
in \ref{eqn:Z-rho}
as
\begin{equation}
\label{eqn:Z-rho-2}
    \lim_{N\gg 1}\EZDATWO=\exp\bigg\{\dfrac{N}{2}\big[M\ln\dfrac{2\pi}{\ND\beta}-
    \sum_{\mu=1}^{M}\int
        d\lambda\ln(\DeltaMu-\lambda)\rho^{\infty}_{\AMAT}(\lambda)\big]\bigg\}
\end{equation}
\michael{We just added an expectation. Is that correct? I think we missed something.}
\charles{No, this is correct.  But I obviously did not explain it well.  Will review}

\paragraph{Step 1.4}

Using the Self-Averaging Property,
\begin{equation}
   \ln \EZDATWO \simeq \Expected[\mathbf{A}_M]{\ln \ZD} ,
\end{equation}
It follows from \EQN~\ref{eqn:Z-rho-2}
that
\begin{equation}
   \lim_{N\gg 1} \ln \EZDATWO
   \simeq \dfrac{N M}{2}\ln\dfrac{2\pi}{\ND\beta}
         -\dfrac{N}{2}\sum_{\mu=1}^{M}\int d\lambda\ln(\DeltaMu-\lambda)\rho^{\infty}_{\AMAT}(\lambda)  .
\end{equation}
%
The $N$-dependence now cancels out,
and we are left an approximate expression due to the remaining dependence of the continuum limiting density
$\rho^{\infty}_{\AMAT}(\lambda)$ (for $\AMAT=\AMATN$)
%
\begin{equation}
\label{eqn:ln-Z-Nlim2}
    \lim_{N \gg 1}\dfrac{2}{N}\ln \EZDATWO
    = M\ln\dfrac{2\pi}{\ND\beta}-\sum_{\mu=1}^{M}\int d\lambda\ln(\DeltaMu-\lambda)\rho^{\infty}_{\AMAT}(\lambda)  .
\end{equation}
This completes the derivation of \EQN~\ref{eqn:ZD_step1}; 
we have an expression for the expected value of $\ZD$,
evaluated in the \LargeN (continuum) limit in $N$.


\subsubsection{Step 2: The Saddle Point Approximation (SPA): Explicitly forming the Large Deviation Principle (LDP)}
\label{sxn:tanaka_step2}
We now evaluate $\EZDA$ in \EQN~\ref{eqn:ZD1} as $\EZDATWO$ 
to  establish \EQN~\ref{eqn:ZD_step2}, \charles{in Steps $2.1-2.6$}.

Using the LDP (and following similar approaches in spin glass theory \cite{PP95}),
below we will show that we can write the expected value of $\ZD$ 
in terms of $d\mu(\mathbf{X})$ now (which is equivalent to $d\mu(\AMAT_{1})$)
and in the large-$N$ approximation, as
\begin{equation}
  \label{eqn:LDP}
 \lim_{N \gg 1} \EZDAONE=
  \int\exp\left(\beta N \Trace{\GNORM(\mathbf{X})}-NI(\mathbf{X})+o(N)\right)d\mu(\mathbf{X})
\end{equation}
where $I(\mathbf{X})$ is  \RateFunction, defined below,
and  $\GNORM(\mathbf{X})$ is what we are eventually solving for.

\paragraph{Step 2.0} We start with the \emph{expected} \PartitionFunction
\begin{align}
  \label{eqn:avg_ZD}
  \EZDA
  &=
  \int d\mu(\AMAT)\int  d\mu(\WMAT)
      \exp\Bigl[
         \tfrac{\beta}{2}\Trace{\WMAT^{\TR}\AMAT\,\WMAT}
        -\tfrac{\beta}{2}\Trace{\WMAT\DMAT\WMAT^{\TR}}
      \Bigr].
\end{align}
The average over $\AMAT$ affects only the first exponential; applying the SPA, we
\textbf{define} a matrix function $\GFANCY$, depending solely on
$\XMAT=\frac1N\WMAT^{\TR}\WMAT$, by
\begin{align}
  \label{eqn:def_GFANCY}
  \int d\mu(\AMAT)\,
        \exp\Bigl[\tfrac{\beta}{2}\Trace{\WMAT^{\TR}\AMAT\,\WMAT}\Bigr]
  &=
  \exp\Bigl[\tfrac{\beta N}{2}\Trace{\GFANCY}\Bigr].
\end{align}
which will be valid in the large-$N$ approximation below.

Inserting \eqref{eqn:def_GFANCY} into \eqref{eqn:avg_ZD} gives
\begin{align}
  \label{eqn:ZD_after_Aavg}
  \EZDA
  &=
  \int d\mu(\WMAT)
      \exp\Bigl[
         \tfrac{\beta N}{2}\Trace{\GFANCY}
        -\tfrac{\beta}{2}\Trace{\WMAT\DMAT\WMAT^{\TR}}
      \Bigr].
\end{align}

The now need to  determine an explicit form for
$\GFANCY$. We introduce a new change of measure, 
 $d\mu(\mathbf{W})\rightarrow d\mu(\mathbf{X})$.
Then, we show this lets us express $\EZDAONE$ as $\EZDX$ and to express it using the LDP.
Next, we apply a SPA to solve for $\GMAX=max\;\GNORM$.
Importantly, we also show how to incorporate the inverse-Temperature $\beta$,
which is new.

\paragraph{Step 2.1}
To define the transformation $d\mu(\mathbf{W})\rightarrow d\mu(\mathbf{X})$,  where (recall) $\mathbf{X}=\frac{1}{N}\mathbf{W}^{\top}\mathbf{W}$,
we use the (again) the integral representation of the Dirac delta-function $\delta(x)$:
\begin{equation}
  \label{eqn:dirac}
  \delta(x):=\frac{1}{2\pi}\int_{-\infty}^{\infty} e^{i\hat{x}x} d\hat{x}.
\end{equation}
%
This lets us express the transformation of measure $d\mu(\mathbf{W})\rightarrow d\mu(\mathbf{X})$
(approximately) as
\begin{align}
\nonumber
  d\mu(\mathbf{W}) &:= \delta(\frac{1}{2}\Trace{N\mathbf{X}-\mathbf{W}^{\top}\mathbf{W}}) d\mu(\mathbf{X}) \\ 
  &= \frac{1}{2\pi}\int_{-\infty}^{\infty} e^{i\frac{1}{2}
    \Trace{\hat{X}(N\mathbf{X}-\mathbf{W}^{\top}\mathbf{W})}
  }
  d\mu(\mathbf{X})d\mu(\hat{X} ),
\end{align}
where $\hat{X}$ is a scalar (or really matrix of scalars),
and we have a $1/2$ term for mathematical consistency below.
\footnote{The full change of measure would require a delta function constraint
for each matrix element $X_{i,j}$, i.e., 
$\delta\left(\frac{1}{2}N\left(X_{i,j}-[\mathbf{W}^{\top}\mathbf{W}]_{i,j}\right)\right)$.
Here, we assume the Trace constraint is sufficient for our level of rigor.
}



\paragraph{Step 2.2}
Next, we take a Wick Rotation, 
\footnote{The Wick rotation converts an oscialltory integral into an exponentially decaying one which should be well defined.  Technically, this is an analytic continuation which needs to be checked, but following standard practice in physics we will assume the resulting integral is analytic and therefore well defined and we will proceed onwards. }$i\XHAT\rightarrow \XHAT$, so that the terms under the integral are all real (not complex), giving:
\begin{align}
  \label{eqn:dmuX}
  d\mu(\mathbf{W}) &= \red{\mathcal{N}_{Wick}}\int_{-i\infty}^{i\infty} e^{\red{-}\frac{1}{2}\Trace{\hat{X}(N\mathbf{X}-\mathbf{W}^{\top}\mathbf{W})}} d\mu(\mathbf{X})d\mu(\hat{X} ).
\end{align}
where $d\mu(\hat{X})$ is a measure over $\tfrac{M(M-1)}{2}$ Lagrange multipliers, and the normalization is  $$\mathcal{N}_{Wick}=(\frac{1}{2\pi i})^{\tfrac{M(M-1)}{2}}$$.


\nred{The minus sign has not been propagated through yet and may be confusing; I will fix shortly. The final result is not changed.  }

\paragraph{Step 2.3}
We now insert \ref{eqn:dmuX} into~\ref{eqn:ZD_after_Aavg},  which lets
express $\EZDA$ as an integral over the \Teacher Correlation matrices

\begin{align}
  \nonumber
  \EZDA&=
  \red{\mathcal{N}_{Wick}} \int_{\XMAT}  \int_{-i\infty}^{i\infty}
  e^{N\frac{\beta}{2} \Trace{\GFANCY}+\frac{N}{2}\Trace{\hat{X}\mathbf{X}} }
  e^{-\frac{1}{2}\Trace{\hat{X}\mathbf{W}^{\top}\mathbf{W}}}
  e^{\frac{\beta}{2}\Trace{\mathbf{W}\mathbf{D}\mathbf{W}^{\top}}}
  d\mu(\hat{X} )
  d\mu(\mathbf{\XMAT}) \\ 
  \nonumber
  &=
  \red{\mathcal{N}_{Wick}} \int_{\XMAT}  \int_{-i\infty}^{i\infty}
  e^{N\frac{\beta}{2}\Trace{\GFANCY}+ \frac{N}{2}\Trace{\hat{X}\mathbf{X}}}
  e^{-\frac{1}{2}\Trace{\mathbf{W}\hat{X}\mathbf{W}^{\top}}+
  \frac{\beta}{2}\Trace{\mathbf{W}\mathbf{D}\mathbf{W}^{\top}}}
  d\mu(\hat{X} )
  d\mu(\mathbf{\XMAT}) \\ 
  \label{eqn:ZD3}
    &=
  \red{\mathcal{N}_{Wick}} \int_{\XMAT}  \int_{-i\infty}^{i\infty}
  e^{N\frac{\beta}{2}\Trace{\GFANCY}+
  \frac{N}{2}\Trace{\hat{X}\mathbf{X}}}
  e^{\frac{1}{2}\Trace{\mathbf{W}(\beta\mathbf{D}-\hat{X})\mathbf{W}^{\top}}}
  d\mu(\hat{X} )
  d\mu(\mathbf{\XMAT})  .
\end{align}

\paragraph{Step 2.4}
We can now rearrange terms to make this expression look like the \EQN~\ref{eqn:LDP}
%Let us write
%express \EQN~\ref{eqn:ZD3} as the expected value $\EZDA$ in terms of 
%\michael{Huh?}
%\charles{SOME EXPLANATIONS NEEDED: }
%\begin{equation}
%  \EZDA=\int\exp\left(\beta N\Trace{\GNORM}-NI(\mathbf{X})+o(N)\right)d\mu(\mathbf{X})
%\end{equation}


In Large Deviations Theory, the \RateFunction is defined by the Legendre Transform,
\begin{equation}
\label{eqn:rate-fun}
    \mathcal{I}(\mathbf{X})=\underset{\mathbf{\check{X}}}{\sup}
    \left[Tr\dfrac{1}{2}\mathbf{{X}}^{\top}\mathbf{\check{X}}-\ln\mathbb{M}(\mathbf{\check{X}})\right]  ,
\end{equation}
where $\mathbb{M}(\mathbf{\check{X}})$ is the \MomentGeneratingFunction,
$\ln\mathbb{M}(\mathbf{\check{X}})$, is the \CumulantGeneratingFunction, and
and $\mathbf{\check{X}}$ is a (matrix of) \emph{Lagrange Multiplier}(s).  $\mathbb{M}(\mathbf{\check{X}})$ is defined in terms of the (unnormalized) density $p(\mathbf{x})$ as
\begin{equation}
\label{eqn:rate_function}
   \mathbb{M}(\mathbf{\check{X}})=\exp\left(\frac{1}{2}\mathbf{x}^{\top}\mathbf{\check{X}}\mathbf{x}\right)  ,
  p(\mathbf{x})d\mathbf{x}
\end{equation}
which, in term, is defined in terms of the source matrix $\mathbf{D}$,
\begin{equation}
\label{eqn:X_density}
p(\mathbf{x})=\exp\left(-\tfrac{1}{2}\mathbf{x}^{\top}\beta\mathbf{D}\mathbf{x}\right)  .
\end{equation}
%
The moment generating function $ \mathbb{M}(\mathbf{\check{X}})$ is then given by
%
\begin{equation}
\mathbb{M}(\mathbf{\check{X}}) 
   = \int \exp\left(-\frac{1}{2}\mathbf{x}^T(\beta\mathbf{D} - \mathbf{\check{X}})\mathbf{x}\right) d\mathbf{x} 
   = (2\pi)^{\frac{M}{2}} \Det{\beta\mathbf{D} - \mathbf{\check{X}}}^{-\frac{1}{2}}  .
\end{equation}

\paragraph{Step 2.5}
The \SaddlePointApproximation (SPA) can be used to solve for $\mathcal{I}(\mathbf{\check{X}})$ 
by solving for the stationary conditions
\begin{equation}
  \frac{\partial}{\partial \mathbf{\check{X}}} I(\mathbf{X},\mathbf{\check{X}}) = 0  .
\end{equation}
%
First, let us compute $ \ln \mathbb{M}(\mathbf{\check{X}}) $ as:
%
\begin{equation}
\ln \mathbb{M}(\mathbf{\check{X}}) = \frac{M}{2} \ln(2\pi) - \frac{1}{2} \ln \Det{\beta\mathbf{D} - \mathbf{\check{X}}}.
\end{equation}
\michael{Make det notation consistent throughout.}
\charles{OK lets double check that everywhere;}
%
Substituting this into the expression for the Legendre transform, we obtain:
%
\begin{equation}
I(\mathbf{X},\mathbf{\check{X}}) 
   = \sup_{\mathbf{\check{X}}} \left[\frac{1}{2} \Trace{\mathbf{X} \mathbf{\check{X}}} - \frac{M}{2} \ln(2\pi) + \frac{1}{2} \ln \Det{\beta\mathbf{D} - \mathbf{\check{X}}} \right].
\end{equation}
%
The supremum of this expression is attained at the value of $\mathbf{\check{X}}$ that satisfies:
%
\begin{equation}
\frac{\partial}{\partial \mathbf{\check{X}}} \left[\frac{1}{2} \Trace{\mathbf{X} \mathbf{\check{X}}} + \frac{1}{2} \ln \Det{\beta\mathbf{D} - \mathbf{\check{X}}} \right] = 0.
\end{equation}
%
Taking the derivative, we obtain
%
\begin{equation}
\frac{1}{2} \mathbf{X} + \frac{1}{2} (\beta\mathbf{D} - \mathbf{\check{X}})^{-1} = 0,
\end{equation}
%
which simplifies to:
%
\begin{equation}
\mathbf{X} = (\beta\mathbf{D} - \mathbf{\check{X}})^{-1} \quad \Rightarrow \quad \mathbf{\check{X}} = \beta\mathbf{D} - \mathbf{X}^{-1}.
\end{equation}
%
Substituting $ \mathbf{\check{X}} = \beta\mathbf{D} - \mathbf{X}^{-1} $ back into the expression for $ I(\mathbf{X})$, we obtain:
%
\begin{equation}
I(\mathbf{X}) = \frac{1}{2} \left[\Trace{\mathbf{X} (\beta\mathbf{D} - \mathbf{X}^{-1})} - \frac{M}{2} \ln(2\pi) + \frac{1}{2} \ln \Det{\mathbf{X}^{-1}}\right].
\end{equation}
%
%
\begin{equation}
\Trace{\mathbf{X}\beta\mathbf{D} - \mathbf{I}} = \Trace{\mathbf{X}\beta\mathbf{D}} - N,
\end{equation}
\begin{equation}
\ln \Det{\mathbf{X}^{-1}} = -\ln \Det{\mathbf{X}},
\end{equation}
we get:
%
\begin{equation}
I(\mathbf{X}) = \frac{1}{2} \left[\Trace{\mathbf{X}\beta\mathbf{D}} - \ln \Det{\mathbf{X}} - M - M\ln(2\pi)\right].
\end{equation}

Finally, we express $I(\mathbf{X})$ in the form:

\begin{equation}
I(\mathbf{X}) = \frac{1}{2} \left[ -M(1 + \ln(2\pi)) + \Trace{\mathbf{X}\beta\mathbf{D}} - \ln \Det{\mathbf{X}} \right].
\end{equation}

\nred{THE DERIVATION ABOVE FOR $\GFANCY$ may have some TYPOS: CHECK}
\paragraph{Step 2.6}
\begin{equation}
  \beta\GFANCY=\mathbb{M}(1+\ln 2\pi)+\beta\Trace{\GNORM(\mathbf{X})} - \Trace{\mathbf{X}\beta\mathbf{D}} +  \ln \Det{\mathbf{X}}.
\end{equation}

We restrict our solution to those where $\XMAT$ and $\beta\mathbf{D}$ can be diagonalized simultaneously.
In particular, this lets us write
\begin{equation}
\Trace{\mathbf{X}\beta\mathbf{D}} = \sum_{\mu=1}^{M}\beta\delta_{\mu}\lambda_{\mu}   ,
\end{equation}
where $\beta\delta_{\mu}$ and $\lambda_{\mu}$ denote the eigenvalues of $\XMAT$ and $\beta\mathbf{D}$, resp.

We can now write the maximum value of $\GNORM$, $\GMAX$, as
\begin{equation}
\label{eqn:gmax_final}
\beta\GMAX = M \left( 1 + \ln \frac{2\pi}{\beta} \right) - \sum_{\mu=1}^{M} \min_{\beta\delta_{\mu}} \left[\beta\delta_{\mu}\lambda_{\mu}
- \beta\GNORM(\lambda_{\mu}) + \ln \lambda_{\mu} \right]   .
\end{equation}


\subsubsection{Expressing the~\GEN~$(\GNORM(\lambda))$ as the Integrated~\RTransform~$(R(z))$ of the~\CorrelationMatrix~$(\AMAT)$}
\label{sxn:tanaka_end}
Having completed both steps, let us combine Eqns.~\ref{eqn:ZD_step1},~\ref{eqn:ln-Z-Nlim2}
with~\ref{eqn:ZD_step2} and~\ref{eqn:gmax_final}.
We follow the first arguments by Tanaka\cite{Tanaka2007} (which follows Cherrier\cite{Cherrier2003}).
\begin{align}
   M\ln(\dfrac{{2}\pi}{\beta})-\sum_{\mu=1}^{M}\int \ln(\beta\delta_{\mu}-\lambda)\rho^{\infty}_{\AMAT}(\lambda)d\lambda
      = M \left( 1 + \ln \frac{2\pi}{\beta} \right) - \sum_{\mu=1}^{M} \min_{\beta\delta_{\mu}} \left[\beta\delta_{\mu}\lambda_{\mu}
      - \beta\GNORM(\lambda_{\mu}) + \ln \lambda_{\mu} \right]   .
\end{align}
By cancelling the $\ln \frac{2\pi}{\beta}$ term from both sides, we obtain
\begin{align}
   -\sum_{\mu=1}^{M}\int \ln(\beta\delta_{\mu}-\lambda)\rho^{\infty}_{\AMAT}(\lambda)d\lambda
   =
   M - \sum_{\mu=1}^{M} \min_{\beta\delta_{\mu}} \left[\beta\delta_{\mu}\lambda_{\mu}
   - \beta\GNORM(\lambda_{\mu}) + \ln \lambda_{\mu} \right]   .
\end{align}
Since this is true for every $\mu$, we can solve this for any arbitrary eigenvalue $\lambda_{\mu}$.
\michael{There must be some other assumption that the $M$ term is spread out uniformly over eigenvalues?}
\charles{Yes. We assumed this earlier in A62.  Why dont you add some comments here}

Dropping the $\mu$ subscript, we have the following identity:
\begin{align}
\label{eqn:concave_id} 
 \min_{\delta} \left[\beta\delta\lambda - \beta\GNORM(\lambda) + \ln \lambda \right]
 = 1 -\int \ln(\beta\delta-\lambda)\rho^{\infty}_{\AMAT}(\lambda)d\lambda   .
\end{align}

We need to invert \ref{eqn:concave_id} in order to find $\beta\GNORM(\lambda)$.
If we choose the eigenvalues of $\mathbf{D}$ such that $\beta\delta_{\mu}>\lambda_{max}$ for all $\mu$, then 
this relation is concave and therefore invertible via a Legendre transform.  

This gives
\begin{equation}
\beta\GNORM(\lambda) = \beta\delta(\lambda) \lambda - \int \ln[\beta\delta(\lambda) - \lambda] \rho^{\infty}_{\AMAT}(\lambda) d\lambda - \ln \lambda - 1  ,
\end{equation}
where we need to  define  $ \beta\delta(\lambda)$, which (not to be confused with the Dirac delta-function), describes
the functional dependence between the eigenvalues of the source matrix $\DMAT$ and the \Student \CorrelationMatrix $\AMAT$.

$\GNORM(\lambda)$ is computed by minimizing over $\delta$, ensuring the relationship holds for the entire spectrum.
So let us take the derivative of $\beta\GNORM$ w/r.t $\lambda$. 
Term by term, this gives:
\begin{equation}
\dfrac{d}{d\lambda} \beta\delta(\lambda)\lambda = \beta\delta(\lambda) + \dfrac{d \beta\delta(\lambda)}{d\lambda} \lambda
\end{equation}
\begin{equation}
\dfrac{d}{d\lambda} \ln\lambda = \dfrac{1}{\lambda}
\end{equation}
\begin{align}
\dfrac{d}{d\lambda} \int \ln[\beta\delta(\lambda) - \lambda] \rho^{\infty}_{\AMAT}(\lambda) d\lambda
&= \int \dfrac{d}{d\lambda} \ln[\beta\delta(\lambda) - \lambda] \rho^{\infty}_{\AMAT}(\lambda) d\lambda \\ \nonumber
&= \int \dfrac{d \beta\delta(\lambda)}{d\lambda}\dfrac{ \rho^{\infty}_{\AMAT}(\lambda)}{\beta\delta(\lambda) - \lambda}d\lambda  \\ \nonumber
&=  \dfrac{d \beta\delta(\lambda)}{d\lambda}\int\dfrac{ \rho^{\infty}_{\AMAT}(\lambda)}{\beta\delta(\lambda) - \lambda}d\lambda
\end{align}

We can now simplify by  defining $\delta(\lambda)$ implicitly by the integral relation
\begin{equation}
\lambda = \int \frac{\rho^{\infty}_{\AMAT}(\lambda)}{\beta\delta(\lambda) - \lambda} d\lambda.
\end{equation}
Combining terms, this gives
\begin{equation}
\frac{d\beta\GNORM(\lambda)}{d\lambda} = \beta\delta(\lambda) - \frac{1}{\lambda},
\end{equation}

 Inverting the derivative, we obtain an integral equation for $\beta\GNORM(\lambda)$ 
\begin{equation}
\beta\GNORM(\lambda) = \int_0^\lambda \left(\beta\delta(z) - \frac{1}{z}\right) dz.
\end{equation}

Notice since  $\beta\delta(\lambda) \approx \frac{1}{\lambda}$ for $\lambda \ll 1$, then
as $\GNORM(0) = 0$ and we set the lower integrand to $0$ (for now).  Even though Tanaka’s original proof assumes an analytic continuation without branch cuts, a heavy-tailed spectrum merely shifts the lower limit of the $R$-transform integral, so the expression for $\beta\mathcal G(\lambda)$ continues to hold.

To further connect these to the \RTransform $R_{\AMAT}(z)$, we recall that the \CauchyStieltjes (or just \Cauchy See~\ref{eqn:Cz}) transform $\mathcal{C}_{\AMAT}(z)$  is given by:

\begin{equation}
\mathcal{C}_{\AMAT}(z) = \int \frac{\rho_{\AMAT}(\lambda)}{z - \lambda} d\lambda.
\end{equation}

The relationship between the \Cauchy transform and the \RTransform is  then expressed as:

\begin{equation}
\mathcal{C}_{\AMAT}\left(R_{\AMAT}(z) + \frac{1}{z}\right) = z,
\end{equation}

which implies:

\begin{equation}
\beta\GNORM(\lambda) = \int_0^{\lambda} R_{\AMAT}(z) dz.
\end{equation}

Although we note that, at least for our purposes, $R(z)$ may and probably
will have a branchcut at the start of the tail of ESD of $\rho_{\AMAT}$,
so we actually want
\begin{equation}
\beta\GNORM(\lambda) = \int_{\LambdaECSmin}^{\lambda} R_{\AMAT}(z) dz.
\end{equation}
where $\LambdaECSmin$ corresponds to the start of the branchcut if necessary.








%\subsubsection{Selecting $\AMAT:=\AMAT_M$ instead of $\AMAT_N$}
\label{sxn:tanaka_end}
In principle, we could have selected $\AMAT:=\AMAT_M=\tfrac{1}{N}\SMAT^{\top}\SMAT$  for the \Student Correlation matrix,
thereby avoiding the discussion on the \DualityOfMeasures altogether.
Doing this, however, would make $\AMAT$ $M\times M$, thereby
require defining the \SourceMatrix $\DMAT$ as an
$N \times N$ matrix, with presumably $N-M$ zero eigenvalues.
This would cause $\DMAT$ to violate the condition $\DeltaMu > \beta\lambda$
for all eigenvalues $\lambda$ of $\AMAT_M$.
In this case, it would be challenging to define the large-$N$ limit.

